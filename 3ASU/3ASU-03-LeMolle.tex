% !TEX encoding = UTF-8 Unicode
\documentclass[a5paper,11pt,twopages]{amsart}
\usepackage[italian]{babel}
\usepackage[utf8]{inputenc}
\usepackage[T1]{fontenc}
%\usepackage[margin=1cm,top=1cm,right=1.5cm]{geometry}
\usepackage[a5paper, twoside, margin=1cm, inner=1.5cm, outer=1cm]{geometry}

%%%%%%%%%%%%%%%%%%%%%%%%%%%%%%%%%%%%%%%%%%%%%%%%%%%%%%%%%%%%%%%
% PACCHETTI
%%%%%%%%%%%%%%%%%%%%%%%%%%%%%%%%%%%%%%%%%%%%%%%%%%%%%%%%%%%%%%%
\usepackage{amssymb}
\usepackage{mathrsfs}
\usepackage{hyperref}
%\usepackage[all]{xy} % For \xymatrix
\usepackage[usenames,dvipsnames]{xcolor}
\hypersetup{colorlinks,%
citecolor=Black,%
filecolor=Black,%
linkcolor=Black,%
urlcolor=Black}
%\usepackage{marvosym}	% Per il fulmine: \Lightning
\usepackage{enumitem}	% Per personalizzare gli elenchi
%\usepackage{booktabs}	% Per i comandi \toprule \midrule \bottomrule , che fanno righe orrizontali nelle tabelle
%\usepackage{array}		% Per formattare meglio le celle nelle tabelle.
\usepackage{mathtools}	% Per \mathclap
\mathtoolsset{showonlyrefs=true} % Per mostrare solo i numeri che servono. With Mathtools: See https://tex.stackexchange.com/questions/4728/how-do-i-number-equations-only-if-they-are-referred-to-in-the-text
%\usepackage[pdftex]{graphicx}	% Per il frontespizio...
%\usepackage{makeidx}	% Per fare l'indice analitico. Già in amsart, amsbook and amsproc. 
%\makeindex
%\allowdisplaybreaks % per avere equazioni su più pagine
\usepackage{tikz}
%
%\usepackage{framed} % for the environment ``framed''. For more complicated things, use \usepackage{mdframed}.
%\usepackage{refcheck} % Mostra i label non usati ``Unused label...'' nel log file. See https://tex.stackexchange.com/questions/209782/get-list-of-unused-labels
\usepackage[normalem]{ulem} % for strikeout: \sout{striketouted} (preferable over 'soul')
%\usepackage{soul} % for strikeout: \st{striketouted} (see 'ulem' package, which is preferable)
\usepackage{siunitx} % For SI units
\usepackage{yhmath} % for \widetriangle


%%%%%%%%%%%%%%%%%%%%%%%%%%%%%%%%%%%%%%%%%%%%%%%%%%%%%%%%%%%%%%%
% COMANDI
%%%%%%%%%%%%%%%%%%%%%%%%%%%%%%%%%%%%%%%%%%%%%%%%%%%%%%%%%%%%%%%
\newcommand{\scr}[1]{\mathscr{#1}}
\newcommand{\frk}[1]{\mathfrak{#1}}
\newcommand{\bb}[1]{\mathbb{#1}}
\newcommand{\cal}[1]{\mathcal{#1}}
%
\newcommand{\N}{\mathbb{N}}	% Numeri naturali
\newcommand{\Z}{\mathbb{Z}}	% Numeri interi
\newcommand{\Q}{\mathbb{Q}}	% Numeri razionali
\newcommand{\R}{\mathbb{R}}	% Numeri reali
  
\newcommand{\sen}{\operatorname{sen}}
\newcommand{\parallelo}{{/\!\!/}}
\newcommand{\pics}{../pics} % cartella delle immagini

%%%%%%%%%%%%%%%%%%%%%%%%%%%%%%%%%%%%%%%%%%%%%%%%%%%%%%%%%%%%%%%
% Subsectioning for notes:
%\renewcommand{\thesubsection}{{\bf\S\arabic{section}.\arabic{subsection}}}
\renewcommand{\thesubsection}{{\bf\arabic{subsection}}}


%%%%%%%%%%%%%%%%%%%%%%%%%%%%%%%%%%%%%%%%%%%%%%%%%%%%%%%%%%%%%%%
%%%%%%%%%%%%%%%%%%%%%%%%%%%%%%%%%%%%%%%%%%%%%%%%%%%%%%%%%%%%%%%
\title{La Legge Di Hooke (3SU)}
\date{\today. \IfFileExists{../.gittex}{\input{../.gittex}}{}}

\begin{document}
%\maketitle
\vspace{-.5cm}
\begin{center}
{\bf LA LEGGE DI HOOKE (3SU)}
\end{center}
%\vspace{-.5cm}
\begin{tabular}{lp{2cm}lp{6cm}}
Classe: & \dotfill & Nome: & \dotfill \\
Data: & \dotfill & Cognome: & \dotfill 
\end{tabular}


%%%%%%%%%%%%%%%%%%%%%%%%%%%%%%%%%%%%%%%%%%%%%%%%%%%%%%%%%%%%%%%
\subsection{Riscaldamento}
\makebox{}

\begin{minipage}[b]{.5\textwidth}
\begin{itemize}[leftmargin=*]
\item $22 - 34 + 81 =$
\item $\frac{4}{7} - \frac{2}{35} =$
\end{itemize}
\end{minipage}
\begin{minipage}[b]{.5\textwidth}
\begin{itemize}[leftmargin=*]
\item $(a-b)(a+b) =$
\item $\frac{ (x-1)^2 - 1 }{ x } =$
\end{itemize}
\end{minipage}


%%%%%%%%%%%%%%%%%%%%%%%%%%%%%%%%%%%%%%%%%%%%%%%%%%%%%%%%%%%%%%%
%%%%%%%%%%%%%%%%%%%%%%%%%%%%%%%%%%%%%%%%%%%%%%%%%%%%%%%%%%%%%%%
%%%%%%%%%%%%%%%%%%%%%%%%%%%%%%%%%%%%%%%%%%%%%%%%%%%%%%%%%%%%%%%
%%%%%%%%%%%%%%%%%%%%%%%%%%%%%%%%%%%%%%%%%%%%%%%%%%%%%%%%%%%%%%%
%%%%%%%%%%%%%%%%%%%%%%%%%%%%%%%%%%%%%%%%%%%%%%%%%%%%%%%%%%%%%%%
\subsection{Esercizio}
Nei seguenti disegni, sono rappresentate delle molle sia a riposo che sotto sforzo.
Nelle immagini delle molle sotto sforzo, disegna gli scarti $\Delta \vec x$ e le forze elastiche $\vec F$ date dalla legge di Hooke.

\begin{center}
\boxed{\includegraphics[width=.20\textwidth]{\pics/statica-07}}
\boxed{\includegraphics[width=.32\textwidth]{\pics/statica-08}}
\boxed{\includegraphics[width=.40\textwidth]{\pics/statica-09}}
\end{center}

%%%%%%%%%%%%%%%%%%%%%%%%%%%%%%%%%%%%%%%%%%%%%%%%%%%%%%%%%%%%%%%
\subsection{Esercizio}
Una molla di costante elastica $k=34\unit{N/m}$ e di lunghezza a riposo $\ell=45\unit{cm}$ è dentro una scatola lunga $30\unit{cm}$.
Quali sono le forze impresse dalla molla sulle due pareti della scatola?


%%%%%%%%%%%%%%%%%%%%%%%%%%%%%%%%%%%%%%%%%%%%%%%%%%%%%%%%%%%%%%%
\subsection{Esercizio}
Appendo un corpo di massa $m=13\unit{g}$ a una molla con costante elastica $k=50\unit{N/m}$, che a sua volta è appesa al soffitto.
La molla a riposo è lunga $\ell = 7\unit{cm}$.
Quanto è lunga la molla quando il corpo è appeso?

%%%%%%%%%%%%%%%%%%%%%%%%%%%%%%%%%%%%%%%%%%%%%%%%%%%%%%%%%%%%%%%
\subsection{Esercizio}\label{par6958e97f}
Lucia e Marco hanno costruito una bilancia casalinga.
Hanno messo una molla in verticale (un tubo la tiene dritta).
In basso la molla appoggia al tavolo, in alto invece c'è una piano su cui appoggiare gli oggetti da pesare.

\begin{enumerate}[label=(\alph*)]
\item
Lucia e Marco devono innanzitutto calcolare la costante elastica della molla.
Come fanno? 
(Discutiamo la risposta in classe.)
%\item
%Quanto vale la costante elastica $k$ della molla?
\item
Trovata la costante elastica $k$ della molla, possono provare a pesare un libro di massa $M$ (sconosciuta).
Lo appoggiano sulla bilancia e misurano che la molla si contrae di $3{,}4\unit{cm}$.
Quanto pesa il libro? Quanto vale $M$?
\end{enumerate}



%\begin{center}
%\includegraphics[width=.45\textwidth]{\pics/statica-10}
%\includegraphics[width=.45\textwidth]{\pics/statica-11}
%\end{center}

%Sì, fanno così: ci appoggiano sopra un corpo di massa $m=100\unit{g}$ e misurano quanto la molla si contrae.
%La loro misura da $\Delta x = 1\unit{cm}$.
%[...] 
%Quanto vale la costante elastica $k$ della molla?
%
%A questo punto possono provare a pesare un libro di massa $M$ (sconosciuta).
%Lo appoggiano sulla bilancia e misurano che la molla si contrae di $3{,}4\unit{cm}$.
%Quanto pesa il libro? Quanto vale $M$?

%%%%%%%%%%%%%%%%%%%%%%%%%%%%%%%%%%%%%%%%%%%%%%%%%%%%%%%%%%%%%%%
\subsection{Esercizio}
Due molle identiche sono messe dentro a una scatola.
La loro costante elastica è $k$ e la loro lunghezza a riposo è $\ell$.
La scatola è lunga $L$ e larga $H$
($k=60\unit{N/m}$, $\ell = 10\unit{cm}$,
$L=15\unit{cm}$, $H=8\unit{cm}$).

Possiamo mettere le due molle in due modi: in serie per il lungo o in parallelo per il largo.
\begin{enumerate}[label=(\alph*)]
\item
Se le mettiamo in serie per il lungo, quali forze esercitano le molle sulle pareti della scatola?
\item
E se le mettiamo in parallelo per il largo?
\end{enumerate}


\begin{center}
\includegraphics[width=.8\textwidth]{\pics/statica-12}
\end{center}

%%%%%%%%%%%%%%%%%%%%%%%%%%%%%%%%%%%%%%%%%%%%%%%%%%%%%%%%%%%%%%%
\begin{minipage}[b]{.5\textwidth}
\subsection{Esercizio}
Torniamo alla bilancia descritta in~\ref{par6958e97f}.
Tra Lucia e Marco nasce una diatriba.
Lucia sostiene che dentro al tubo ci siano due molle, una sopra l'altra, identiche tra loro.
Marco invece sostiene che ci sia una sola molla.
\end{minipage}
\begin{minipage}[t]{.5\textwidth}
\begin{center}
\includegraphics[width=.9\textwidth]{\pics/statica-13}
\end{center}
\end{minipage}
\vspace{-.5cm}
\begin{enumerate}[label=(\alph*)]
\item
Se avesse ragione Lucia, quale sarebbe la costante elastica delle due molle?
\item
C'è un modo, senza aprire il tubo che contiene le molle, di verificare chi dei due ha ragione?
\end{enumerate}

%%%%%%%%%%%%%%%%%%%%%%%%%%%%%%%%%%%%%%%%%%%%%%%%%%%%%%%%%%%%%%%
\subsection{Esercizio}
Due molle di costante elastica $k_1$ e $k_2$ rispettivamente, e di lunghezze a riposo $\ell_1$ e $\ell_2$, rispettivamente, sono appese una dopo l'altra al soffitto, in serie.
Alla seconda molla, più in basso, viene agganciata una massa $m$.
Quanto saranno lunghe le due molle?

%\begin{center}
%\includegraphics[width=.5\textwidth]{\pics/statica-16}
%\end{center}

%%%%%%%%%%%%%%%%%%%%%%%%%%%%%%%%%%%%%%%%%%%%%%%%%%%%%%%%%%%%%%%%
%\subsection{Esercizio}
%Due molle di costante elastica $k_1$ e $k_2$ rispettivamente, e di lunghezze a riposo $\ell_1$ e $\ell_2$, rispettivamente, sono appese a fianco l'altra al soffitto, in parallelo.
%Viene agganciata una massa $m$ alle due
%Quanto saranno lunghe le due molle?
%
%\begin{center}
%\includegraphics[width=.75\textwidth]{\pics/statica-16}
%\end{center}

%%%%%%%%%%%%%%%%%%%%%%%%%%%%%%%%%%%%%%%%%%%%%%%%%%%%%%%%%%%%%%%
\subsection{Esercizio}
Inventa tu un esercizio e risolvilo.

%%%%%%%%%%%%%%%%%%%%%%%%%%%%%%%%%%%%%%%%%%%%%%%%%%%%%%%%%%%%%%%
\subsection{Defaticamento}
\newcommand{\mymod}{\mathrm{mod}\ }
\makebox{}\\

\begin{minipage}[b]{.5\textwidth}
\begin{itemize}[leftmargin=*]
\item $122 \cdot 37 - 199 =$
\item $\frac{x}{11} + \frac{11}{x} =$
\item $\log_{10}(100) =$
\item $11 + 2 \equiv ... \ (\mymod 12)$
\item $5 + 8 \equiv ... \ (\mymod 12)$
\item $5 + 8 \equiv ... \ (\mymod 2)$
\end{itemize}
\end{minipage}
\begin{minipage}[b]{.5\textwidth}
\begin{itemize}[leftmargin=*]
\item $( 5(x+1)^3 - 1 ) \cdot x^{-1} =$
\item $\frac{a}{b} - \frac{b}{a} =$
\item $2^3 \cdot 2^5 =$
\item $a^x \cdot a^y =$
\item $(a^{x+y} a^{-y})^c=$
\item $\lim_{n\to\infty} \frac1n =$
\end{itemize}
\end{minipage}
\begin{itemize}[leftmargin=.8cm]
\item Trova $a,b$ tali che $\frac{2x+1}{x^2-1} = \frac{a}{x-1} + \frac{b}{x+1}$.
\item Trova $n\in\N$ tale che $2^n + 3^n = 13$.
\item Trova $q\in\Z$ e $r\in\{0,1,2\}$ tali che $3q+r = 13$.
\end{itemize}

\let\thefootnote\relax\footnotetext{\today}


%%%%%%%%%%%%%%%%%%%%%%%%%%%%%%%%%%%%%%%%%%%%%%%%%%%%%%%%%%%%%%%
\end{document}
