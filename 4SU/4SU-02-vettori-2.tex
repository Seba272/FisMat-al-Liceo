% !TEX encoding = UTF-8 Unicode
\documentclass[a5paper,11pt]{amsart}
\usepackage[italian]{babel}
\usepackage[utf8]{inputenc}
\usepackage[T1]{fontenc}
\usepackage[margin=1cm,top=.5cm,right=1.5cm]{geometry}

%%%%%%%%%%%%%%%%%%%%%%%%%%%%%%%%%%%%%%%%%%%%%%%%%%%%%%%%%%%%%%%
% PACCHETTI
%%%%%%%%%%%%%%%%%%%%%%%%%%%%%%%%%%%%%%%%%%%%%%%%%%%%%%%%%%%%%%%
\usepackage{amssymb}
\usepackage{mathrsfs}
\usepackage{hyperref}
%\usepackage[all]{xy} % For \xymatrix
\usepackage[usenames,dvipsnames]{xcolor}
\hypersetup{colorlinks,%
citecolor=Black,%
filecolor=Black,%
linkcolor=Black,%
urlcolor=Black}
%\usepackage{marvosym}	% Per il fulmine: \Lightning
\usepackage{enumitem}	% Per personalizzare gli elenchi
%\usepackage{booktabs}	% Per i comandi \toprule \midrule \bottomrule , che fanno righe orrizontali nelle tabelle
%\usepackage{array}		% Per formattare meglio le celle nelle tabelle.
\usepackage{mathtools}	% Per \mathclap
\mathtoolsset{showonlyrefs=true} % Per mostrare solo i numeri che servono. With Mathtools: See https://tex.stackexchange.com/questions/4728/how-do-i-number-equations-only-if-they-are-referred-to-in-the-text
%\usepackage[pdftex]{graphicx}	% Per il frontespizio...
%\usepackage{makeidx}	% Per fare l'indice analitico. Già in amsart, amsbook and amsproc. 
%\makeindex
%\allowdisplaybreaks % per avere equazioni su più pagine
\usepackage{tikz}
%
%\usepackage{framed} % for the environment ``framed''. For more complicated things, use \usepackage{mdframed}.
%\usepackage{refcheck} % Mostra i label non usati ``Unused label...'' nel log file. See https://tex.stackexchange.com/questions/209782/get-list-of-unused-labels
\usepackage[normalem]{ulem} % for strikeout: \sout{striketouted} (preferable over 'soul')
%\usepackage{soul} % for strikeout: \st{striketouted} (see 'ulem' package, which is preferable)
\usepackage{siunitx} % For SI units
\usepackage{yhmath} % for \widetriangle


%%%%%%%%%%%%%%%%%%%%%%%%%%%%%%%%%%%%%%%%%%%%%%%%%%%%%%%%%%%%%%%
% COMANDI
%%%%%%%%%%%%%%%%%%%%%%%%%%%%%%%%%%%%%%%%%%%%%%%%%%%%%%%%%%%%%%%
\newcommand{\scr}[1]{\mathscr{#1}}
\newcommand{\frk}[1]{\mathfrak{#1}}
\newcommand{\bb}[1]{\mathbb{#1}}
\newcommand{\cal}[1]{\mathcal{#1}}
%
\newcommand{\N}{\mathbb{N}}	% Numeri naturali
\newcommand{\Z}{\mathbb{Z}}	% Numeri interi
\newcommand{\Q}{\mathbb{Q}}	% Numeri razionali
\newcommand{\R}{\mathbb{R}}	% Numeri reali
  
\newcommand{\sen}{\operatorname{sen}}
\newcommand{\parallelo}{{/\!\!/}}
\newcommand{\pics}{../pics} % cartella delle immagini

%%%%%%%%%%%%%%%%%%%%%%%%%%%%%%%%%%%%%%%%%%%%%%%%%%%%%%%%%%%%%%%
% Subsectioning for notes:
%\renewcommand{\thesubsection}{{\bf\S\arabic{section}.\arabic{subsection}}}
\renewcommand{\thesubsection}{{\bf\arabic{subsection}}}


%%%%%%%%%%%%%%%%%%%%%%%%%%%%%%%%%%%%%%%%%%%%%%%%%%%%%%%%%%%%%%%
%%%%%%%%%%%%%%%%%%%%%%%%%%%%%%%%%%%%%%%%%%%%%%%%%%%%%%%%%%%%%%%
\title{I vettori (4SU) -- 2}
\date{\today. \IfFileExists{../.gittex}{\input{../.gittex}}{}}

\begin{document}
\maketitle
\vspace{-.5cm}
\begin{tabular}{lp{2cm}lp{6cm}}
Classe: & \dotfill & Nome: & \dotfill \\
Data: & \dotfill & Cognome: & \dotfill 
\end{tabular}


%%%%%%%%%%%%%%%%%%%%%%%%%%%%%%%%%%%%%%%%%%%%%%%%%%%%%%%%%%%%%%%
\subsection{Riscaldamento}
\makebox{}

\begin{minipage}[b]{.5\textwidth}
\begin{itemize}[leftmargin=*]
\item $34 + 21 - 18 =$
\item $\frac{22}{11} + \frac{3}{22} =$
\end{itemize}
\end{minipage}
\begin{minipage}[b]{.5\textwidth}
\begin{itemize}[leftmargin=*]
\item $(x+y)(x-y) =$
\item $\frac{x^2+2xy+y^2}{x^2-y^2} =$
\end{itemize}
\end{minipage}


%%%%%%%%%%%%%%%%%%%%%%%%%%%%%%%%%%%%%%%%%%%%%%%%%%%%%%%%%%%%%%%
%%%%%%%%%%%%%%%%%%%%%%%%%%%%%%%%%%%%%%%%%%%%%%%%%%%%%%%%%%%%%%%
%%%%%%%%%%%%%%%%%%%%%%%%%%%%%%%%%%%%%%%%%%%%%%%%%%%%%%%%%%%%%%%
%%%%%%%%%%%%%%%%%%%%%%%%%%%%%%%%%%%%%%%%%%%%%%%%%%%%%%%%%%%%%%%
%\subsection{Esercizio}\label{par6957e1d8}
%Disegna i vettori con le seguenti coordinate:
%$\vec v=(1,1)$, $\vec w=(0,3)$, $\vec u=(2,-1)$, $\vec k=(-1,-1)$.
%
%Poi, disegna tu tre vettori e determinane le coordinate.
%
%%%%%%%%%%%%%%%%%%%%%%%%%%%%%%%%%%%%%%%%%%%%%%%%%%%%%%%%%%%%%%%%
%\subsection{Esercizio}
%Disegna i vettori $\vec v$, $\vec w$, $\vec u$ e $\vec k$ dall'Esercizio~\ref{par6957e1d8} applicati al punto $P=(2,3)$.
%
%%%%%%%%%%%%%%%%%%%%%%%%%%%%%%%%%%%%%%%%%%%%%%%%%%%%%%%%%%%%%%%%
%\subsection{Esercizio}
%Si prendano i vettori dall'Esercizio~\ref{par6957e1d8} e si facciano almeno tre somme, sia con il disegno che con le coordinate.
%
%%%%%%%%%%%%%%%%%%%%%%%%%%%%%%%%%%%%%%%%%%%%%%%%%%%%%%%%%%%%%%%%
%\subsection{Esercizio}
%Si prendano tre numeri reali e tre vettori in coordinate e si facciano almeno tre moltiplicazioni scalare per vettore.

%%%%%%%%%%%%%%%%%%%%%%%%%%%%%%%%%%%%%%%%%%%%%%%%%%%%%%%%%%%%%%%
\subsection{Esercizio}\label{par695d2163}
Disegna i seguenti vettori e calcolane il modulo:
$\vec v=(1,1)$, $\vec w=(0,3)$, $\vec u=(2,-1)$, $\vec k=(-1,-1)$.

%%%%%%%%%%%%%%%%%%%%%%%%%%%%%%%%%%%%%%%%%%%%%%%%%%%%%%%%%%%%%%%
\subsection{Esercizio}\label{par695d1af2}
Io ti do due vettori:
\begin{equation}
\begin{aligned}
	\vec e_1 &= (1,0) , &
	\vec e_2 &= (0,1) .
\end{aligned}
\end{equation}
Trova due scalari $a,b\in\R$ tali per cui
\begin{equation}
	a \vec e_1 + b \vec e_2 = (4,-5) .
\end{equation}

%%%%%%%%%%%%%%%%%%%%%%%%%%%%%%%%%%%%%%%%%%%%%%%%%%%%%%%%%%%%%%%
\subsection{Esercizio}
Io ti do due vettori:
\begin{equation}
\begin{aligned}
	\vec v &= (2,1) , &
	\vec w &= (-1,2) .
\end{aligned}
\end{equation}
Trova due scalari $a,b\in\R$ tali per cui
\begin{equation}
	a \vec v + b \vec w = (5,-2) .
\end{equation}

%%%%%%%%%%%%%%%%%%%%%%%%%%%%%%%%%%%%%%%%%%%%%%%%%%%%%%%%%%%%%%%
\subsection{Esercizio}
Supponi che $\vec v$ sia un vettore non nullo.
Trova $t\in\R$ e un versore $\hat u$ tale che $\vec v = t \hat u$.

Scrivi i quattro vettori dell'Esercizio~\ref{par695d2163} come ``scalare per versore''.

%%%%%%%%%%%%%%%%%%%%%%%%%%%%%%%%%%%%%%%%%%%%%%%%%%%%%%%%%%%%%%%
\subsection{Esercizio}
Considera i due vettori $\vec e_1$ e $\vec e_2$ dell'Esercizio~\ref{par695d1af2}.
Sono dei versori?

Per quali $a\in\R$ e $b\in\R$ il vettore $a\vec e_1 + b\vec e_2$ è un versore?
Fai un disegno di tutti questi versori nel piano cartesiano.

%%%%%%%%%%%%%%%%%%%%%%%%%%%%%%%%%%%%%%%%%%%%%%%%%%%%%%%%%%%%%%%
\subsection{Defaticamento}
\makebox{}

\begin{minipage}[b]{.5\textwidth}
\begin{itemize}[leftmargin=*]
\item $22+4\cdot 13 - 124 =$
\item $\frac{22}{11} + \frac{3}{22} - \frac{1}{11} =$
\end{itemize}
\end{minipage}
\begin{minipage}[b]{.5\textwidth}
\begin{itemize}[leftmargin=*]
\item $a(x+y)^2 =$
\item $\frac{b+1}{a} + \frac{b}{a^2} =$
\end{itemize}
\end{minipage}

%%%%%%%%%%%%%%%%%%%%%%%%%%%%%%%%%%%%%%%%%%%%%%%%%%%%%%%%%%%%%%%
\end{document}
