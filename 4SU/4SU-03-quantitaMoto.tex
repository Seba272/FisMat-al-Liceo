% !TEX encoding = UTF-8 Unicode
\documentclass[a5paper,11pt,twopages]{amsart}
\usepackage[italian]{babel}
\usepackage[utf8]{inputenc}
\usepackage[T1]{fontenc}
%\usepackage[margin=1cm,top=1cm,right=1.5cm]{geometry}
\usepackage[a5paper, twoside, margin=1cm, inner=1.5cm, outer=1.5cm]{geometry}

%%%%%%%%%%%%%%%%%%%%%%%%%%%%%%%%%%%%%%%%%%%%%%%%%%%%%%%%%%%%%%%
% PACCHETTI
%%%%%%%%%%%%%%%%%%%%%%%%%%%%%%%%%%%%%%%%%%%%%%%%%%%%%%%%%%%%%%%
\usepackage{amssymb}
\usepackage{mathrsfs}
\usepackage{hyperref}
%\usepackage[all]{xy} % For \xymatrix
\usepackage[usenames,dvipsnames]{xcolor}
\hypersetup{colorlinks,%
citecolor=Black,%
filecolor=Black,%
linkcolor=Black,%
urlcolor=Black}
%\usepackage{marvosym}	% Per il fulmine: \Lightning
\usepackage{enumitem}	% Per personalizzare gli elenchi
%\usepackage{booktabs}	% Per i comandi \toprule \midrule \bottomrule , che fanno righe orrizontali nelle tabelle
%\usepackage{array}		% Per formattare meglio le celle nelle tabelle.
\usepackage{mathtools}	% Per \mathclap
\mathtoolsset{showonlyrefs=true} % Per mostrare solo i numeri che servono. With Mathtools: See https://tex.stackexchange.com/questions/4728/how-do-i-number-equations-only-if-they-are-referred-to-in-the-text
%\usepackage[pdftex]{graphicx}	% Per il frontespizio...
%\usepackage{makeidx}	% Per fare l'indice analitico. Già in amsart, amsbook and amsproc. 
%\makeindex
%\allowdisplaybreaks % per avere equazioni su più pagine
\usepackage{tikz}
%
%\usepackage{framed} % for the environment ``framed''. For more complicated things, use \usepackage{mdframed}.
%\usepackage{refcheck} % Mostra i label non usati ``Unused label...'' nel log file. See https://tex.stackexchange.com/questions/209782/get-list-of-unused-labels
\usepackage[normalem]{ulem} % for strikeout: \sout{striketouted} (preferable over 'soul')
%\usepackage{soul} % for strikeout: \st{striketouted} (see 'ulem' package, which is preferable)
\usepackage{siunitx} % For SI units
\usepackage{yhmath} % for \widetriangle


%%%%%%%%%%%%%%%%%%%%%%%%%%%%%%%%%%%%%%%%%%%%%%%%%%%%%%%%%%%%%%%
% COMANDI
%%%%%%%%%%%%%%%%%%%%%%%%%%%%%%%%%%%%%%%%%%%%%%%%%%%%%%%%%%%%%%%
\newcommand{\scr}[1]{\mathscr{#1}}
\newcommand{\frk}[1]{\mathfrak{#1}}
\newcommand{\bb}[1]{\mathbb{#1}}
\newcommand{\cal}[1]{\mathcal{#1}}
%
\newcommand{\N}{\mathbb{N}}	% Numeri naturali
\newcommand{\Z}{\mathbb{Z}}	% Numeri interi
\newcommand{\Q}{\mathbb{Q}}	% Numeri razionali
\newcommand{\R}{\mathbb{R}}	% Numeri reali
  
\newcommand{\sen}{\operatorname{sen}}
\newcommand{\parallelo}{{/\!\!/}}
\newcommand{\pics}{../pics} % cartella delle immagini

%%%%%%%%%%%%%%%%%%%%%%%%%%%%%%%%%%%%%%%%%%%%%%%%%%%%%%%%%%%%%%%
% Subsectioning for notes:
%\renewcommand{\thesubsection}{{\bf\S\arabic{section}.\arabic{subsection}}}
\renewcommand{\thesubsection}{{\bf\arabic{subsection}}}


%%%%%%%%%%%%%%%%%%%%%%%%%%%%%%%%%%%%%%%%%%%%%%%%%%%%%%%%%%%%%%%
%%%%%%%%%%%%%%%%%%%%%%%%%%%%%%%%%%%%%%%%%%%%%%%%%%%%%%%%%%%%%%%
\title{La conservazione della quantità di moto (4SU)}
\date{\today. \IfFileExists{../.gittex}{\input{../.gittex}}{}}

\begin{document}
%\maketitle
\vspace{-.5cm}
\begin{center}
{\bf LA CONSERVAZIONE DELLA QUANTITÀ DI MOTO (4SU)}
\end{center}
%\vspace{-.5cm}
\begin{tabular}{lp{2cm}lp{6cm}}
Classe: & \dotfill & Nome: & \dotfill \\
Data: & \dotfill & Cognome: & \dotfill 
\end{tabular}


%%%%%%%%%%%%%%%%%%%%%%%%%%%%%%%%%%%%%%%%%%%%%%%%%%%%%%%%%%%%%%%
\subsection{Riscaldamento}
\makebox{}

\begin{minipage}[b]{.5\textwidth}
\begin{itemize}[leftmargin=*]
\item $22 - 34 + 81 =$
\item $\frac{4}{7} - \frac{2}{35} =$
\end{itemize}
\end{minipage}
\begin{minipage}[b]{.5\textwidth}
\begin{itemize}[leftmargin=*]
\item $(a-b)(a+b) =$
\item $\frac{ (x-1)^2 - 1 }{ x } =$
\end{itemize}
\end{minipage}


%%%%%%%%%%%%%%%%%%%%%%%%%%%%%%%%%%%%%%%%%%%%%%%%%%%%%%%%%%%%%%%
%%%%%%%%%%%%%%%%%%%%%%%%%%%%%%%%%%%%%%%%%%%%%%%%%%%%%%%%%%%%%%%
%%%%%%%%%%%%%%%%%%%%%%%%%%%%%%%%%%%%%%%%%%%%%%%%%%%%%%%%%%%%%%%
\subsection{Esercizio}
Due carrellini di massa $m$ ciascuno sono tenuti fermi su dei binari a una distanza $L$ l'uno dall'altro.
Tra loro c'è una molla compressa.
%La lunghezza a riposo della molla è $\ell$ e la costante elastica della molla è $k_e$.
I due carrellini vengono lasciati andare: se un carrellino raggiunge velocità $v_1$, quale sarà la velocità raggiunta dal secondo carrellino?
($m=1{,}7\,\unit{Kg}$, $L=13\,\unit{cm}$, $v_1=1{,}3\,\unit{m/s}$).

\begin{center}
\includegraphics[width=.75\textwidth]{\pics/quantMoto-04}
\end{center}

%%%%%%%%%%%%%%%%%%%%%%%%%%%%%%%%%%%%%%%%%%%%%%%%%%%%%%%%%%%%%%%
\subsection{Esercizio}
Due carrellini di massa $m_1$ e $m_2$ rispettivamente sono tenuti fermi su dei binari a una distanza $L$ l'uno dall'altro.
Tra loro c'è una molla compressa.
%La lunghezza a riposo della molla è $\ell$ e la costante elastica della molla è $k_e$.
I due carrellini vengono lasciati andare: se un carrellino raggiunge velocità $v_1$, quale sarà la velocità raggiunta dal secondo carrellino?
($m_1=1{,}7\,\unit{Kg}$, $m_2=0{,}7\,\unit{Kg}$, $L=13\,\unit{cm}$, $v_1=1{,}3\,\unit{m/s}$).

%%%%%%%%%%%%%%%%%%%%%%%%%%%%%%%%%%%%%%%%%%%%%%%%%%%%%%%%%%%%%%%
\subsection{Esercizio}
Un carellino di massa $M$ sta fermo su dei binari.
Sul carrellino è montata uno scivolo di altezza $h$ e in cima allo scivolo sta una pallina di massa $m$.
La pallina rotola giù dallo scivolo e lascia il carrellino con velocità perfettamente orizzontale.
%Sapendo che la massa del carrellino (scivolo incluso) è $M$ e la massa della pallina è $m$, e sapendo che l'altezza dello scivolo è $h$
Che velocità raggiungono carrellino ($v_M$) e pallina ($v_m$)?
($M=1{,}7\,\unit{Kg}$, $m_2=0{,}7\,\unit{Kg}$, $h=72\,\unit{cm}$).

\begin{center}
\includegraphics[width=.4\textwidth]{\pics/quantMoto-05}
\includegraphics[width=.4\textwidth]{\pics/quantMoto-06}
\end{center}

%%%%%%%%%%%%%%%%%%%%%%%%%%%%%%%%%%%%%%%%%%%%%%%%%%%%%%%%%%%%%%%
\subsection{Esercizio}
Due carrellini di massa $m_1$ e $m_2$ rispettivamente sono tenuti fermi su dei binari a una distanza $L$ l'uno dall'altro.
Tra loro c'è una molla compressa.
La lunghezza a riposo della molla è $\ell$ e la costante elastica della molla è $k_e$.
I due carrellini vengono lasciati andare: quale sarà la velocità raggiunta dai due carrellini?
($m_1=1{,}7\,\unit{Kg}$, $m_2=0{,}7\,\unit{Kg}$, $L=13\,\unit{cm}$, $\ell = 24\,\unit{cm}$, $k_e = 814{,}07\,\unit{N/m}$).




%%%%%%%%%%%%%%%%%%%%%%%%%%%%%%%%%%%%%%%%%%%%%%%%%%%%%%%%%%%%%%%%
%\subsection{Esercizio}
%Rhtü è un alieno con un'astronave a molla.
%L'astronave con Rhtü e tutto quanto ha massa totale $M$ e
%sta viaggiando a velocità $\vec v$.
%Per frenare, Rhtü lancia un proiettile nella direzione di $\vec v$.
%\begin{enumerate}
%\item	Se il proiettile ha massa $m$, a che velocità deve essere lanciato perché Rhtü si fermi?
%\item	Se il proiettile viene lanciato usando una molla di costante elastica $k_e$, quanto questa molla deve essere compressa per lanciare il proiettile?
%\end{enumerate}
%
%\begin{center}
%\includegraphics[width=.7\textwidth]{\pics/quantMoto-07}
%\end{center}

%%%%%%%%%%%%%%%%%%%%%%%%%%%%%%%%%%%%%%%%%%%%%%%%%%%%%%%%%%%%%%%
\subsection{Esercizio}
Annamaria vuole misurare la velocità di un proiettile quando esce da un fucile.
%Siccome il proiettile viaggia molto veloce e la sua velocità è difficile da misurare,
%decide di costruire il seguente marchingegno.
Mette il fucile su un carrello di massa $M$ molto grande (compresa la massa del fucile).
Quindi, con il carrello fermo ma libero di muoversi, fa sparare il fucile parallelamente ai binari del carrello e misura la velocità $V$ del carrello dopo lo sparo.
Se il proiettile ha massa $m$, quanto veloce è stato sparato?
($M=34\unit{kg}$, $V = 0{,}3 \unit{m/s}$, $m=53\unit{g}$).

%%%%%%%%%%%%%%%%%%%%%%%%%%%%%%%%%%%%%%%%%%%%%%%%%%%%%%%%%%%%%%%
\subsection{Esercizio}
Il \emph{Saturn V} portò i primi uomini sulla Luna il 16 luglio 1969.
Alla partenza (\emph{liftoff} in inglese), questo razzo
aveva una massa totale di $3\,038\,500\, \unit{kg}$
e consumava circa 13 tonnellate di carburante al secondo.
Il carburante bruciando si espandeva e usciva dall'ugello (\emph{nozzle} in inglese) a forte velocità.
La spinta verso l'alto era la reazione all'accelerazione della massa del carburante verso il basso.
Infatti, la variazione di quantità di moto del carburante (da fermo a velocità $v$ verso il basso) provoca una spinta verso l'alto che deve essere almeno pari al peso del razzo.
A che velocità esce il gas dall'ugello?



%%%%%%%%%%%%%%%%%%%%%%%%%%%%%%%%%%%%%%%%%%%%%%%%%%%%%%%%%%%%%%%
\subsection{Esercizio}
Inventa tu un esercizio e risolvilo.

%%%%%%%%%%%%%%%%%%%%%%%%%%%%%%%%%%%%%%%%%%%%%%%%%%%%%%%%%%%%%%%
\subsection{Defaticamento}
\newcommand{\mymod}{\mathrm{mod}\ }
\makebox{}\\

\begin{minipage}[b]{.5\textwidth}
\begin{itemize}[leftmargin=*]
\item $122 \cdot 37 - 199 =$
\item $\frac{x}{11} + \frac{11}{x} =$
\item $\log_{10}(100) =$
\item $11 + 2 \equiv ... \ (\mymod 12)$
\item $5 + 8 \equiv ... \ (\mymod 12)$
\item $5 + 8 \equiv ... \ (\mymod 2)$
\end{itemize}
\end{minipage}
\begin{minipage}[b]{.5\textwidth}
\begin{itemize}[leftmargin=*]
\item $( 5(x+1)^3 - 1 ) \cdot x^{-1} =$
\item $\frac{a}{b} - \frac{b}{a} =$
\item $2^3 \cdot 2^5 =$
\item $a^x \cdot a^y =$
\item $(a^{x+y} a^{-y})^c=$
\item $\lim_{n\to\infty} \frac1n =$
\end{itemize}
\end{minipage}
\begin{itemize}[leftmargin=.8cm]
\item Trova $a,b$ tali che $\frac{2x+1}{x^2-1} = \frac{a}{x-1} + \frac{b}{x+1}$.
\item Trova $n\in\N$ tale che $2^n + 3^n = 13$.
\item Trova $q\in\Z$ e $r\in\{0,1,2\}$ tali che $3q+r = 13$.
\end{itemize}

\let\thefootnote\relax\footnotetext{\today}


%%%%%%%%%%%%%%%%%%%%%%%%%%%%%%%%%%%%%%%%%%%%%%%%%%%%%%%%%%%%%%%
\end{document}
