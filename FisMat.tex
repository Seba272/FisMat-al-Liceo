% !TEX encoding = UTF-8 Unicode
\documentclass[a4paper,12pt]{amsart}
\usepackage[italian]{babel}
\usepackage[utf8]{inputenc}
\usepackage[T1]{fontenc}
%\usepackage{cmbright}
%\usepackage[sfdefault]{carlito}
%\setmainfont{calibri}
%For Japanese kanji and kana: but Typeset with XeLaTeX !
%\usepackage{xeCJK}
%\setCJKmainfont{MS Mincho} % for \rmfamily
%\setCJKsansfont{MS Gothic} % for \sffamily
\usepackage{csquotes}
\usepackage[style=numeric,%alphabetic,%
	useprefix,%
	giveninits=true,%
	hyperref,%
	doi=false,%
	url=false,%
	isbn=false,%
	backend=bibtex,%
	maxbibnames=99%
	]{biblatex}
\bibliography{./BIB}

%%%%%%%%%%%%%%%%%%%%%%%%%%%%%%%%%%%%%%%%%%%%%%%%%%%%%%%%%%%%%%%
% PACCHETTI
%%%%%%%%%%%%%%%%%%%%%%%%%%%%%%%%%%%%%%%%%%%%%%%%%%%%%%%%%%%%%%%
\usepackage{amssymb}
\usepackage{mathrsfs}
\usepackage{hyperref}
%\usepackage[all]{xy} % For \xymatrix
\usepackage[usenames,dvipsnames]{xcolor}
\hypersetup{colorlinks,%
citecolor=Black,%
filecolor=Black,%
linkcolor=Black,%
urlcolor=Black}
%\hypersetup{colorlinks,%
%citecolor=OliveGreen,%
%filecolor=magenta,%
%linkcolor=RoyalBlue,%
%urlcolor=cyan}
%\usepackage{mparhack}	% Per correggere le note a margine, che a volte vengono dalla parte sbagliata.
%\usepackage{todonotes} % Per le note a margine in stile Thomas Z.
%\usepackage{comment}	% Per i commenti lunghi: \begin{comment} ... \end{comment}
%\usepackage{marvosym}	% Per il fulmine: \Lightning
\usepackage{enumitem}	% Per personalizzare gli elenchi
%\usepackage{booktabs}	% Per i comandi \toprule \midrule \bottomrule , che fanno righe orrizontali nelle tabelle
%\usepackage{array}		% Per formattare meglio le celle nelle tabelle.
\usepackage{mathtools}	% Per \mathclap
\mathtoolsset{showonlyrefs=true} % Per mostrare solo i numeri che servono. With Mathtools: See https://tex.stackexchange.com/questions/4728/how-do-i-number-equations-only-if-they-are-referred-to-in-the-text
%\usepackage[pdftex]{graphicx}	% Per il frontespizio...
%\usepackage{makeidx}	% Per fare l'indice analitico. Già in amsart, amsbook and amsproc. 
%\makeindex
%\allowdisplaybreaks % per avere equazioni su più pagine
\usepackage{tikz}
%
%\usepackage{framed} % for the environment ``framed''. For more complicated things, use \usepackage{mdframed}.
%\usepackage{refcheck} % Mostra i label non usati ``Unused label...'' nel log file. See https://tex.stackexchange.com/questions/209782/get-list-of-unused-labels
\usepackage[normalem]{ulem} % for strikeout: \sout{striketouted} (preferable over 'soul')
%\usepackage{soul} % for strikeout: \st{striketouted} (see 'ulem' package, which is preferable)
\usepackage{siunitx} % For SI units
\usepackage{yhmath} % for \widetriangle


%%%%%%%%%%%%%%%%%%%%%%%%%%%%%%%%%%%%%%%%%%%%%%%%%%%%%%%%%%%%%%%
% COMANDI
%%%%%%%%%%%%%%%%%%%%%%%%%%%%%%%%%%%%%%%%%%%%%%%%%%%%%%%%%%%%%%%
\newcommand{\scr}[1]{\mathscr{#1}}
\newcommand{\frk}[1]{\mathfrak{#1}}
\newcommand{\bb}[1]{\mathbb{#1}}
\newcommand{\cal}[1]{\mathcal{#1}}
%
\newcommand{\N}{\mathbb{N}}	% Numeri naturali
\newcommand{\Z}{\mathbb{Z}}	% Numeri interi
\newcommand{\Q}{\mathbb{Q}}	% Numeri razionali
\newcommand{\R}{\mathbb{R}}	% Numeri reali
\newcommand{\C}{\mathbb{C}}	% Numeri complessi
%\newcommand{\K}{\mathbb{K}}	% Campo di numeri
%\newcommand{\Id}{\mathrm{Id}}	% Mappa identità
%\newcommand{\Span}{\mathrm{span}}	% Span
%\newcommand{\Ker}{\mathrm{Ker}}	% Ker
\newcommand{\diff}{\mathrm{d}}	% 'd' di derivata
\newcommand{\did}{\,\mathrm{d}} % 'd' di derivata, con spazio prima, per integrale.
\newcommand{\de}{\partial}		% Derivata parziale
\renewcommand{\div}{\operatorname{div}}	% div di Divergenza.
%\newcommand{\into}{\hookrightarrow}		% (->  %)
%\newcommand{\dto}{\dashrightarrow}
\newcommand{\ol}[1]{\overline{#1}}	% overline breve
%\DeclareMathOperator{\arctanh}{arctanh}
\newcommand{\IF}{\Leftarrow}	% <=
\newcommand{\THEN}{\Rightarrow}	% =>
\newcommand{\IFF}{\Leftrightarrow}	% <=>
\newcommand{\ddx}{\framebox[\width]{$\Rightarrow$} }
\newcommand{\ssx}{\framebox[\width]{$\Leftarrow$} }
\newcommand{\ci}{\framebox[\width]{$\subset$} }
\newcommand{\ic}{\framebox[\width]{$\supset$} }
\newcommand{\lebox}{\framebox[\width]{$\le$} }
\newcommand{\gebox}{\framebox[\width]{$\ge$} }
%\renewcommand{\Re}{\operatorname{Re}}
%\renewcommand{\Im}{\operatorname{Im}}
% \hel is the left-lower corner for restriction of measures.
\newcommand{\hel}
{
\hskip2.5pt{\vrule height6pt width.5pt depth0pt}
\hskip-.2pt\vbox{\hrule height.5pt width6pt depth0pt}
\,
}
\newcommand{\Lin}{\mathtt{Lin}}
\newcommand{\End}{\mathtt{End}}
\newcommand{\J}{\mathtt{J}}
\newcommand{\ad}{\mathrm{ad}}
\newcommand{\Sym}{\mathtt{Sym}}
\newcommand{\jet}{\mathfrak{j}}
\newcommand{\Jet}{\mathtt{J}}
\newcommand{\HorDer}{\mathtt{HD}}
\newcommand{\bbar}[1]{\overline{\overline{#1}}}
\newcommand{\Der}{\mathtt{Der}}
\newcommand{\Aut}{\mathtt{Aut}}
\newcommand{\Poly}{\scr P}
\newcommand{\UniEnvAlg}{\scr U}
\newcommand{\Tensor}{\scr T}
\newcommand{\HH}{\bb H}
\newcommand{\Ad}{\operatorname{Ad}}
\newcommand{\w}{\mathbf{w}}
\newcommand{\Haus}{\scr H}
\newcommand{\dist}{\mathrm{dist}}
\newcommand{\spt}{\mathrm{spt}}
\newcommand{\abs}[1]{\left\lvert #1 \right\rvert}
\newcommand{\norm}[1]{\left\lVert #1 \right\rVert}
\newcommand{\nnorm}[1]{\left\lvert\!\left\lvert\!\left\lvert #1 \right\rvert\!\right\rvert\!\right\rvert}

\newcommand{\lcontr}{\lrcorner}
\newcommand{\rcontr}{\raisebox{\depth}{\scalebox{1}[-1]{$\lrcorner$}}} % serve graphicx

% Per \one, la mappa che vale 1
\usepackage{dsfont}
\newcommand{\one}{{\mathds 1\!}}

% Mean integral
\def\Xint#1{\mathchoice
      {\XXint\displaystyle\textstyle{#1}}%
      {\XXint\textstyle\scriptstyle{#1}}%
      {\XXint\scriptstyle\scriptscriptstyle{#1}}%
      {\XXint\scriptscriptstyle\scriptscriptstyle{#1}}%
      \!\int}
   \def\XXint#1#2#3{{\setbox0=\hbox{$#1{#2#3}{\int}$}
        \vcenter{\hbox{$#2#3$}}\kern-.5\wd0}}
%   \def\ddashint{\Xint=}
%   \def\dashint{\Xint-}
   \def\fint{\Xint-}
  
\newcommand{\sen}{\operatorname{sen}}
\newcommand{\parallelo}{{/\!\!/}}
\newcommand{\pics}{pics} % Ma in ogni file viene ridefinito

%%%%%%%%%%%%%%%%%%%%%%%%%%%%%%%%%%%%%%%%%%%%%%%%%%%%%%%%%%%%%%%
% TEOREMI
%%%%%%%%%%%%%%%%%%%%%%%%%%%%%%%%%%%%%%%%%%%%%%%%%%%%%%%%%%%%%%%
\theoremstyle{plain}
\newtheorem{proposition}{Proposition}[section]
\newtheorem{theorem}[proposition]{Theorem}
\newtheorem{lemma}[proposition]{Lemma}
\newtheorem{corollary}[proposition]{Corollary}
\newtheorem{thm}{Theorem}[section]
\renewcommand{\thethm}{\Alph{thm}}
%\newtheorem{question}{Question}

\theoremstyle{definition}
\newtheorem{definition}[proposition]{Definition}
\newtheorem{remark}[proposition]{Remark}
%\declaretheorem[name=Remark,sibling=proposition,qed={\lower-0.3ex\hbox{$\blacklozenge$}}]{remark} % but with \usepackage{thmtools}

\theoremstyle{remark}

%%%%%%%%%%%%%%%%%%%%%%%%%%%%%%%%%%%%%%%%%%%%%%%%%%%%%%%%%%%%%%%
% Subsectioning for notes:
\renewcommand{\thesubsection}{{\bf\S\arabic{section}.\arabic{subsection}}}


%%%%%%%%%%%%%%%%%%%%%%%%%%%%%%%%%%%%%%%%%%%%%%%%%%%%%%%%%%%%%%%

\title[Matematica e Fisica]{Appunti di Matematica e Fisica per le scuole superiori}
\author[Nicolussi Golo]{Sebastiano Nicolussi Golo}
%\date{\today}
\date{\today. \IfFileExists{./.gittex}{\input{./.gittex}}{}}

\begin{document}
\maketitle

\setcounter{tocdepth}{2}
\phantomsection
\addcontentsline{toc}{section}{Contents}
\tableofcontents

%%%%%%%%%%%%%%%%%%%%%%%%%%%%%%%%%%%%%%%%%%%%%%%%%%%%%%%%%%%%%%%
%%%%%%%%%%%%%%%%%%%%%%%%%%%%%%%%%%%%%%%%%%%%%%%%%%%%%%%%%%%%%%%

% !TEX encoding = UTF-8 Unicode
%!TEX root = FisMat.tex

\section{I vettori}
I vettori sembrano sconvolgere le studentesse e gli studenti a tal punto che qualcuno vuole smettere di studiare matematica e fisica.
Prima di fare scelte drastiche di cui potreste pentirvi in futuro, dovete leggere questa frase:
{\it un vettore è una freccia.}

Mi direte che la sto facendo troppo facile, ma fidatevi. 
I vettori sono frecce, e noi li useremo come tali.
Nelle prossime pagine cercheremo esaminare attentamente i vettori, introdurremo un gergo tecnico appropriato, ma non ci allontaneremmo dal concetto di freccia.
Questo lavoro serve solo ad affinare la nostra consapevolezza sulle frecce: adesso un vettore è solo una freccia, ma per la fine di questo modulo i vettori saranno frecce -- consapevolmente.

Questi sono dei vettori:

\usetikzlibrary{arrows.meta}
\tikzset{>=Latex} % Set tip of arrows as black triangle
\usetikzlibrary{decorations.pathreplacing}

\begin{center}
\begin{tikzpicture}
	\draw[->] (0,0) -- (0,2);
	\draw[->] (0,0) -- (1,3);
	\draw[->] (4,2) -- (2,1);
	\draw[->] (4,1) -- (5,3);
\end{tikzpicture}
\end{center}

%%%%%%%%%%%%%%%%%%%%%%%%%%%%%%%%%%%%%%%%%%%%%%%%%%%%%%%%%%%%%%%
\subsection{Il vettore}
Un \emph{vettore}, in quanto freccia, porta con se tre informazioni:
\begin{enumerate}
\item	La \emph{direzione}, ossia la retta su cui giace la freccia. Attenzione, due rette parallele sono la stessa direzione.
\item	Il \emph{verso}: data la retta, il vettore (la freccia) può puntare in due direzioni opposte -- di là o di qua.
\item	Il \emph{modulo}, ossia la lunghezza del vettore. Il modulo di un vettore $v$ si denota con $\norm{v}$.
\end{enumerate}

Un vettore, in quanto freccia, ha due punti speciali:
\begin{enumerate}
\item	la \emph{coda} è il punto da cui il vettore parte,
\item	la \emph{punta} è dove il vettore arriva.
\end{enumerate}

\begin{center}
\includegraphics[width=.75\textwidth]{pics/vettori-01}
%\begin{tikzpicture}
%	\draw[->] (0,0) -- (5,0);
%	\draw[decorate, decoration={brace, amplitude=10pt, raise=5pt, mirror}] 
%    (0,0) -- (5,0) 
%    node[midway, below=15pt] {modulo};
%\end{tikzpicture}
\end{center}

%%%%%%%%%%%%%%%%%%%%%%%%%%%%%%%%%%%%%%%%%%%%%%%%%%%%%%%%%%%%%%%
\subsection{Esercizio}\label{par6957d6fa}
Tra i seguenti vettori:
\begin{itemize}
\item	quali hanno la stessa direzione?
\item	quali hanno la stessa direzione e lo stesso verso?
\item	quali hanno lo stesso modulo?
\end{itemize}

Quindi, quali delle seguenti frecce sono lo stesso vettore?

\begin{center}
\includegraphics[width=.75\textwidth]{pics/vettori-02}
\end{center}

%%%%%%%%%%%%%%%%%%%%%%%%%%%%%%%%%%%%%%%%%%%%%%%%%%%%%%%%%%%%%%%
\subsection{Usanze di nome}
Se dobbiamo dare un nome a un vettore, possiamo usare una lettera qualunque, come $v$, $w$ oppure $z$, perché non siamo in dittatura quindi è nostro dovere perseguire la chiarezza, non la uniformità dei costumi.
È costume mettere una freccina sopra il nome di un vettore per indicare che stiamo parlando di un vettore.
Per esempio, $\vec v$, $\vec w$ e $\vec z$.
Questo costume risulta particolarmente utile se vogliamo parlare del modulo di un vettore senza usare le barrette: così, il modulo di $\vec v$ viene indicato con $v$ (senza freccina):
\begin{equation}
	\norm{\vec v} = v .
\end{equation}

Questa usanza è tipica in fisica, ma meno in matematica.
Io cercherò di adottarla il più possibile, anche se sono un matematico.

%%%%%%%%%%%%%%%%%%%%%%%%%%%%%%%%%%%%%%%%%%%%%%%%%%%%%%%%%%%%%%%
\subsection{Il vettore applicato}
Un \emph{vettore applicato} contiene una informazione in più: dove sta la coda, ossia il \emph{punto di applicazione}.
Così, nell'esercizio~\ref{par6957d6fa} abbiamo visto che lo stesso vettore può comparire in posti diversi del foglio.
Un vettore applicato sta invece in un punto preciso.

%%%%%%%%%%%%%%%%%%%%%%%%%%%%%%%%%%%%%%%%%%%%%%%%%%%%%%%%%%%%%%%
\subsection{Esercizio}
Abbiamo tre vettori, che chiamiamo $\vec v$, $\vec w$ e $\vec z$, e due punti che chiamiamo $P$ e $Q$.
Disegna i vettori applicati $(\vec v,P)$, $(\vec w,Q)$, e $(\vec v,Q)$.
In altre parole, applica il vettore $\vec v$ al punto $P$, il vettore $\vec w$ al punto $Q$ e il vettore $\vec v$ al punto $Q$.

\begin{center}
\includegraphics[width=.75\textwidth]{pics/vettori-03}
\end{center}

%%%%%%%%%%%%%%%%%%%%%%%%%%%%%%%%%%%%%%%%%%%%%%%%%%%%%%%%%%%%%%%
\subsection{Somma di vettori}\label{par6957e2da}
I vettori si possono sommare con il così detto metodo del ``punto-coda''.
Quindi, se abbiamo due vettori $\vec v$ e $\vec w$, la somma $\vec v+\vec w$ è un vettore ottenuto così:
\begin{enumerate}
\item	si fissa un punto e si applica $\vec v$ a questo punto;
\item 	si applica $\vec w$ alla punta di $\vec v$;
\item 	il vettore $\vec v+\vec w$ è il vettore con la coda nella coda di $\vec v$ e la punta nella punta di $\vec w$.
\end{enumerate}

\begin{center}
\includegraphics[width=.75\textwidth]{pics/vettori-04}
\end{center}

Si noti che $\vec v+\vec w$ e $\vec w+\vec v$ sono uguali!

%%%%%%%%%%%%%%%%%%%%%%%%%%%%%%%%%%%%%%%%%%%%%%%%%%%%%%%%%%%%%%%
\subsection{Esercizio}
Disegnare dei vettori e farne la somma.
Disegnare quattro vettori $\vec v_1$, $\vec v_2$, $\vec v_3$ e $\vec v_4$ e fare la somma $\vec v_1+\vec v_2+\vec v_3+\vec v_4$.

%%%%%%%%%%%%%%%%%%%%%%%%%%%%%%%%%%%%%%%%%%%%%%%%%%%%%%%%%%%%%%%
\subsection{Moltiplicazione di un vettore per uno scalare}\label{par6957e5e4}
La parola ``\emph{scalare}'' è un sinonimo di ``numero'' nel gergo dei vettori.
Possiamo moltiplicare un vettore $\vec v$ per un numero reale qualunque $t\in\R$.

Per esempio, $2\vec v$ è $\vec v+\vec v$, e $3\vec v=\vec v+\vec v+\vec v$, mentre $\frac12\vec v$ è metà di $\vec v$.

Quindi, dati un vettore $\vec v$ e uno scalare $t\in\R$, costruiamo il vettore $t\vec v$ così:
\begin{enumerate}
\item	$t\vec v$ ha la stessa direzione di $\vec v$;
\item	se $t=0$, $t\vec v$ è un punto; se $t>0$, $t\vec v$ ha lo stesso verso di $\vec v$; se $t<0$, $t\vec v$ ha verso opposto di $\vec v$;
\item	$t\vec v$ ha modulo $|t|\cdot\norm{\vec v}$.
\end{enumerate}


In particolare, $-\vec v$ è il vettore ``opposto'' a $\vec v$.
Giustamente, $\vec v+(-\vec v) = 0$.

\begin{center}
\includegraphics[width=.75\textwidth]{pics/vettori-05}
\end{center}

%%%%%%%%%%%%%%%%%%%%%%%%%%%%%%%%%%%%%%%%%%%%%%%%%%%%%%%%%%%%%%%
\subsection{Esercizio}
Disegnare tre vettori, scegliere tre numeri, e fare almeno tre moltiplicazioni scalare per vettore.

%%%%%%%%%%%%%%%%%%%%%%%%%%%%%%%%%%%%%%%%%%%%%%%%%%%%%%%%%%%%%%%
\subsection{Vettori nel piano cartesiano}\label{par6957e126}
Possiamo descrivere ciascun vettore nel piano cartesiano usando due numeri, chiamati \emph{coordinate del vettore}:
\begin{enumerate}
\item	Applichiamo il vettore all'origine, ossia mettiamo la coda del vettore nel punto $(0,0)$;
\item	I due numeri che descrivono il vettore sono le coordinate (ascissa e ordinata) della punta del vettore.
\end{enumerate}


\begin{center}
\includegraphics[width=.75\textwidth]{pics/vettori-06}
\end{center}

Se il vettore è applicato altrove, questi due numeri comunque determinano, data la coda del vettore, dove cadrà la punta del vettore.

\begin{center}
\includegraphics[width=.75\textwidth]{pics/vettori-07}
\end{center}


%%%%%%%%%%%%%%%%%%%%%%%%%%%%%%%%%%%%%%%%%%%%%%%%%%%%%%%%%%%%%%%
\subsection{Esercizio}\label{par6957e1d8}
Disegna i vettori con le seguenti coordinate:
$\vec v=(1,1)$, $\vec w=(0,3)$, $\vec u=(2,-1)$, $\vec k=(-1,-1)$.

Poi, disegna tu tre vettori e determinane le coordinate.

%%%%%%%%%%%%%%%%%%%%%%%%%%%%%%%%%%%%%%%%%%%%%%%%%%%%%%%%%%%%%%%
\subsection{Coordinate di un vettore applicato}
Questo concetto lo abbiamo già visto in~\ref{par6957e126}, ma vale la pena sottolinearlo.

Un vettore nel piano cartesiano è determinato da due numeri, le sue due coordinate.
Un vettore applicato nel piano cartesiano ha bisogno del doppio di coordinate: due coordinate del vettore e due coordinate per il suo punto di applicazione.

Domanda: dato un vettore $\vec v=(a,b)$ applicato nel punto $(x,y)$, la sua coda è nel punto $(x,y)$ per definizione, ma dove sta la sua punta?

Domanda: possiamo immaginare vettori anche nello spazio 3D.
Quante coordinate ci servono per determinare un vettore nello spazio tridimensionale?
Quante coordinate per un vettore applicato nello spazio tridimensionale?

%%%%%%%%%%%%%%%%%%%%%%%%%%%%%%%%%%%%%%%%%%%%%%%%%%%%%%%%%%%%%%%
\subsection{Esercizio}
Disegna i vettori $\vec v$, $\vec w$, $\vec u$ e $\vec k$ dall'Esercizio in~\ref{par6957e1d8} applicati al punto $P=(2,3)$.

%%%%%%%%%%%%%%%%%%%%%%%%%%%%%%%%%%%%%%%%%%%%%%%%%%%%%%%%%%%%%%%
\subsection{Somma di vettori in coordinate}
Abbiamo visto come si sommano i vettori usando il metodo ``punta-coda'' in~\ref{par6957e2da}.
Con le coordinate, la somma di vettori diventa molto facile: basta sommare i numeri!
Per esempio, se abbiamo due vettori $\vec v = (v_x,v_y)$ e $\vec w=(w_x,w_y)$, possiamo trovare la somma $\vec v+\vec w$ facendo
\begin{equation}
	\vec v + \vec w 
	= (v_x,v_y) + (w_x,w_y)
	= (v_x + w_x, v_y + w_y) .
\end{equation}
Per esempio,
\begin{equation}
	(2,4) + (1,2) = (3,6) .
\end{equation}

A questo punto la lettrice diligente dovrebbe riflettere sul perché sommare le coordinate sia equivalente a eseguire il metodo ``punto-coda''.

%%%%%%%%%%%%%%%%%%%%%%%%%%%%%%%%%%%%%%%%%%%%%%%%%%%%%%%%%%%%%%%
\subsection{Esercizio}
Si prendano i vettori dall'esercizio in~\ref{par6957e1d8} e si facciano almeno tre somme, sia con il disegno che con le coordinate.

%%%%%%%%%%%%%%%%%%%%%%%%%%%%%%%%%%%%%%%%%%%%%%%%%%%%%%%%%%%%%%%
\subsection{Moltiplicazione scalare per vettore in coordinate}
Abbiamo visto in~\ref{par6957e5e4} come moltiplicare un vettore per uno scalare.
Se si hanno le coordinate del vettore, questa moltiplicazione è semplicemente la moltiplicazione delle coordinate!

Per esempio, se abbiamo un vettore $\vec v = (v_x,v_y)$ e uno scalare $t\in\R$, allora
\begin{equation}
	t\vec v
	= t(v_x,v_y)
	= (t v_x, t v_y) .
\end{equation}
Per esempio, 
\begin{equation}
	2 (2,-3) = (4,-6) . 
\end{equation}

Si guardi con attenzione la seguente riga ti conti:
\begin{equation}
	\vec v + \vec v = (v_x,v_y) + (v_x,v_y) = (v_x + v_x, v_y + v_y) = (2v_x,2v_y) = 2\vec v .
\end{equation}

%%%%%%%%%%%%%%%%%%%%%%%%%%%%%%%%%%%%%%%%%%%%%%%%%%%%%%%%%%%%%%%
\subsection{Esercizio}
Si prendano tre numeri reali e tre vettori in coordinate e si facciano almeno tre moltiplicazioni scalare per vettore.

%%%%%%%%%%%%%%%%%%%%%%%%%%%%%%%%%%%%%%%%%%%%%%%%%%%%%%%%%%%%%%%
\subsection{Il modulo in termini delle coordinate}
Il modulo di un vettore si può ricavare dalle coordinate usando il teorema di Pitagora:\footnote{Il Teorema di Pitagora dice che: ``In ogni triangolo rettangolo, la somma dei quadrati costruiti sui cateti è uguale al quadrato costruito sull'ipotenusa.'' Qui per ``quadrato'' si intende ``l'area del quadrato''.}
se $\vec v=(a,b)$, allora $v=\norm{\vec v} = \sqrt{a^2 + b^2}$.

\begin{center}
\includegraphics[width=.75\textwidth]{pics/vettori-08}
\end{center}

%%%%%%%%%%%%%%%%%%%%%%%%%%%%%%%%%%%%%%%%%%%%%%%%%%%%%%%%%%%%%%%
\subsection{Esercizio}
Calcola il modulo dei vettori $\vec v$, $\vec w$, $\vec u$ e $\vec k$ dall'Esercizio in~\ref{par6957e1d8}.

%%%%%%%%%%%%%%%%%%%%%%%%%%%%%%%%%%%%%%%%%%%%%%%%%%%%%%%%%%%%%%%
\subsection{Esercizio}
Io ti do due vettori:
\begin{equation}
\begin{aligned}
	\vec e_1 &= (1,0) ,\\
	\vec e_2 &= (0,1) .
\end{aligned}
\end{equation}
Trova due scalari $a,b\in\R$ tali per cui
\begin{equation}
	a \vec e_1 + b \vec e_2 = (4,-5) .
\end{equation}

Possiamo interpretare questo problema così: 
immagina che i due vettori $\vec e_1$ e $\vec e_2$ siano le uniche due direzioni in cui puoi muoverti.
Se ti muovi in direzione $\vec e_1$ per una distanza $a$, significa che dal punto in cui sei vai fino alla punta di $a\vec e_1$.
Quindi, partendo dall'origine $(0,0)$ devi arrivare al punto $(4,-5)$ seguendo un po' $\vec e_1$ e un po' $\vec e_2$.
Il problema ti chiede di trovare quanto devi seguire l'uno e l'altro vettore.

%%%%%%%%%%%%%%%%%%%%%%%%%%%%%%%%%%%%%%%%%%%%%%%%%%%%%%%%%%%%%%%
\subsection{Esercizio}
Io ti do due vettori:
\begin{equation}
\begin{aligned}
	\vec v &= (2,1) ,\\
	\vec w &= (-1,2) .
\end{aligned}
\end{equation}
Trova due scalari $a,b\in\R$ tali per cui
\begin{equation}
	a \vec e_1 + b \vec e_2 = (5,-2) .
\end{equation}





\newpage
% !TEX encoding = UTF-8 Unicode
%!TEX root = FisMat.tex

\section{Statica}

%%%%%%%%%%%%%%%%%%%%%%%%%%%%%%%%%%%%%%%%%%%%%%%%%%%%%%%%%%%%%%%
\subsection{Introduzione}
Ci occuperemo di studiare sistemi fermi.
Per ``sistemi'' intendiamo gruppi di oggetti, come per esempio un candelabro appeso tramite una carrucola e controbilanciato da una molla.
Se un sistema è fermo, significa che è in equilibrio -- per definizione di \emph{equilibrio}.
Vogliamo trovare le condizioni che rendono equilibrato un sistema.

%\emph{equilibrio del punto materiale e del corpo rigido}.

Il nostro metodo di osservazione sarà il seguente:
\begin{enumerate}
\item	Facciamo un \emph{diagramma delle forze}, ossia un disegno del sistema con dei vettori applicati che rappresentano le forze esterne.
	A questo punto dovremo usare immaginazione e accettare un certo grado di incertezza.
\item	Per ciascun oggetto del sistema, studiamo le forze che agiscono su di esso e applichiamo le leggi di equilibrio del punto materiale e del corpo rigido.
\item	A questo punto abbiamo tutti gli elementi per dedurre ogni proprietà di equilibrio del sistema.
\end{enumerate}
Messo così, questo piano di lavoro è astratto, ma lo terremo come vademecum.

%%%%%%%%%%%%%%%%%%%%%%%%%%%%%%%%%%%%%%%%%%%%%%%%%%%%%%%%%%%%%%%
\subsection{La forza}
La \emph{forza} è una grandezza fisica vettoriale che non si vede, ma ha effetti visibili (questa non è una definizione).
La sua unità di misura è il \emph{Newton} che si denota con $\unit{N}$.
In relazione alle unità di misura fondamentali, dobbiamo ricordarci che:
\begin{equation}
	\unit{N} = \unit[per-mode = symbol]{\kilogram\,\metre\per\second^2}
\end{equation}

Esempi di forza: 
\begin{itemize}[leftmargin=*]
\item	Il \emph{peso} è una forza. 
	Noi sperimentiamo il peso costantemente sul pianeta Terra: esso è la conseguenza della nostra relazione con la Terra, che ci attrae ad essa e noi attraiamo la Terra a noi.
	Quando ci pesiamo sulla bilancia, esprimiamo questa misura in chilogrammi: tecnicamente, stiamo sbagliando. 
	Il chilogrammo $\unit{kg}$ è l'unità di misura della massa.
	Quindi se dico ``Io peso 70 chilogrammi'', dovrei invece dire ``Io ho una massa di 70 chilogrammi'', oppure ``Io peso $686{,}7$ Newton''.
	Nella vita quotidiana non è un problema, perché peso e massa sono direttamente proporzionali e questa proporzionalità non cambia mai:
	\begin{equation}
		\text{Peso}_{\text{sulla Terra}} = g \cdot \text{Massa},
	\end{equation}
	dove $g$ è l'accelerazione di gravità, 
	\begin{equation}
		g = 9{,}81\,\unit[per-mode = symbol]{\metre\per\second^2} .
	\end{equation}
\item	La Luna gira attorno alla Terra, come se Terra e Luna si tenessero per mano.
	Cosa le tiene assieme? 
	La stessa forza di gravità che tiene noi per terra.
	Che i fatti celesti siano soggetti alle stesse leggi dei fatti terrestri è una importante intuizione della fisica.
	La \emph{Legge di gravitazione universale}\footnote{La legge di gravitazione universale fu formulata da Isaac Newton nel 1687} afferma che due corpi di massa $M$ e $m$, rispettivamente, si attraggono con una forza
	\begin{equation}
		\vec F = G \frac{ M \cdot m}{r^2} ,
	\end{equation}
	dove $r$ è la distanza tra i (centri di massa dei) due corpi e $G$ è la \emph{costante gravitazionale universale}
	\begin{equation}
		G = 6{,}67\times 10^{-11}\ \unit{ \frac{ Nm^2 }{ kg^2 }} .
	\end{equation}
\item	Se schiacciamo o tiriamo una molla, la molla ci restituisce una forza, detta \emph{forza elastica}.
\item	Se una palla ci colpisce, la botta che percepiamo è una forza.
	Siccome però è una forza intensa che dura poco, quella nostra esperienza è meglio descritta dall'\emph{impulso}, che vedremo tra poco.
\item	Quando avviciniamo due magneti, l'attrazione o la repulsione tra loro è una forza, chiamata \emph{forza magnetica}.
\item	Se strofiniamo un maglione di lana e lo avviciniamo a dei pezzetti di carta, questi verranno attratti dal maglione e ci rimarranno attaccati.
	Anche questa attrazione è una forza: la \emph{forza elettrostatica}.
\item	Se proviamo a trascinare un oggetto pesante (per esempio una lavatrice), faremo un certo grado di fatica.
	Se però mettiamo l'oggetto su un carrellino munito di ruote, riusciremo a trascinare lo stesso oggetto con grande facilità.
	Senza carrellino, la \emph{forza di attrito} si opponeva al moto e noi dovevamo vincerla. 
	Con il carrellino, la forza di attrito è diminuita enormemente.
\end{itemize}


%%%%%%%%%%%%%%%%%%%%%%%%%%%%%%%%%%%%%%%%%%%%%%%%%%%%%%%%%%%%%%%
\subsection{Equilibrio del punto materiale}
Un \emph{punto materiale} è un punto adimensionale, ossia senza dimensioni spaziali, che ha una sua posizione e una sua massa.

Un punto materiale è in \emph{equilibrio}, se e solo se la somma di tutte le forze agenti su di esso è zero.

Un punto materiale è un'astrazione: non esiste un oggetto che non occupi un volume.
Però, ci sono situazioni in cui non importa il volume di un oggetto.
Per esempio, se studiamo il moto dei pianeti attorno al sole, non è importante che dimensioni abbiano e possiamo riassumere l'intero pianeta Terra, con tutti i casini che accadono su di essa, a un punto infinitesimo di massa $M_T = 5{,}972 \times 10^{24} \ \unit{kg}$.

%immaginiamo che un oggetto sia riassunto in un punto adimensionale (ossia che non occupa volume) che ha una sua posizione e una sua massa.



%%%%%%%%%%%%%%%%%%%%%%%%%%%%%%%%%%%%%%%%%%%%%%%%%%%%%%%%%%%%%%%
\subsection{Equilibrio del corpo rigido}
Un \emph{corpo rigido} è un oggetto che ha una massa e una forma rigida, ossia che non cambia.

Un corpo rigido è in \emph{equilibrio} se e solo se
\begin{enumerate}
\item	la somma di tutte le forze agenti su di esso è zero, e 
\item	la somma di tutti i momenti delle forze agenti su di esso rispetto ad un punto è zero.
\end{enumerate}

Anche il corpo rigido è un'astrazione, perché nella realtà gli oggetti cambiano forma quando sottoposti a delle forze.
In molti casi però tale deformazione è talmente piccola da potersi trascurare.

%%%%%%%%%%%%%%%%%%%%%%%%%%%%%%%%%%%%%%%%%%%%%%%%%%%%%%%%%%%%%%%
\subsection{Esempio di un sistema in equilibrio}
Consideriamo un lampadario di massa $m=3\unit{kg}$ appeso al soffitto.
Vogliamo sapere quale forza agisce sul soffitto.

Per prima cosa, facciamo un diagramma delle forze:
\begin{center}
\includegraphics[width=.75\textwidth]{pics/statica-01}
\end{center}
Sappiamo che il peso del lampadario è $P=gm$, dove $g$ è l'accelerazione di gravità,
e sappiamo che il peso del lampadario punta verso il basso.
Così abbiamo il peso come vettore $\vec P$.

In questo diagramma ci sono tre oggetti: il lampadario, il cavo che tiene il lampadario e il soffitto.
Guardiamo un oggetto alla volta.

Primo, il lampadario è fermo, quindi in equilibrio.
Questo significa che la somma delle forze che agiscono sul lampadario deve essere zero.
Quindi c'è una forza che bilancia la forza peso: ovviamente è la tensione $\vec T$ del cavo che tiene il lampadario:
\begin{center}
\includegraphics[width=.75\textwidth]{pics/statica-02}
\end{center}
Ricapitolando, siccome la somma delle forze che agiscono sul lampadario deve essere zero, abbiamo 
\begin{equation}
	\vec P + \vec T = 0 ,
\end{equation}
ossia $\vec T = - \vec P$.
Così, la tensione sul cavo è $T=P=gm$ ed è opposta alla forza peso, ossia verso l'alto.

Secondo, il cavo che tiene il lampadario è fermo, quindi la somma delle forze che agiscono su di esso deve essere zero.
\begin{center}
\includegraphics[width=.75\textwidth]{pics/statica-03}
\end{center}
Ho disegnato due forze sul cavo: una forza $\vec T_g$ ad un capo del cavo, e una forza $\vec T_s$ all'altro capo del cavo.
Siccome la somma è zero, dobbiamo avere 
\begin{equation}
	\vec T_g+\vec T_s = 0 ,
\end{equation}
ossia $\vec T_s = -\vec T_g$.

Naturalmente, $\vec T_g = -\vec T$.
Per questo motivo, abbiamo
\begin{equation}
	\vec T_s = -\vec T_g = -(-\vec T) = \vec T = -\vec P .
\end{equation}

Infine, chiamo $\vec F$ la forza che agisce sul soffitto:
\begin{center}
\includegraphics[width=.75\textwidth]{pics/statica-04}
\end{center}
Naturalmente, $\vec F = - \vec T_s$, e quindi $\vec F = \vec P$.

Concludiamo che la forza che agisce sul soffitto ha modulo $gm$ e punta verso il basso.

NB! 
Il soffitto è in equilibrio, però in questo problema non siamo interessati alle sue condizioni di equilibrio.

Questo esercizio può sembrare pedante, ma vuole mostrare come si propagano le forze grazie ai principi di equilibrio del corpo rigido.

%%%%%%%%%%%%%%%%%%%%%%%%%%%%%%%%%%%%%%%%%%%%%%%%%%%%%%%%%%%%%%%
\subsection{La Legge di Hooke}
La {Legge di Hooke} descrive la forza impressa da una molla quando compressa o allungata.
Quando una molla è a riposo ha una sua lunghezza.
Se allungata o compressa, denotiamo con $\Delta x$ la differenza rispetto alla lunghezza a riposo.
Per distinguere se questo scarto è di allungamento o compressione, definiamo il vettore $\Delta\vec x$ come il vettore che ha coda nel punto di riposo della molla e punta nella posizione attuale della molla:

\begin{center}
\includegraphics[width=.75\textwidth]{pics/statica-05}
\end{center}

La \emph{legge di Hooke} afferma che la \emph{forza elastica} $\vec F$ impressa dalla molla è
\begin{equation}
	\vec F = - k \Delta\vec x ,
\end{equation}
dove $k$ è la \emph{costante elastica} della molla.

La costante elastica dipende dalla molla e va misurata sperimentalmente.
L'unità di misura della costante elastica è $\unit{\frac{N}{m}}$.
Così, una molla con costante elastica $1\unit{N\per m}$ da una forza di un Newton per ogni metro di compressione (o allungamento).

\begin{center}
\includegraphics[width=.75\textwidth]{pics/statica-06}
\end{center}

NB!
La legge di Hooke vale solo se lo spostamento $\Delta x$ è piccolo abbastanza: 
se si allunga o contrae la molla troppo, questa subisce altri fenomeni di deformazione e quindi la legge di Hooke non vale più.
Per esempio, se si tira troppo la molla si rompe e quindi non esercita più alcuna forza.

NB!
Una molla esercita una forza su entrambi gli estremi, con la medesima intensità (ossia modulo), medesima direzione, ma verso opposto.

\begin{center}
\includegraphics[width=.75\textwidth]{pics/statica-06-1}
\end{center}

%%%%%%%%%%%%%%%%%%%%%%%%%%%%%%%%%%%%%%%%%%%%%%%%%%%%%%%%%%%%%%%
\subsection{Esercizio}
Una molla di costante elastica $k=34\unit{N/m}$ e di lunghezza a riposo $\ell=45\unit{cm}$ è dentro una scatola lunga $30\unit{cm}$.
Quali sono le forze impresse dalla molla sulle due pareti della scatola?

%%%%%%%%%%%%%%%%%%%%%%%%%%%%%%%%%%%%%%%%%%%%%%%%%%%%%%%%%%%%%%%
\subsection{Esercizio}
Nei seguenti disegni, sono rappresentate delle molle sia a riposo che sotto sforzo.
Nelle immagini delle molle sotto sforzo, disegna gli scarti $\Delta \vec x$ e le forze elastiche $\vec F$ date dalla legge di Hooke.

\begin{center}
\includegraphics[width=.75\textwidth]{pics/statica-07}
\end{center}

\begin{center}
\includegraphics[width=.75\textwidth]{pics/statica-08}
\end{center}

\begin{center}
\includegraphics[width=.75\textwidth]{pics/statica-09}
\end{center}

%%%%%%%%%%%%%%%%%%%%%%%%%%%%%%%%%%%%%%%%%%%%%%%%%%%%%%%%%%%%%%%
\subsection{Esercizio}
Appendo un corpo di massa $m=13\unit{g}$ a una molla con costante elastica $k=50\unit{N/m}$.
La molla a riposo è lunga $\ell = 7\unit{cm}$.
Quanto è lunga la molla quando il corpo è appeso?

%%%%%%%%%%%%%%%%%%%%%%%%%%%%%%%%%%%%%%%%%%%%%%%%%%%%%%%%%%%%%%%
\subsection{Esercizio}\label{par6958e97f}
Lucia e Marco hanno costruito una bilancia casalinga.
Hanno messo una molla in verticale (un tubo la tiene dritta).
In basso la molla appoggia al tavolo, in alto invece c'è una piano su cui appoggiare gli oggetti da pesare.

Lucia e Marco devono innanzitutto calcolare la costante elastica della molla.
Come fanno? 
Prova a rispondere prima di continuare a leggere.

\begin{center}
\includegraphics[width=.45\textwidth]{pics/statica-10}
\includegraphics[width=.45\textwidth]{pics/statica-11}
\end{center}

Sì, fanno così: ci appoggiano sopra un corpo di massa $m=100\unit{g}$ e misurano quanto la molla si contrae.
La loro misura da $\Delta x = 1\unit{cm}$.
Quanto vale la costante elastica $k$ della molla?

A questo punto possono provare a pesare un libro di massa $M$ (sconosciuta).
Lo appoggiano sulla bilancia e misurano che la molla si contrae di $3{,}4\unit{cm}$.
Quanto pesa il libro? Quanto vale $M$?


%%%%%%%%%%%%%%%%%%%%%%%%%%%%%%%%%%%%%%%%%%%%%%%%%%%%%%%%%%%%%%%
%\subsection{Esercizio}
%Due molle sono messe dentro a una scatola.
%Hanno costanti elastiche $k_1$ e $k_2$, rispettivamente, e lunghezze a riposo $\ell_1$ e $\ell_2$.
%La scatola è lunga $L$ e larga $H$.
%
%Possiamo mettere le due molle in due modi: in serie per il lungo o in parallelo per il largo.
%
%Se le mettiamo in serie per il lungo, quanto quali forze esercitano le molle sulle pareti della scatola?
%E se le mettiamo in parallelo per il largo?

% e supponiamo che $L<\ell_1 + \ell_2$.

%%%%%%%%%%%%%%%%%%%%%%%%%%%%%%%%%%%%%%%%%%%%%%%%%%%%%%%%%%%%%%%
\subsection{Esercizio}
Due molle identiche sono messe dentro a una scatola.
La loro costante elastica è $k$ e la loro lunghezza a riposo è $\ell$.
La scatola è lunga $L$ e larga $H$.

Possiamo mettere le due molle in due modi: in serie per il lungo o in parallelo per il largo.
Se le mettiamo in serie per il lungo, quanto quali forze esercitano le molle sulle pareti della scatola?
E se le mettiamo in parallelo per il largo?

\begin{center}
\includegraphics[width=.8\textwidth]{pics/statica-12}
\end{center}

Se preferisci avere dei numeri, considera $k=60\unit{N/m}$, $\ell = 10\unit{cm}$,
$L=15\unit{cm}$, $H=8\unit{cm}$.

%%%%%%%%%%%%%%%%%%%%%%%%%%%%%%%%%%%%%%%%%%%%%%%%%%%%%%%%%%%%%%%
\subsection{Esercizio}
Torniamo alla bilancia descritta in~\ref{par6958e97f}.
Tra Lucia e Marco nasce una diatriba.
Lucia sostiene che dentro al tubo ci siano due molle, una sopra l'altra, identiche tra loro.
Marco invece sostiene che ci sia una sola molla.
Se avesse ragione Lucia, quale sarebbe la costante elastica delle due molle?

C'è un modo, senza aprire il tubo che contiene le molle, di verificare chi dei due ha ragione?\footnote{La risposta è no. Questo è un esempio in cui possiamo fare modelli diversi per la stessa situazione pratica.
In altre parole, possiamo immaginare che dentro al tubo ci siano una o due molle (con lunghezze e costanti elastiche appropriate), e comunque dare le stesse previsioni corrette.
Se non possiamo aprire il tubo, non sapremo mai quante molle ci sono dentro.
Anzi, magari non c'è nemmeno una molla, ma un sofisticato meccanismo elettromagnetico.
}


\begin{center}
\includegraphics[width=.75\textwidth]{pics/statica-13}
\end{center}

%%%%%%%%%%%%%%%%%%%%%%%%%%%%%%%%%%%%%%%%%%%%%%%%%%%%%%%%%%%%%%%
\subsection{Esercizio ($\star$)}
Abbiamo due molle di costante elastiche $k_1$ e $k_2$, rispettivamente, e lunghezze a riposo $\ell_1$ e $\ell_2$.
Vengono messe in serie, una dopo l'altra, in una scatola di lunghezza totale $L$.
Supponiamo $L<\ell_1+\ell_2$ (quindi le molle sono compresse).
Nella scatola, misuriamo che la prima molla è ora lunga $x_1$ e la seconda molla è lunga $x_2$.
Ovviamente, $x_1+x_2 = L$.

\begin{center}
\includegraphics[width=.75\textwidth]{pics/statica-14}
\end{center}

\begin{enumerate}
\item	(Poco facile) 
	Quanto valgono $x_1$ e $x_2$ in termini dei dati sulle molle e $L$?
	(Se vuoi numeri: $k_1 = 10\unit{N/m}$, $k_2 = 25\unit{N/m}$, $\ell_1=34\unit{cm}$, $\ell_2=23\unit{cm}$, $L=49\unit{cm}$)
\item	(Meno difficile)
	Per semplificare, supponi $k_1=k_2=k$ e $\ell_1=\ell_2=\ell$. 
	Quanto valgono $x_1$ e $x_2$?
	(Se vuoi numeri: $k = 15\unit{N/m}$, $\ell=20\unit{cm}$, $L=32\unit{cm}$)
\item	(Meno difficile, ma poco facile)
	Supponi $k_2 = 2k_1$ e $\ell_2=\ell_1$.
	Quanto valgono $x_1$ e $x_2$?
	(Se vuoi numeri: $k_1 = 16\unit{N/m}$, $\ell_1=19\unit{cm}$, $L=53\unit{cm}$)
\end{enumerate}

%%%%%%%%%%%%%%%%%%%%%%%%%%%%%%%%%%%%%%%%%%%%%%%%%%%%%%%%%%%%%%%
\subsection{Esercizio ($\star\star$)}
Un lampadario è appeso al soffitto con due molle messe a triangolo.
Le molle sono identiche, con costante elastica $k$ e lunghezza a riposo $\ell$.
Il triangolo formato dalle molle ha base $L$ (lungo il soffitto) e altezza $h$.
Al vertice opposto al soffitto è appeso il lampadario, di peso $P$.

\begin{center}
\includegraphics[width=.75\textwidth]{pics/statica-15}
\end{center}

\begin{enumerate}
\item
	Queste cinque quantità $k$, $\ell$, $L$, $h$ e $P$,
	sono in relazione tra loro.
	Scrivere una formula che le mette in relazione.
	(Con questa formula, potrai rispondere velocemente a tutte le domande seguenti).
\item	
	Dati $k=125\unit{N/m}$, $\ell=1\unit{m}$, $L=1\unit{m}$, $h=1\unit{m}$,
	quanto vale $P$?
\item
	Dati $k=125\unit{N/m}$, $\ell=1\unit{m}$, $L=1\unit{m}$, $P=20\unit{N}$, quanto vale $h$?
\end{enumerate}

%%%%%%%%%%%%%%%%%%%%%%%%%%%%%%%%%%%%%%%%%%%%%%%%%%%%%%%%%%%%%%%
\subsection{Esercizio}
Due molle di costante elastica $k_1$ e $k_2$ rispettivamente, e di lunghezze a riposo $\ell_1$ e $\ell_2$, rispettivamente, sono appese dopo l'altra al soffitto, in serie.
Alla seconda molla, più in basso, viene agganciata una massa $m$.
Quanto saranno lunghe le due molle?

\begin{center}
\includegraphics[width=.75\textwidth]{pics/statica-16}
\end{center}

%%%%%%%%%%%%%%%%%%%%%%%%%%%%%%%%%%%%%%%%%%%%%%%%%%%%%%%%%%%%%%%%
%\subsection{Esercizio}
%Due molle di costante elastica $k_1$ e $k_2$ rispettivamente, e di lunghezze a riposo $\ell_1$ e $\ell_2$, rispettivamente, sono appese a fianco l'altra al soffitto, in parallelo.
%Viene agganciata una massa $m$ alle due
%Quanto saranno lunghe le due molle?
%
%\begin{center}
%\includegraphics[width=.75\textwidth]{pics/statica-16}
%\end{center}

%%%%%%%%%%%%%%%%%%%%%%%%%%%%%%%%%%%%%%%%%%%%%%%%%%%%%%%%%%%%%%%
\subsection{Esercizio}
Inventa tu un esercizio e risolvilo.

%%%%%%%%%%%%%%%%%%%%%%%%%%%%%%%%%%%%%%%%%%%%%%%%%%%%%%%%%%%%%%%
\subsection{Una questione di triangoli simili}\label{par695a342c}
Nel seguito, studieremo il piano inclinato e useremo una semplice proprietà dei triangoli simili.
Questa semplice proprietà ci permetterà di fare a meno di usare la trigonometria, presumo non si conosca ancora.

Innanzitutto, due triangoli sono simili se hanno gli stessi angoli.
Chiamiamo $\widetriangle{ABC}$ e $\widetriangle{A'B'C'}$ i due triangoli, dove a lettera uguale corrisponde angolo uguale.
È una proprietà dei triangoli simili che i lati corrispondenti siano proporzionali tutti con la stessa proporzione.
In altre parole:
\begin{equation}\label{eq695a3689}
	\frac{\overline{AB}}{\overline{A'B'}} 
	= \frac{\overline{AC}}{\overline{A'C'}}
	= \frac{\overline{BC}}{\overline{B'C'}} .
\end{equation}

\begin{center}
\includegraphics[width=.75\textwidth]{pics/statica-16-01}
\end{center}

Quindi, se ci viene dato $\widetriangle{ABC}$ e un lato di $\widetriangle{A'B'C'}$, possiamo trovare gli altri due lati di $\widetriangle{A'B'C'}$.
Per esempio, se $\widetriangle{ABC}$ è il triangolo di lati $\overline{AB}=3$, $\overline{AC}=4$ e $\overline{BC}=5$,
e se so che $\overline{A'B'} = 1$, allora 
\begin{equation}
	\overline{A'C'} = \frac{\overline{AC}}{\overline{AB}} \overline{A'B'} 
	= \frac{4}{3} ,
\end{equation}
e
\begin{equation}
	\overline{B'C'} = \frac{\overline{BC}}{\overline{AB}} \overline{A'B'}
	= \frac{5}{3} .
\end{equation}



%%%%%%%%%%%%%%%%%%%%%%%%%%%%%%%%%%%%%%%%%%%%%%%%%%%%%%%%%%%%%%%
\subsection{Il piano inclinato}
Se un oggetto, ad esempio un libro, sta su un tavolo orizzontale il peso è controbilanciato dalla forza vincolare del tavolo.
Quindi, il libro esercita una forza sul tavolo e il tavolo risponde con una forza uguale in modulo e direzione e contraria in verso. 
Questa forza del tavolo sul libro è chiamata \emph{forza vincolare}.
Il nome viene dal fatto che il tavolo è un \emph{vincolo}, ossia impedisce il movimento al libro.

\begin{center}
\includegraphics[width=.75\textwidth]{pics/statica-17}
\end{center}

Supponiamo ora che il tavolo non sia più orizzontale, ma bensì inclinato:
il libro è su un \emph{piano inclinato}.
Il peso del libro è comunque una forza diretta verso terra, verticalmente.
La forza vincolare del tavolo invece non può che essere perpendicolare al piano.
Se facciamo la somma di questi due vettori, non otteniamo zero.

\begin{center}
\includegraphics[width=.75\textwidth]{pics/statica-18}
\end{center}

Ora ti spiego un buon modo di analizzare questa situazione.
Innanzitutto, prendiamo la retta parallela al piano del tavolo e la retta perpendicolare al piano del tavolo.
Con queste due rette, scomponiamo le forze come abbiamo imparato in~\ref{par6959428e}.
Siccome la forza vincolare $\vec F$ è perpendicolare al piano, non c'è niente da scomporre.
La forza peso $\vec P$ invece si scompone nelle due componenti $\vec P_{\parallelo}$, parallela al piano del tavolo, e $\vec P_{\perp}$, perpendicolare al piano del tavolo.

\begin{center}
\includegraphics[width=.65\textwidth]{pics/statica-19}
\includegraphics[width=.34\textwidth]{pics/statica-20}
\end{center}

Il peso del libro, infatti, ha due conseguenze.
La prima è che il libro preme sul tavolo: questa forza è~$\vec P_{\perp}$.
La seconda è che il libro tende a scivolare giù lungo il piano: la causa di questo movimento è la forza~$\vec P_{\parallelo}$.

Siccome il libro non affonda nel tavolo, abbiamo
\begin{equation}
	\vec P_{\perp} + \vec F = 0 ,
\end{equation}
ossia, la forza vincolare $\vec F$ bilancia $\vec P_{\perp}$ e quindi $\vec F =  -\vec P_{\perp}$.

Cosa bilancia la forza $\vec P_{\parallelo}$?
La risposta dipende dalla situazione.
Può essere l'attrito, che vedremo tra poco, oppure un filo con un contrappeso, oppure una molla, o tutte queste e tante altre cose insieme.
Se niente bilancia la forza $\vec P_{\parallelo}$, o se non è abbastanza bilanciata, allora il libro scivolerà giù dal tavolo.

%%%%%%%%%%%%%%%%%%%%%%%%%%%%%%%%%%%%%%%%%%%%%%%%%%%%%%%%%%%%%%%
\subsection{Esempio}\label{subs695a37fb}
Un carrello sta su un piano inclinato.
Descriviamo questo piano inclinato con un triangolo rettangolo che ha cateto di base pari a $4\unit{m}$, cateto di altezza pari a $3\unit{m}$ e quindi ipotenusa pari a $5\unit{m}$.
Il carrello ha una massa $M=15\unit{kg}$.
Al carrello è agganciato un filo che, parallelo al piano inclinato, arriva in cima al piano inclinato e quindi scende verticale con una massa $m$ appesa.
Quanto deve valere $m$ per tenere in equilibrio il carrello?\\

Risolviamo l'esercizio insieme.
Innanzitutto facciamo un disegno con le forze principali:

\begin{center}
\includegraphics[width=.75\textwidth]{pics/statica-21}
\end{center}

Le forze disegnate sono:
\begin{itemize}
\item	$\vec P_p$: il peso del pesetto. 
	Il modulo di questa forza è $P_p = gm$, dove $g$ è l'accelerazione di gravità.
	Questa forza è verticale e punta verso il basso.
\item	$\vec T_p$: la tensione del filo attaccato al pesetto.
	Siccome il pesetto è fermo e le uniche forze che agiscono su di esso sono $\vec P_p$ e $\vec T_p$, possiamo già dire che $\vec P_p + \vec T_p = 0$, ossia 
		\begin{equation}
			\vec T_p = - \vec P_p .
		\end{equation} 
	Quindi il modulo di $\vec T_p$ è $T_p = gm$.
\item	$\vec P_c$: il peso del carrello.
	Il modulo di questa forza è $P_c = gM$.
	Questa forza è verticale e punta verso il basso.
\item	$\vec F_v$: la forza vincolare del piano inclinato.
	Questa forza è perpendicolare al piano inclinato, e il suo modulo dipende dal peso del carrello.
\item	$\vec T_c$: la tensione del filo attaccato al carrello.
	Siccome la tensione lungo il filo non cambia modulo\footnote{la carrucola non ha attrito},
	possiamo già dire che $T_c = T_p = gm$.
\end{itemize}

Studiamo in dettaglio la situazione delle forze sul carrello:

\begin{center}
\includegraphics[width=.5\textwidth]{pics/statica-22}
\end{center}

Tre forze agiscono sul carrello: $\vec P_c$, $\vec F_v$ e $\vec T_c$.
Siccome il carrello è in equilibrio abbiamo
\begin{equation}\label{eq695a1776}
	\vec P_c + \vec F_v + \vec T_c = 0 .
\end{equation}
Scomponiamo $\vec P_c$ nelle due componenti parallela $\vec P_{\parallelo}$ e perpendicolare $\vec P_\perp$ al piano inclinato.
Usando questa scomposizione in~\eqref{eq695a1776}, otteniamo
\begin{equation}
\begin{aligned}
	0 &= \vec P_c + \vec F_v + \vec T_c \\
	&= \vec P_{\parallelo} + \vec P_\perp + \vec F_v + \vec T_c \\
	&= (\vec P_{\parallelo} + \vec T_c ) + (\vec P_\perp + \vec F_v) .
\end{aligned}
\end{equation}
Per quello\footnotemark{} che abbiamo imparato in \ref{par695a248c},
\footnotetext{
In~\eqref{eq695a2590} abbiamo imparato che 
``prima sommo, poi scompongo $=$ prima scompongo, poi sommo''.
}
 sappiamo che 
i vettori $\vec P_{\parallelo} + \vec T_c$ e $\vec P_\perp + \vec F_v$
sono le componenti di $\vec P_c + \vec F_v + \vec T_c$ lungo le due rette.
Siccome $\vec P_c + \vec F_v + \vec T_c=0$, allora queste componenti devono essere zero, e quindi otteniamo
\begin{equation}
	\begin{cases}
	\vec P_{\parallelo} + \vec T_c &= 0 ,\\
	\vec P_\perp + \vec F_v &= 0 .
	\end{cases}
\end{equation}

Di queste due equazioni, quella che più ci interessa è la prima.
Infatti, $\vec P_{\parallelo} + \vec T_c = 0$ implica che il modulo di $\vec T_c$ 
(che sappiamo essere $T_c = gm$), è uguale al modulo di $\vec P_{\parallelo}$:
\begin{equation}\label{eq695a377b}
	gm = P_{\parallelo} .
\end{equation}
Siccome $m$ è l'incognita da determinare e siccome $\vec P_{\parallelo}$ dovremmo saperlo calcolare, questa dovrebbe darci la risposta.\\

Dobbiamo calcolare $\vec P_{\parallelo}$ e lo facciamo usando un po' di sana geometria euclidea, quella che conosciamo da almeno duemiladuecento anni.
Ho provato a riassumere in~\ref{par695a342c} quello che ci serve.

Se guardiamo bene lo schema delle forze sul carrello, e in particolare la scomposizione della forza peso, noteremo la presenza di due triangoli simili:
\begin{center}
\includegraphics[width=.45\textwidth]{pics/statica-23}
\includegraphics[width=.45\textwidth]{pics/statica-24}
\end{center}
Il triangolo piccolo è quello dato dalla scomposizione della forza peso,
quindi i suoi lati sono $P$, $P_\parallelo$ e $P_\perp$.
Il triangolo più grande invece è quello che descrive il piano inclinato, e i suoi lati sono la base $b=4\unit{m}$, l'altezza $a=3\unit{m}$ e l'ipotenusa $i=5\unit{m}$.

{\bf Esercizio:} individua gli angoli uguali.

Dopo che hai provato a individuare gli angoli uguali\footnote{Innanzi tutto, nota che in entrambi i triangoli c'è un angolo retto.
Poi, nota che $i$ e $P_\parallelo$ sono paralleli, e che $a$ e $P$ sono anche paralleli. Quindi l'angolo tra $i$ e $a$ deve essere uguale all'angolo tra $P_\parallelo$ e $P$.
Infine, rimane un angolo in ciascun triangoli, che quindi questi due angoli devono essere uguali.
Il disegno è:
\begin{center}
\includegraphics[width=.2\textwidth]{pics/statica-25}
\end{center}
}
potrai usare~\eqref{eq695a3689} e dire che
\begin{equation}
	\frac{ P_\parallelo }{ a } = \frac{ P }{ i },
\end{equation}
e quindi 
\begin{equation}\label{eq695a378d}
	P_\parallelo = \frac{ P }{ i } a 
	= \frac{gM }{5\unit{m}} 3\unit{m}
	= \frac{3}{5} g M .
\end{equation}

Finalmente possiamo concludere usando~\eqref{eq695a377b} e~\eqref{eq695a378d}:
\begin{equation}
	gm = \frac{3}{5} g M ,
\end{equation}
cioè
\begin{equation}
	m = \frac{3}{5} M 
	= \frac{3}{5} 15 \unit{kg}
	= 9 \unit{kg} .
\end{equation}

%%%%%%%%%%%%%%%%%%%%%%%%%%%%%%%%%%%%%%%%%%%%%%%%%%%%%%%%%%%%%%%
\subsection{Stesso esempio, meno parole}
Nel precedente paragrafo~\ref{subs695a37fb}, ho scritto tanto perché desidero mostrare il ragionamento nella sua interezza.
Quello è quel che dovrebbe succedere nella tua testa quando affronti un problema di fisica.
Però, quando ti chiedo di risolvere un problema in una verifica, non hai il tempo necessario per scrivere tutte quelle cose: il pensiero è molto, molto più veloce della penna (o della tastiera).
Devi essere sintetico nella scrittura, e qui ti mostro come io descriverei lo stesso problema con meno parole.
Non essere però sintetico nel pensiero!\\

Innanzitutto facciamo un disegno con le forze principali:
\begin{center}
\includegraphics[width=.75\textwidth]{pics/statica-21}
\end{center}

Le forze disegnate sono:
\begin{itemize}
\item	$\vec P_p$: il peso del pesetto.
	$P_p = gm$, $g$ è l'accelerazione di gravità.
\item	$\vec T_p$: la tensione del filo attaccato al pesetto.
	Siccome il pesetto è fermo: $\vec P_p + \vec T_p = 0$, 
	quindi $T_p = gm$.
\item	$\vec P_c$: il peso del carrello.
	$P_c = gM$.
\item	$\vec F_v$: la forza vincolare del piano inclinato.
\item	$\vec T_c$: la tensione del filo attaccato al carrello.
	Siccome la tensione lungo il filo non cambia modulo: $P_c = gM$.
\end{itemize}

Le forze sul carrello:

\begin{center}
\includegraphics[width=.5\textwidth]{pics/statica-22}
\end{center}

Siccome il carrello è in equilibrio:
\begin{equation}\label{eq695a1776}
	\vec P_c + \vec F_v + \vec T_c = 0 .
\end{equation}
Scomponiamo $\vec P_c$ nelle due componenti parallela $\vec P_{\parallelo}$ e perpendicolare $\vec P_\perp$ al piano inclinato.
Quindi:
\begin{equation}
	\begin{cases}
	\vec P_{\parallelo} + \vec T_c &= 0 ,\\
	\vec P_\perp + \vec F_v &= 0 .
	\end{cases}
\end{equation}
Quindi:
\begin{equation}\label{eq695a377b_bis}
	P_{\parallelo} = T_c .
\end{equation}


Per calcolare $\vec P_{\parallelo}$ usiamo la presenza di due triangoli simili:
\begin{center}
\includegraphics[width=.5\textwidth]{pics/statica-25}
\end{center}
%Il triangolo piccolo è quello dato dalla scomposizione della forza peso,
%quindi i suoi lati sono $P$, $P_\parallelo$ e $P_\perp$.
dove $b=4\unit{m}$,  $a=3\unit{m}$ e  $i=5\unit{m}$.

Per la proporzionalità tra lati corrispondenti in triangoli simili, otteniamo
\begin{equation}
	\frac{ P_\parallelo}{ a } = \frac{ P }{ i },
\end{equation}
e quindi 
\begin{equation}\label{eq695a378d_bis}
	P_\parallelo = \frac{ P }{ i } a 
	= \frac{gM }{5\unit{m}} 3\unit{m}
	= \frac{3}{5} g M .
\end{equation}

Finalmente possiamo concludere usando~\eqref{eq695a377b_bis} e~\eqref{eq695a378d_bis}:
\begin{equation}
	gm = \frac{3}{5} g M ,
\end{equation}
cioè
\begin{equation}
	m = \frac{3}{5} M 
	= \frac{3}{5} 15 \unit{kg}
	= 9 \unit{kg} .
\end{equation}

%%%%%%%%%%%%%%%%%%%%%%%%%%%%%%%%%%%%%%%%%%%%%%%%%%%%%%%%%%%%%%%
\subsection{Esercizio}
Un carrello sta su un piano inclinato.
Descriviamo questo piano inclinato con un triangolo rettangolo che ha cateto di base pari a $24\unit{m}$, cateto di altezza pari a $7\unit{m}$.
% e quindi ipotenusa pari a $25\unit{m}$.
Il carrello ha una massa $M=23\unit{kg}$.
Al carrello è agganciato un filo che, parallelo al piano inclinato, arriva in cima al piano inclinato e quindi scende verticale con una massa $m$ appesa.
Quanto deve valere $m$ per tenere in equilibrio il carrello?\\

%%%%%%%%%%%%%%%%%%%%%%%%%%%%%%%%%%%%%%%%%%%%%%%%%%%%%%%%%%%%%%%
\subsection{Esercizio}
Un carrello sta su un piano inclinato.
Descriviamo questo piano inclinato con un triangolo rettangolo che ha cateto di base pari a $4\unit{m}$, cateto di altezza pari a $3\unit{m}$ e quindi ipotenusa pari a $5\unit{m}$.
Il carrello ha una massa $M=20\unit{kg}$.
Al carrello è agganciata una molla con costante elastica $k=250 \unit{N/m}$.
Di quanto si allunga la molla?

\begin{center}
\includegraphics[width=.5\textwidth]{pics/statica-26}
\end{center}

%%%%%%%%%%%%%%%%%%%%%%%%%%%%%%%%%%%%%%%%%%%%%%%%%%%%%%%%%%%%%%%
\subsection{Esercizio}
Inventa un esercizio e risolvilo.


%%%%%%%%%%%%%%%%%%%%%%%%%%%%%%%%%%%%%%%%%%%%%%%%%%%%%%%%%%%%%%%
\subsection{L'attrito}
Quando un oggetto è premuto contro una superficie subisce una \emph{forza d'attrito (radente)}.
La forza d'attrito si oppone al moto con una intensità che dipende dalla forza a cui si oppone.
Per esempio, se un baule sta sul pavimento senza che nessuno lo spinga, la forza d'attrito è nulla.
Se però provo a spingerlo, inizialmente il baule non si sposta: la mia spinta viene controbilanciata dalla forza d'attrito.

Se aumento gradualmente la mia spinta, ad un certo punto il baule si sposta.
Cosa succede?
La forza d'attrito tra baule e pavimento può raggiungere un valore massimo.
Se io spingo con una forza superiore a quel valore massimo, la mia spinta vincerà sulla forza d'attrito e quindi il baule si muove.

La \emph{forza d'attrito statico massimale} $F_{a,\max}$ è direttamente proporzionale alla forza che preme l'oggetto contro la superficie, ossia
\begin{equation}
	F_{a,\max} = \mu_s F_n ,
\end{equation}
dove
$F_n$ è il modulo della forza perpendicolare che preme l'oggetto sulla superficie,
mentre la costante $\mu_s$ (``mü esse'') si chiama \emph{coefficiente d'attrito statico} e dipende dalle due superfici in contatto.
Il coefficiente d'attrito è un numero puro, ossia non ha unità di misura.

\begin{center}
\includegraphics[width=.75\textwidth]{pics/statica-27}
\end{center}

%%%%%%%%%%%%%%%%%%%%%%%%%%%%%%%%%%%%%%%%%%%%%%%%%%%%%%%%%%%%%%%
\subsection{Esempio}
Un libro sta su un tavolo orizzontale e viene premuto dalla mia mano.
Il peso del libro è $P$ mentre la forza impressa da me è $F_{\text{mano}}$.
Il coefficiente d'attrito statico tra libro e tavolo è $\mu_s$.
Quanto devo spingere il libro perché si scivoli lungo il tavolo?

La risposta è semplice.
Siccome la forza totale perpendicolare al tavolo è la somma del peso e della forza impressa dalla mia mano, ottengo
$F_n = P + F_{\text{mano}}$.
Per poter smuovere il libro devo spingere con una forza maggiore della forza d'attrito massimale, ossia
$F_{a,\max} = \mu_s F_n = \mu_s (P + F_{\text{mano}})$.

Per esempio, se $\mu_s = 0{,}23 $ il peso è $P=5\,\unit{N}$ e io premo con $F_{\text{mano}} = 4{,}5\,\unit{N}$,
allora la forza necessaria per smuovere il libro è
\begin{equation}
	F_{a,\max} = \mu_s(P + F_{\text{mano}})
	= 0{,}23 (5\,\unit{N} + 4{,}5\,\unit{N})
	= 2{,}185 \,\unit{N} .
\end{equation}
Per referenza, $2{,}185 \,\unit{N}$ è il peso di un oggetto di massa $223\,\unit{g}$ circa.

\begin{center}
\includegraphics[width=.75\textwidth]{pics/statica-28}
\end{center}


%%%%%%%%%%%%%%%%%%%%%%%%%%%%%%%%%%%%%%%%%%%%%%%%%%%%%%%%%%%%%%%
\subsection{Esercizio}
Studiamo la situazione dell'esempio in~\ref{subs695a37fb}, ma con l'aggiunta dell'attrito statico.

Un carrello sta su un piano inclinato.
Descriviamo questo piano inclinato con un triangolo rettangolo che ha cateto di base pari a $4\unit{m}$, cateto di altezza pari a $3\unit{m}$ e quindi ipotenusa pari a $5\unit{m}$.
Il carrello ha una massa $M=15\unit{kg}$.
Al carrello è agganciato un filo che, parallelo al piano inclinato, arriva in cima al piano inclinato e quindi scende verticale con una massa $m$ appesa.
Il carrello ha le ruote bloccate e quindi tra carrello e piano c'è una forza d'attrito con coefficiente d'attrito statico $\mu_s = 0{,}2$.

Quanto deve valere $m$ per tenere in equilibrio il carrello?
\\

%%%%%%%%%%%%%%%%%%%%%%%%%%%%%%%%%%%%%%%%%%%%%%%%%%%%%%%%%%%%%%%
\subsection{Esercizio $\star$}
Abbiamo due corpi di massa $m$ ciascuno appoggiati su un piano.
Il coefficiente d'attrito statico tra i corpi e la superficie è $\mu_s$.
Le due masse sono a distanza $L$ e sono unite da una molla.
La molla ha lunghezza a riposo $\ell$ e costante elastica $k_e$.
Quanto vale $L$ al massimo prima che i corpi vengano avvicinati dalla molla?
In altre parole, qual è la distanza massima $L_{\max}$ tra i due corpi per cui rimangono fermi?
\begin{center}
\includegraphics[width=.75\textwidth]{pics/statica-29}
\end{center}

Per avere dei numeri: 
$m= 0,{935} \,\unit{kg}$,
$\mu_s = 0{,}35$, 
$\ell = 10\,\unit{cm}$,
$k_e = 115\,\unit{N/m}$.










































\newpage
% !TEX encoding = UTF-8 Unicode
%!TEX root = FisMat.tex

\renewcommand{\pics}{pics}

%%%%%%%%%%%%%%%%%%%%%%%%%%%%%%%%%%%%%%%%%%%%%%%%%%%%%%%%%%%%%%%
\section{Conservazione della quantità di moto}

\subsection{Introduzione}
Hai studiato per un anno all'università di Austin, in Texas, e finalmente torni a casa.
Arrivi da Malpensa con il treno a Rovereto e la tua famiglia ti aspetta in stazione.
La tua sorellina ti corre incontro e ti salta addosso abbracciandoti forte.
Anche tuo papà, sopraffatto dall'emozione, ti corre addosso.
Se l'impatto con la sorellina è stato dolce, tuo papà ti stende.
Perché?
Siccome hai studiato fisica a Austin, e le leggi fisiche valgono là quanto qua, tu sai che, a pari velocità, tuo papà ha più quantità di moto della tua sorellina.

In questa unità studieremo la conservazione della quantità di moto e gli urti.

%%%%%%%%%%%%%%%%%%%%%%%%%%%%%%%%%%%%%%%%%%%%%%%%%%%%%%%%%%%%%%%
\subsection{Conservazione della quantità di moto}
Un corpo che ha massa $m$ e velocità $\vec v$, ha \emph{quantità di moto}
\begin{equation}\label{eq695ada53}
	\vec p = m \vec v .
\end{equation}

Il \emph{principio della conservazione della quantità di moto} dice che
{\it se la somma delle forze agenti su un sistema è zero, la quantità di moto totale del sistema è costante.}

%%%%%%%%%%%%%%%%%%%%%%%%%%%%%%%%%%%%%%%%%%%%%%%%%%%%%%%%%%%%%%%
\subsection{Esempio}
Il piccolo Pietro si è messo dentro un carrello della spesa.
La massa di Pietro è $m_p = 34\,\unit{kg}$ e la massa del carrello è $m_c = 12\,\unit{kg}$.
Il carrello è fermo.
Poi Pietro prende un pollo di massa $m_p = 1{,}3\,\unit{kg}$ e lo lancia fuori dal carrello.
La velocità del pollo è $v_p = 1{,}3\,\unit{m/s}$.
Con sorpresa di Pietro, il lancio del pollo fa muovere il carrello nella direzione opposta a velocità $v_c$.
\begin{enumerate}
\item	Quanto vale $v_c$?
\item	Il pollo cade nel carrello: dopo che il pollo è caduto nel carrello, la velocità di questo è $v_c'$. Quanto vale $v_c'$?
\end{enumerate}

\begin{center}
\includegraphics[width=.5\textwidth]{\pics/quantMoto-01}
\includegraphics[width=.45\textwidth]{\pics/quantMoto-02}
\end{center}


\begin{center}
\includegraphics[width=.4\textwidth]{\pics/quantMoto-03}
\end{center}

%%%%%%%%%%%%%%%%%%%%%%%%%%%%%%%%%%%%%%%%%%%%%%%%%%%%%%%%%%%%%%%
\subsection{Esercizio}
Due carrellini di massa $m$ ciascuno sono tenuti fermi su dei binari a una distanza $L$ l'uno dall'altro.
Tra loro c'è una molla compressa.
%La lunghezza a riposo della molla è $\ell$ e la costante elastica della molla è $k_e$.
I due carrellini vengono lasciati andare: se un carrellino raggiunge velocità $v_1$, quale sarà la velocità raggiunta dal secondo carrellino?

\begin{center}
\includegraphics[width=.75\textwidth]{\pics/quantMoto-04}
\end{center}

%%%%%%%%%%%%%%%%%%%%%%%%%%%%%%%%%%%%%%%%%%%%%%%%%%%%%%%%%%%%%%%
\subsection{Esercizio}
Due carrellini di massa $m_1$ e $m_2$ rispettivamente sono tenuti fermi su dei binari a una distanza $L$ l'uno dall'altro.
Tra loro c'è una molla compressa.
%La lunghezza a riposo della molla è $\ell$ e la costante elastica della molla è $k_e$.
I due carrellini vengono lasciati andare: se un carrellino raggiunge velocità $v_1$, quale sarà la velocità raggiunta dal secondo carrellino?


%%%%%%%%%%%%%%%%%%%%%%%%%%%%%%%%%%%%%%%%%%%%%%%%%%%%%%%%%%%%%%%
\subsection{Esercizio}
Due carrellini di massa $m_1$ e $m_2$ rispettivamente sono tenuti fermi su dei binari a una distanza $L$ l'uno dall'altro.
Tra loro c'è una molla compressa.
La lunghezza a riposo della molla è $\ell$ e la costante elastica della molla è $k_e$.
I due carrellini vengono lasciati andare: quale sarà la velocità raggiunta da dai due carrellini?

[Suggerimento:
hai due incognite (le due velocità), quindi ti servono due equazioni.
La conservazione della quantità di moto ti da una equazione.
Devi usare la conservazione dell'energia per ottenere una seconda equazione.]

%%%%%%%%%%%%%%%%%%%%%%%%%%%%%%%%%%%%%%%%%%%%%%%%%%%%%%%%%%%%%%%
\subsection{Esercizio}
Un carellino di massa $M$ sta fermo su dei binari.
Sul carrellino è montata uno scivolo di altezza $h$ e in cima allo scivolo sta una pallina di massa $m$.
La pallina rotola giù dallo scivolo e lascia il carrellino con velocità perfettamente orizzontale.
%Sapendo che la massa del carrellino (scivolo incluso) è $M$ e la massa della pallina è $m$, e sapendo che l'altezza dello scivolo è $h$
Che velocità raggiungono carrellino ($v_M$) e pallina ($v_m$)?

\begin{center}
\includegraphics[width=.4\textwidth]{\pics/quantMoto-05}
\includegraphics[width=.4\textwidth]{\pics/quantMoto-06}
\end{center}

NB! In questo esercizio è importante notare che la quantità di moto non si conserva, perché c'è una forza esterna che agisce sul sistema: la forza di gravità.
Però, la forza di gravità agisce verticalmente. 
Quindi nella direzione orizzontale la quantità di moto si conserva.

%%%%%%%%%%%%%%%%%%%%%%%%%%%%%%%%%%%%%%%%%%%%%%%%%%%%%%%%%%%%%%%
\subsection{Esercizio}
Rhtü è un alieno con un'astronave a molla.
L'astronave con Rhtü e tutto quanto ha massa totale $M$ e
sta viaggiando a velocità $\vec v$.
Per frenare, Rhtü lancia un proiettile nella direzione di $\vec v$.
\begin{enumerate}
\item	Se il proiettile ha massa $m$, a che velocità deve essere lanciato perché Rhtü si fermi?
\item	Se il proiettile viene lanciato usando una molla di costante elastica $k_e$, quanto questa molla deve essere compressa per lanciare il proiettile?
\end{enumerate}

\begin{center}
\includegraphics[width=.7\textwidth]{\pics/quantMoto-07}
\end{center}

%%%%%%%%%%%%%%%%%%%%%%%%%%%%%%%%%%%%%%%%%%%%%%%%%%%%%%%%%%%%%%%
\subsection{Esercizio}
Annamaria vuole misurare la velocità di un proiettile quando esce da un fucile.
Siccome il proiettile viaggia molto veloce e la sua velocità è difficile da misurare,
decide di costruire il seguente marchingegno.
Mette il fucile su un carrello di massa $M$ molto grande (compresa la massa del fucile).
Quindi, con il carrello fermo ma libero di muoversi, fa sparare il fucile parallelamente ai binari del carrello e misura la velocità $V$ del carrello dopo lo sparo.
Se il proiettile ha massa $m$, quanto veloce è stato sparato?

Se vuoi dei numeri: $M=34\unit{kg}$, $V = 0{,}3 \unit{m/s}$, $m=53\unit{g}$.

%%%%%%%%%%%%%%%%%%%%%%%%%%%%%%%%%%%%%%%%%%%%%%%%%%%%%%%%%%%%%%%
\subsection{Urti}
Un urto tra due o più corpi è un urto, uno scontro, una interazione veloce e confusa.
È molto difficile studiare un urto nel dettaglio, ma possiamo fare una buona analisi del prima e del dopo.
Per esempio, se due carrelli si scontrano a velocità $v$ ciascuno,
dopo l'urto i due carrelli avranno certe velocità $v_1$ e $v_2$.
Di per se, non si possono prevedere queste nuove velocità senza ulteriori informazioni.
Però ci sono delle regole che devono essere soddisfatte.

In tutti gli urti viene rispettato il principio della conservazione della quantità di moto.
Noi considereremo per lo più urti in assenza di forze esterne.
Di fatti, se pure ci possono essere forze esterne, nei momenti immediatamente prima e dopo un urto queste sono trascurabili rispetto alle forze impegnate dall'urto stesso.
Per questo motivo, la quantità di moto si conserva tra (immediatamente) prima e (immediatamente) dopo l'urto.

Non tutti gli urti conservano l'energia cinetica.
Solitamente, parte dell'energia cinetica si trasforma nell'urto in energia termica dovuta alla deformazione dei corpi.
Per esempio, se lascio cadere un pallone per terra, dopo il rimbalzo non tornerà all'altezza iniziale: lo scarto di altezza corrisponde all'energia persa nel rimbalzo.
Infatti, durante l'impatto con il terreno, il pallone si deforma e questa deformazione consuma energia.

Però, ci sono molte situazioni in cui l'energia cinetica persa nell'urto è trascurabile ai fini pratici.
In questi casi, possiamo assumere che l'energia cinetica si conserva.

Per chiarezza, distinguiamo tre tipi di urti:
\begin{enumerate}
\item	Urti elastici, in cui l'energia cinetica prima e dopo l'urto è la stessa.
\item	Urti anaelastici, in cui l'energia cinetica non si conserva.
\item	Urti perfettamente anaelastici, in cui i due corpi che si scontrano rimangono uniti dopo l'urto.
\end{enumerate}

Vedremo negli esempi che nel caso di urti elastici e perfettamente anaelastici saremo in grado di prevedere le velocità dopo l'urto.

%%%%%%%%%%%%%%%%%%%%%%%%%%%%%%%%%%%%%%%%%%%%%%%%%%%%%%%%%%%%%%%
\subsection{Esempio: urto perfettamente anaelastico}
Due carrelli stanno sugli stessi binari e corrono l'uno contro l'altro a velocità $v$.
Supponendo che le masse dei carrelli siano $m_1$ e $m_2$, e che l'urto sia perfettamente anaelastico, quale sarà la velocità dei due carrelli attaccati dopo l'urto?\\

\begin{center}
\includegraphics[width=.75\textwidth]{\pics/quantMoto-08}
\end{center}

Sappiamo che deve conservarsi la quantità di moto.
Quindi, se chiamiamo $v'$ la velocità dopo l'urto, avremo
\begin{equation}\label{eq695adb2a}
	m_1v - m_2v = (m_1+m_2)v' ,
\end{equation}
dove $m_1v$ è la quantità di moto del primo carrello (che viaggia con verso positivo),
$-m_2v$ è la quantità di moto del secondo carrello (che viaggia con verso negativo),
e $(m_1+m_2)v'$ è la quantità di moto dei due carrelli uniti dopo l'urto.

NB! 
La legge di conservazione della quantità di moto che abbiamo scritto in~\eqref{eq695ada53} è una equazione vettoriale.
Per questo motivo dobbiamo tenere un occhio sul segno delle velocità: se corre in un verso è positiva, se punta nell'altro verso è negativa.

Dalla relazione~\eqref{eq695adb2a} otteniamo immediatamente che
\begin{equation}
	v' = \frac{ m_1v - m_2v }{ m_1+m_2 } = \frac{ m_1 - m_2 }{ m_1+m_2 } v .
\end{equation}

Per esempio, se $m_1=m_2$, allora $v'=0$.
In altre parole, se due carrelli di massa uguale si scontrano in modo completamente anaelastico, allora rimarranno fermi lì dove si sono scontrati.

Se invece $m_1 > m_2$, quindi il primo carrello è più massivo del secondo,
allora $v'>0$. 
Se però $m_1 < m_2$, allora $v'<0$.
In altre parole, se i due carrelli hanno masse diverse, dopo lo scontro proseguiranno nel verso in cui andava il carrello più pesante.
Penso sia quello che ci aspettavamo, non credi anche tu?

%%%%%%%%%%%%%%%%%%%%%%%%%%%%%%%%%%%%%%%%%%%%%%%%%%%%%%%%%%%%%%%
\subsection{Esercizio}
Due carrelli stanno sugli stessi binari e corrono l'uno contro l'altro a velocità $v_1$ e $v_2$.
Le masse dei carrelli sono rispettivamente $m_1$ e $m_2$.
Supponendo che l'urto sia perfettamente anaelastico, quale sarà la velocità dei due carrelli attaccati dopo l'urto?

%%%%%%%%%%%%%%%%%%%%%%%%%%%%%%%%%%%%%%%%%%%%%%%%%%%%%%%%%%%%%%%
\subsection{Esercizio}
Due carrelli stanno sugli stessi binari e corrono l'uno contro l'altro a velocità $v$.
Le masse dei carrelli sono rispettivamente $m_1$ e $m_2$.
L'urto è anaelastico, e viene misurata la velocità $v_1'$ del primo carrello dopo l'urto.
Quale è la velocità del secondo carrello dopo l'urto?

%%%%%%%%%%%%%%%%%%%%%%%%%%%%%%%%%%%%%%%%%%%%%%%%%%%%%%%%%%%%%%%
\subsection{Esercizio}
Due carrelli stanno sugli stessi binari e corrono l'uno contro l'altro a velocità rispettivamente $v_1$ e $v_2$.
Le masse dei carrelli sono rispettivamente $m_1$ e $m_2$.
L'urto è anaelastico, e viene misurata la velocità $v_1'$ del primo carrello dopo l'urto.
Quale è la velocità del secondo carrello dopo l'urto?

%%%%%%%%%%%%%%%%%%%%%%%%%%%%%%%%%%%%%%%%%%%%%%%%%%%%%%%%%%%%%%%
\subsection{Esempio: urto elastico di due carrelli -- parte prima}
Due carrelli stanno sugli stessi binari e corrono l'uno contro l'altro a velocità rispettivamente $v_1$ e $v_2$ (possibilmente con segno).
Le masse dei carrelli sono rispettivamente $m_1$ e $m_2$.
L'urto è elastico.
Quale velocità avranno i due carrelli dopo l'urto?
\\

\begin{center}
\includegraphics[width=.75\textwidth]{\pics/quantMoto-09}
\end{center}

Denotiamo con $v_1'$ e $v_2'$ le velocità dei due carrelli dopo l'urto.
Abbiamo due incognite, quindi ci servono due equazioni.
Dedurremo due equazioni dal principio di conservazione della quantità di moto e, siccome l'urto è elastico, dalla conservazione dell'energia cinetica.

Per esprimere la quantità di moto, abbiamo bisogno di orientare la direzione lungo cui corrono i due carrelli.
Quindi scegliamo un \emph{versore} $\vec e$ che punta la direzione di moto del primo carrello.
Così, $\vec{v_1} = v_1 \vec e$ e $\vec v_2 = v_2\vec e$.

La quantità di moto dei due carrelli prima dell'urto è
\begin{equation}
	\vec p = m_1\vec{v_1} + m_2 \vec{v_2} 
	= ( m_1 v_1 + m_2 v_2 ) \vec{e} .
\end{equation}
La quantità di moto dei due carrelli dopo l'urto è
\begin{equation}
	\vec{p'} = m_1\vec{v_1'} + m_2 \vec{v_2'}
	= ( m_1 v_1' + m_2 v_2' ) \vec{e} .
\end{equation}
Nota che $v_1'$ e $v_2'$ possono avere segno negativo: non lo sappiamo ancora.

Il principio di conservazione della quantità di moto implica che $\vec p = \vec{p'}$
e quindi
\begin{equation}\label{eq695bcdc0}
	 m_1 v_1 + m_2 v_2 = m_1 v_1' + m_2 v_2' .
\end{equation}
Questa è la prima equazione di cui abbiamo bisogno.

L'energia cinetica del sistema prima dell'urto è
\begin{equation}
	E_c = \frac12 m_1 v_1^2 + \frac12 m_2 v_2^2 ,
\end{equation}
metre dopo l'urto è
\begin{equation}
	E_c' = \frac12 m_1 v_1'^2 + \frac12 m_2 v_2'^2 .
\end{equation}
Siccome l'urto è elastico, abbiamo $E_c = E_c'$, cioè
\begin{equation}\label{eq695bce45}
	\frac12 m_1 v_1^2 + \frac12 m_2 v_2^2
	=
	\frac12 m_1 v_1'^2 + \frac12 m_2 v_2'^2 .
\end{equation}
Questa è la seconda equazione di cui abbiamo bisogno.

Quindi $v_1'$ e $v_2'$ sono le soluzioni delle due equazioni~\eqref{eq695bcdc0} e~\eqref{eq695bce45} messe a sistema, cioè
\begin{equation}\label{eq695bcfb8}
	\begin{cases}
	m_1 v_1 + m_2 v_2 = m_1 v_1' + m_2 v_2' , \\
	\frac12 m_1 v_1^2 + \frac12 m_2 v_2^2 = \frac12 m_1 v_1'^2 + \frac12 m_2 v_2'^2 .
	\end{cases}
\end{equation}
Risolvere questo sistema di equazioni è un problema puramente di matematica.
Lo faremo più sotto, ma prima proviamo a risolverlo in alcuni casi semplificati.

%%%%%%%%%%%%%%%%%%%%%%%%%%%%%%%%%%%%%%%%%%%%%%%%%%%%%%%%%%%%%%%
\subsection{Esempio: urto elastico di due carrelli -- parte seconda}
Risolviamo il sistema~\eqref{eq695bcfb8} con alcune ipotesi ulteriori.
Assumiamo che $m = m_1 = m_2$ e $v = v_1 = -v_2$.
In questo caso, abbiamo $m_1 v_1 + m_2 v_2 = 0$ e $\frac12 m_1 v_1^2 + \frac12 m_2 v_2^2 = m v^2$.
Quindi il sistema di equazioni~\eqref{eq695bcfb8} diventa
\begin{equation}
	\begin{cases}
	0 = m v_1' + m v_2' , \\
	m v^2 = \frac12 m v_1'^2 + \frac12 m v_2'^2 .
	\end{cases}
\end{equation}
che, dividendo per $m$ entrambe le equazioni (NB: $m\neq0$), è equivalente a
\begin{equation}
	\begin{cases}
	0 = v_1' + v_2' , \\
	v^2 = \frac12 v_1'^2 + \frac12 v_2'^2 .
	\end{cases}
\end{equation}
Dalla prima equazione otteniamo $v_2' = -v_1'$.
Quindi la seconda equazione diventa $v^2 =  v_1'^2$, ossia $\abs{v_1'} = v$.
Così, abbiamo ottenuto due soluzioni:
una in cui $v_1' = v$ e $v_2' = -v$,
e l'altra in cui $v_1' = -v$ e $v_2' = v$.
Di queste due soluzioni (matematiche) del sistema di equazioni, una ha senso fisico, l'altra no.
Infatti, la prima soluzione ($v_1' = v$ e $v_2' = -v$) descrive una situazione in cui i due carrelli si sono attraversati e poi proseguono dritti: questo è fisicamente impossibile.
La seconda soluzione invece ($v_1' = -v$ e $v_2' = v$) descrive i due carrelli che scappano via in versi opposti rispetto a come sono arrivati: questo è quello che ci aspettiamo.

%%%%%%%%%%%%%%%%%%%%%%%%%%%%%%%%%%%%%%%%%%%%%%%%%%%%%%%%%%%%%%%
\subsection{Esempio: urto elastico di due carrelli -- parte terza}
Risolviamo il sistema~\eqref{eq695bcfb8} con alcune ipotesi ulteriori.
Assumiamo che $m = m_1 = m_2$.
In questo caso, dividendo per $m$ entrambe le equazioni (NB: $m\neq0$), il sistema di equazioni~\eqref{eq695bcfb8} diventa
\begin{equation}
	\begin{cases}
	v_1 + v_2 = v_1' + v_2' , \\
	\frac12 v_1^2 + \frac12 v_2^2 = \frac12 v_1'^2 + \frac12 v_2'^2 .
	\end{cases}
\end{equation}
Dalla prima equazione otteniamo $v_2' = v_1 + v_2 - v_1'$.
Usando questa relazione nella seconda equazione, otteniamo
\begin{equation}\label{eq695bd3ad}
	\frac12 v_1^2 + \frac12 v_2^2 = \frac12 v_1'^2 + \frac12 (v_1 + v_2 - v_1')^2 .
\end{equation}
Qua sembra che dobbiamo fare un po' di conti.
Ricordiamoci che l'incognita qui è $v_1'$, tutto il resto sono quantità date dal problema.
Quindi, per aiutarci nei conti, diamo dei nomi.
Chiamiamo $A = \frac12 v_1^2 + \frac12 v_2^2$ e $B=v_1 + v_2$.
Così, l'equazione~\ref{eq695bd3ad} diventa
\begin{equation}
	A = \frac12 v_1'^2 + \frac12 (B - v_1')^2 .
\end{equation}
Svolgiamo i conti:
\begin{align}
	A 
	&= \frac12 v_1'^2 + \frac12 (B - v_1')^2 \\
	&= \frac12 v_1'^2 + \frac12 (B^2 -2Bv_1' + v_1'^2) \\
	&= ( \frac12 + \frac12) v_1'^2 - Bv_1' + \frac12 B^2 \\
	&= v_1'^2 - Bv_1' + \frac12 B^2 .
\end{align}
Quindi, l'equazione~\eqref{eq695bd3ad} è equivalente a
\begin{equation}
	v_1'^2 - Bv_1' + \frac12 B^2 - A = 0 .
\end{equation}
Questo è un polinomio di secondo grado in $v_1'$ e quindi le sue soluzioni sono:
\begin{equation}
	v_1'{}_{\pm} = \frac{ B \pm \sqrt{ B^2 - 4(\frac12 B^2 - A) } }{ 2 } 
	= \frac{ B \pm \sqrt{ 4A - B^2 } }{ 2 } .
\end{equation}
Sostituendo $A$ e $B$ con i loro valori, otteniamo
\begin{equation}
	v_1'{}_{\pm} = \frac{ v_1 + v_2 \pm (v_1 - v_2) }{ 2 } .
\end{equation}
Come prima, abbiamo ottenuto due soluzioni:
\begin{align*}
	&v_1' = v_1 \text{ e } v_2' = -v_2 , \text{ oppure } \\
	&v_1' = -v_2 \text{ e } v_2' = v_1 .
\end{align*}
La prima soluzione non ha senso fisicamente perché suppone che i due carrelli si siano attraversati.
Quindi succede la seconda: nello scontro, i due carrelli ``si scambiano le velocità''.

%%%%%%%%%%%%%%%%%%%%%%%%%%%%%%%%%%%%%%%%%%%%%%%%%%%%%%%%%%%%%%%
\subsection{Esempio: urto elastico di due carrelli -- parte quarta}
Ritorniamo al sistema di equazioni~\eqref{eq695bcfb8}, senza ipotesi ulteriori.
Denotiamo $p=m_1 v_1 + m_2 v_2$ e $E = \frac12 m_1 v_1^2 + \frac12 m_2 v_2^2$, così che~\eqref{eq695bcfb8} diventa 
\begin{equation}\label{eq695bd982}
	\begin{cases}
	P = m_1 v_1' + m_2 v_2' , \\
	E = \frac12 m_1 v_1'^2 + \frac12 m_2 v_2'^2 .
	\end{cases}
\end{equation}
Dalla prima equazione otteniamo 
\begin{equation}
	v_2' = \frac{ P - m_1 v_1' }{ m_2 } .
\end{equation}
Usando questa relazione nella seconda equazione, otteniamo
\begin{align*}
	E 
	&= \frac12 m_1 v_1'^2 + \frac12 m_2 \left(\frac{ P - m_1 v_1' }{ m_2 }\right)^2 \\
	&= \frac12 m_1 v_1'^2 + \frac12 m_2 \left(\frac{ P^2 + m_1^2 v_1'^2 - 2m_1P v_1' }{ m_2^2 }\right) \\
	&= \frac12 m_1 v_1'^2 + \frac{P^2}{2m_2} + \frac{ m_1^2 v_1'^2 }{ 2m_2 } - \frac{ m_1P v_1' }{ m_2 } \\
	&= \left( \frac12 m_1 + \frac{ m_1^2 }{ 2m_2 } \right) v_1'^2 
		-  \frac{ m_1P }{ m_2 } v_1' + \frac{P^2}{2m_2} \\
	&= \frac12 \frac{ m_1 }{ m_2 } (m_1 + m_2) v_1'^2 
		-  P\frac{ m_1 }{ m_2 } v_1' + \frac{P^2}{2m_2} .
\end{align*}
Quindi, $v_1'$ è soluzione dell'equazione quadratica:
\begin{equation}
	\frac12 \frac{ m_1 }{ m_2 } (m_1 + m_2) v_1'^2 
		-  P\frac{ m_1 }{ m_2 } v_1' + \frac{P^2}{2m_2} - E = 0,
\end{equation}
e così
\begin{align*}
	v_1'{}_\pm 
	&= \frac{ P\frac{ m_1 }{ m_2 } \pm \sqrt{ \left( P\frac{ m_1 }{ m_2 } \right)^2 - 4 \frac12 \frac{ m_1 }{ m_2 } (m_1 + m_2) \left( \frac{P^2}{2m_2} - E \right) } }{ \frac{ m_1 }{ m_2 } (m_1 + m_2) } \\
	&= \frac{ P }{ m_1+m_2 }
		\pm \frac{ \sqrt{ P^2 \left(\frac{m_1^2}{m_2^2} - 2 \frac{m_1}{m_2} (m_1+m_2) \frac1{2m_2} \right) + 2E \frac{m_1}{m_2} (m_1+m_2) } }{ { \frac{ m_1 }{ m_2 } (m_1 + m_2) } } \\
	&= \frac{ P }{ m_1+m_2 }
		\pm \frac{ \sqrt{ \frac{m_1}{m_2} (2(m_1+m_2)E-P^2 ) } }{ { \frac{ m_1 }{ m_2 } (m_1 + m_2) } } .
%	&= \frac{ P }{ m_1+m_2 }
%		\pm \frac{ \sqrt{ 2E-\frac{P^2}{m_1+m_2} } }{ \sqrt{ \frac{ m_1 }{ m_2 } (m_1 + m_2) } } \\
\end{align*}
Inserendo i valori di $E$ e $P$ di nuovo in questa espressione, otteniamo (dopo qualche conto)
\begin{equation}
	2(m_1+m_2)E-P^2
	= m_1m_2 (v_1-v_2)^2
\end{equation}
e quindi (dopo ulteriori conti)
\begin{equation*}
 	v_1'{}_\pm
	= \begin{cases}
	v_1, \text{ oppure }\\
	\frac{m_1-m_2}{m_1+m_2} v_1 + \frac{2m_2}{m_1+m_2} v_2 .
	\end{cases}
\end{equation*}
Usando queste soluzioni per $v_1'{}$, otteniamo (dopo alcuni conti)
\begin{equation}
	v_2'{}_\pm
	= \begin{cases}
	v_2, \text{ oppure }\\
	\frac{2m_1}{m_1+m_2} v_1 + \frac{m_2-m_1}{m_1+m_2} v_2 .
	\end{cases}
\end{equation}

Abbiamo due soluzioni al problema, ma solo una è fisicamente valida.
La prima soluzione (ossia $v_1' = v_1$ e $v_2'=v_2$) non è fisicamente valida perché presuppone che i due carrelli si siano attraversati.
Quindi deve succedere la seconda situazione:
\begin{equation}
	v_1' = \frac{m_1-m_2}{m_1+m_2} v_1 + \frac{2m_2}{m_1+m_2} v_2
	\quad\text{e}\quad
	v_2' = \frac{2m_1}{m_1+m_2} v_1 + \frac{m_2-m_1}{m_1+m_2} v_2 .
\end{equation}

%%%%%%%%%%%%%%%%%%%%%%%%%%%%%%%%%%%%%%%%%%%%%%%%%%%%%%%%%%%%%%%
\subsection{Altri esercizi}
\begin{enumerate}
\item Biliardo
\item Carrelli si scontrano elasticamente a valle: quanto risalgono?
\end{enumerate}








\newpage
\appendix
% !TEX encoding = UTF-8 Unicode
%!TEX root = FisMat.tex

\section{Calcolo Simbolico}

\subsection{Fa qualcosa:}


\begin{enumerate}
\item
$a+a = $
\item
$a-a = $
\item
$a+\frac1a = $
\item
$a+b-a = $
\item
$a+2b+a = $
\item
$a+2(a+b) = $
\item
$(a+b) - a - b = $
\item
$a+b+b+a = $
\item
$a+b+b-a = $
\item
$(a+b)c = $
\item
$a(b+c) = $
\item
$c(a+b) = $
\item
$(a+b)c - c(a+b) = $
\item
$(a+b)(c+d) = $
\item
$(a+b) (x+2y) = $
\item
$(a+b) (a+b) = $
\item
$(a+b)^2 = $
\item
$(a+b)(a-b) = $
\item
$(a-b)(a-b) = $
\item
$(x-y)(x+y) = $
\item
$(x-y)^2 = $
\item
$(a+x)^3 = $
\item
$(a-x)^3 = $
\item
$(a+1)^2 = $
\item
$a^2-1 = $
\item
$a+b(a+b)-(a+b)^2 =$
\item
$a-b(a-b)-(a-b)^2 =$
\item
$a^2-(a+b)a = $
\item
$(x+x^2-2)y = $
\item
$x(x+1)-x^2 = $
\item
$R^2+(r-R)^2+2rR = $
\item
$\sqrt{a^2} = $
\item
$a(x-b)^2+2abx = $
\item
$a\frac1b = $
\item
$\frac1a + \frac1b = $
\item
$\frac{a+b}{c} = $
\item
$\frac{ab+b}{b} = $
\item
$\frac{a}{b} + c = $
\item
$\frac{a+1}{a} = $
\item
$\frac{a}{a+b} - 1 = $
\item
$\frac{x^2-y^2}{x^2+2xy+y^2} = $
\item
$\frac{y}{x+1}+\frac{x+1}{y} = $
\item
$\frac{a+a^2}{a}-1 = $
\end{enumerate}


%%%%%%%%%%%%%%%%%%%%%%%%%%%%%%%%%%%%%%%%%%%%%%%%%%%%%%%%%%%%%%%
\subsection{Espandi}
Espandere un'espressione algebrica significa scriverla come somma di termini semplici.
Per esempio, $(x+2)(x+1)+x3y$ si espande (e semplifica) a $x^2+3xy+3x+2$.
I termini semplici qui sono $x^2$, $3xy$, $3x$, e $2$.

\begin{enumerate}
\item	$(x+2)(x+1)+x3y$
\item	$(a+b)^2$
\item	$2x(a-2b)^2$
\item	$(a+b)(a-b)$
\item	$x(a+b)-y(a-b)+2xy$
\item	$a(a+b)-b(a-b)+2ab$
\item	$2(x+b)(a+y)-2(xy+ab)$
\item	$4x(a+b)(x-y^2)$
\end{enumerate}

%%%%%%%%%%%%%%%%%%%%%%%%%%%%%%%%%%%%%%%%%%%%%%%%%%%%%%%%%%%%%%%
\subsection{Esplicita}
Se abbiamo una equazione possiamo cercare di esplicitare una variabile.
Per esempio, se abbiamo l'equazione 
\begin{equation}\label{eq695cd2e5}
	-4x+a^2(x+2)=a^2-1 ,
\end{equation}
possiamo esplicitare $x$:
\begin{equation}\label{eq695cd2f5}
	x=-\frac{1+a^2}{a^2-4} .
\end{equation}
Questo passaggio non è immediato, ma servono alcuni passaggi intermedi che ti invito a fare su un foglio.

È importante imparare a fare questi passaggi in modo naturale.
È anche importante imparare le sottigliezze di questi passaggi, perché ogni passaggio necessita di alcune condizioni.
Per esempio, nell'espressione che ho scritto per $x$, c'è $a^2-4$ al denominatore.
Siccome non abbiamo condizioni sui valori di $a$, dobbiamo prendere in considerazione il caso in cui $a$ ha un valore per il quale $a^2-4=0$.
Per esempio, $a$ potrebbe assumere i valori 2 o $-2$.
In questo caso, l'espressione che ho scritto per $x$ non ha più senso: stiamo dividendo per zero.
Come è possibile che da una espressione senza condizioni come l'equazione~\eqref{eq695cd2e5}
siamo arrivati a una equazione con condizioni come in~\eqref{eq695cd2f5}?
La risposta è: c'è stato almeno un passaggio in cui abbiamo usato quella condizioni.

Vediamo quindi come si passa da~\eqref{eq695cd2e5} a~\eqref{eq695cd2f5}.
A lato do una giustificazione del passaggio.
La sigla ``PADC'' sta per ``proprietà associativa, distributiva, commutativa'' e si riferisce alle proprietà elementari di somma e moltiplicazioni.
La sigla ``e.l.'' sta per ``entrambi i lati dell'equazione''
\begin{align}
	&-4x+a^2(x+2)=a^2-1 \\
	\IFF & (-4+a^2)x + 2a^2 = a^2-1 \qquad(\text{per PADC}) \\
	\IFF & (-4+a^2)x = a^2-1 - 2a^2 \qquad(\text{aggiungo $- 2a^2$ a e.l.}) \\
	\IFF & (-4+a^2)x = -(1 + a^2) \qquad(\text{per PADC}) \\
%	\IFF & x = -\frac{1+a^2}{a^2-4} && \text{divido per $a^2-4$, SE diverso da zero} \\
	\IFF & 
	\begin{cases}
		x = -\frac{1+a^2}{a^2-4} & \text{divido per $a^2-4$ a e.l., SE diverso da zero,} \\
		1 + a^2 = 0 & \text{SE $a^2-4=0$.}
	\end{cases}
\end{align}
Questo è quindi il vero risultato.
Nel caso in cui $a^2-4=0$, allora $1+a^2 = 1+a^2+(4-4) = 1+(a^2-4)+4 = 5$ e quindi $1 + a^2 = 0$ è falso.
In altre parole, se $a^2-4=0$, allora~\eqref{eq695cd2e5} è falsa per qualunque valore di $x$.
In conclusione
\begin{equation}
	-4x+a^2(x+2)=a^2-1
	\quad\IFF\quad
	a^2-4\neq0, \text{ e }x = -\frac{1+a^2}{a^2-4} .
\end{equation}

In ogni passaggio, le operazioni che possiamo compiere sono:
\begin{center}
\begin{tabular}{|c|c|}
\hline
Operazione & Condizioni \\
\hline
PADC & nessuna \\
Aggiungere {\it qlcs} a e.l. & nessuna \\
Moltiplicare {\it qlcs} a e.l. & {\it qlcs} deve essere non zero \\
Dividere per {\it qlcs} a e.l. & {\it qlcs} deve essere non zero \\
\hline
\end{tabular}
\end{center}

Esercizi:
\begin{enumerate}
\item	Esplicita $x$ in $x+2=0$
\item	Esplicita $x$ in $x+2=a$
\item	Esplicita $x$ in $x+a=b$
\item	Esplicita $a$ in $x+a=b$
\item	Esplicita $x$ in $(x+2)b=c$
\end{enumerate}










%%%%%%%%%%%%%%%%%%%%%%%%%%%%%%%%%%%%%%%%%%%%%%%%%%%%%%%%%%%%%%%
%%%%%%%%%%%%%%%%%%%%%%%%%%%%%%%%%%%%%%%%%%%%%%%%%%%%%%%%%%%%%%%


%\printindex
\printbibliography
\end{document}
