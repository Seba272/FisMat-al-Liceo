% !TEX encoding = UTF-8 Unicode
%!TEX root = FisMat.tex

\section{Calcolo Simbolico}

\subsection{Fa qualcosa:}


\begin{enumerate}
\item
$a+a = $
\item
$a-a = $
\item
$a+\frac1a = $
\item
$a+b-a = $
\item
$a+2b+a = $
\item
$a+2(a+b) = $
\item
$(a+b) - a - b = $
\item
$a+b+b+a = $
\item
$a+b+b-a = $
\item
$(a+b)c = $
\item
$a(b+c) = $
\item
$c(a+b) = $
\item
$(a+b)c - c(a+b) = $
\item
$(a+b)(c+d) = $
\item
$(a+b) (x+2y) = $
\item
$(a+b) (a+b) = $
\item
$(a+b)^2 = $
\item
$(a+b)(a-b) = $
\item
$(a-b)(a-b) = $
\item
$(x-y)(x+y) = $
\item
$(x-y)^2 = $
\item
$(a+x)^3 = $
\item
$(a-x)^3 = $
\item
$(a+1)^2 = $
\item
$a^2-1 = $
\item
$a+b(a+b)-(a+b)^2 =$
\item
$a-b(a-b)-(a-b)^2 =$
\item
$a^2-(a+b)a = $
\item
$(x+x^2-2)y = $
\item
$x(x+1)-x^2 = $
\item
$R^2+(r-R)^2+2rR = $
\item
$\sqrt{a^2} = $
\item
$a(x-b)^2+2abx = $
\item
$a\frac1b = $
\item
$\frac1a + \frac1b = $
\item
$\frac{a+b}{c} = $
\item
$\frac{ab+b}{b} = $
\item
$\frac{a}{b} + c = $
\item
$\frac{a+1}{a} = $
\item
$\frac{a}{a+b} - 1 = $
\item
$\frac{x^2-y^2}{x^2+2xy+y^2} = $
\item
$\frac{y}{x+1}+\frac{x+1}{y} = $
\item
$\frac{a+a^2}{a}-1 = $
\end{enumerate}


%%%%%%%%%%%%%%%%%%%%%%%%%%%%%%%%%%%%%%%%%%%%%%%%%%%%%%%%%%%%%%%
\subsection{Espandi}
Espandere un'espressione algebrica significa scriverla come somma di termini semplici.
Per esempio, $(x+2)(x+1)+x3y$ si espande (e semplifica) a $x^2+3xy+3x+2$.
I termini semplici qui sono $x^2$, $3xy$, $3x$, e $2$.

\begin{enumerate}
\item	$(x+2)(x+1)+x3y$
\item	$(a+b)^2$
\item	$2x(a-2b)^2$
\item	$(a+b)(a-b)$
\item	$x(a+b)-y(a-b)+2xy$
\item	$a(a+b)-b(a-b)+2ab$
\item	$2(x+b)(a+y)-2(xy+ab)$
\item	$4x(a+b)(x-y^2)$
\end{enumerate}

%%%%%%%%%%%%%%%%%%%%%%%%%%%%%%%%%%%%%%%%%%%%%%%%%%%%%%%%%%%%%%%
\subsection{Esplicita}
Se abbiamo una equazione possiamo cercare di esplicitare una variabile.
Per esempio, se abbiamo l'equazione 
\begin{equation}\label{eq695cd2e5}
	-4x+a^2(x+2)=a^2-1 ,
\end{equation}
possiamo esplicitare $x$:
\begin{equation}\label{eq695cd2f5}
	x=-\frac{1+a^2}{a^2-4} .
\end{equation}
Questo passaggio non è immediato, ma servono alcuni passaggi intermedi che ti invito a fare su un foglio.

È importante imparare a fare questi passaggi in modo naturale.
È anche importante imparare le sottigliezze di questi passaggi, perché ogni passaggio necessita di alcune condizioni.
Per esempio, nell'espressione che ho scritto per $x$, c'è $a^2-4$ al denominatore.
Siccome non abbiamo condizioni sui valori di $a$, dobbiamo prendere in considerazione il caso in cui $a$ ha un valore per il quale $a^2-4=0$.
Per esempio, $a$ potrebbe assumere i valori 2 o $-2$.
In questo caso, l'espressione che ho scritto per $x$ non ha più senso: stiamo dividendo per zero.
Come è possibile che da una espressione senza condizioni come l'equazione~\eqref{eq695cd2e5}
siamo arrivati a una equazione con condizioni come in~\eqref{eq695cd2f5}?
La risposta è: c'è stato almeno un passaggio in cui abbiamo usato quella condizioni.

Vediamo quindi come si passa da~\eqref{eq695cd2e5} a~\eqref{eq695cd2f5}.
A lato do una giustificazione del passaggio.
La sigla ``PADC'' sta per ``proprietà associativa, distributiva, commutativa'' e si riferisce alle proprietà elementari di somma e moltiplicazioni.
La sigla ``e.l.'' sta per ``entrambi i lati dell'equazione''
\begin{align}
	&-4x+a^2(x+2)=a^2-1 \\
	\IFF & (-4+a^2)x + 2a^2 = a^2-1 \qquad(\text{per PADC}) \\
	\IFF & (-4+a^2)x = a^2-1 - 2a^2 \qquad(\text{aggiungo $- 2a^2$ a e.l.}) \\
	\IFF & (-4+a^2)x = -(1 + a^2) \qquad(\text{per PADC}) \\
%	\IFF & x = -\frac{1+a^2}{a^2-4} && \text{divido per $a^2-4$, SE diverso da zero} \\
	\IFF & 
	\begin{cases}
		x = -\frac{1+a^2}{a^2-4} & \text{divido per $a^2-4$ a e.l., SE diverso da zero,} \\
		1 + a^2 = 0 & \text{SE $a^2-4=0$.}
	\end{cases}
\end{align}
Questo è quindi il vero risultato.
Nel caso in cui $a^2-4=0$, allora $1+a^2 = 1+a^2+(4-4) = 1+(a^2-4)+4 = 5$ e quindi $1 + a^2 = 0$ è falso.
In altre parole, se $a^2-4=0$, allora~\eqref{eq695cd2e5} è falsa per qualunque valore di $x$.
In conclusione
\begin{equation}
	-4x+a^2(x+2)=a^2-1
	\quad\IFF\quad
	a^2-4\neq0, \text{ e }x = -\frac{1+a^2}{a^2-4} .
\end{equation}

In ogni passaggio, le operazioni che possiamo compiere sono:
\begin{center}
\begin{tabular}{|c|c|}
\hline
Operazione & Condizioni \\
\hline
PADC & nessuna \\
Aggiungere {\it qlcs} a e.l. & nessuna \\
Moltiplicare {\it qlcs} a e.l. & {\it qlcs} deve essere non zero \\
Dividere per {\it qlcs} a e.l. & {\it qlcs} deve essere non zero \\
\hline
\end{tabular}
\end{center}

Esercizi:
\begin{enumerate}
\item	Esplicita $x$ in $x+2=0$
\item	Esplicita $x$ in $x+2=a$
\item	Esplicita $x$ in $x+a=b$
\item	Esplicita $a$ in $x+a=b$
\item	Esplicita $x$ in $(x+2)b=c$
\end{enumerate}








