% !TEX encoding = UTF-8 Unicode
%!TEX root = FisMat.tex

\renewcommand{\pics}{pics}

%%%%%%%%%%%%%%%%%%%%%%%%%%%%%%%%%%%%%%%%%%%%%%%%%%%%%%%%%%%%%%%
\section{Conservazione della quantità di moto}

\subsection{Introduzione}
Hai studiato per un anno all'università di Austin, in Texas, e finalmente torni a casa.
Arrivi da Malpensa con il treno a Rovereto e la tua famiglia ti aspetta in stazione.
La tua sorellina ti corre incontro e ti salta addosso abbracciandoti forte.
Anche tuo papà, sopraffatto dall'emozione, ti corre addosso.
Se l'impatto con la sorellina è stato dolce, tuo papà ti stende.
Perché?
Siccome hai studiato fisica a Austin, e le leggi fisiche valgono là quanto qua, tu sai che, a pari velocità, tuo papà ha più quantità di moto della tua sorellina.

In questa unità studieremo la conservazione della quantità di moto e gli urti.

%%%%%%%%%%%%%%%%%%%%%%%%%%%%%%%%%%%%%%%%%%%%%%%%%%%%%%%%%%%%%%%
\subsection{Conservazione della quantità di moto}
Un corpo che ha massa $m$ e velocità $\vec v$, ha \emph{quantità di moto}
\begin{equation}\label{eq695ada53}
	\vec p = m \vec v .
\end{equation}

Il \emph{principio della conservazione della quantità di moto} dice che
{\it se la somma delle forze agenti su un sistema è zero, la quantità di moto totale del sistema è costante.}

%%%%%%%%%%%%%%%%%%%%%%%%%%%%%%%%%%%%%%%%%%%%%%%%%%%%%%%%%%%%%%%
\subsection{Esempio}
Il piccolo Pietro si è messo dentro un carrello della spesa.
La massa di Pietro è $m_p = 34\,\unit{kg}$ e la massa del carrello è $m_c = 12\,\unit{kg}$.
Il carrello è fermo.
Poi Pietro prende un pollo di massa $m_p = 1{,}3\,\unit{kg}$ e lo lancia fuori dal carrello.
La velocità del pollo è $v_p = 1{,}3\,\unit{m/s}$.
Con sorpresa di Pietro, il lancio del pollo fa muovere il carrello nella direzione opposta a velocità $v_c$.
\begin{enumerate}
\item	Quanto vale $v_c$?
\item	Il pollo cade nel carrello: dopo che il pollo è caduto nel carrello, la velocità di questo è $v_c'$. Quanto vale $v_c'$?
\end{enumerate}

\begin{center}
\includegraphics[width=.5\textwidth]{\pics/quantMoto-01}
\includegraphics[width=.45\textwidth]{\pics/quantMoto-02}
\end{center}


\begin{center}
\includegraphics[width=.4\textwidth]{\pics/quantMoto-03}
\end{center}

%%%%%%%%%%%%%%%%%%%%%%%%%%%%%%%%%%%%%%%%%%%%%%%%%%%%%%%%%%%%%%%
\subsection{Esercizio}
Due carrellini di massa $m$ ciascuno sono tenuti fermi su dei binari a una distanza $L$ l'uno dall'altro.
Tra loro c'è una molla compressa.
%La lunghezza a riposo della molla è $\ell$ e la costante elastica della molla è $k_e$.
I due carrellini vengono lasciati andare: se un carrellino raggiunge velocità $v_1$, quale sarà la velocità raggiunta dal secondo carrellino?

\begin{center}
\includegraphics[width=.75\textwidth]{\pics/quantMoto-04}
\end{center}

%%%%%%%%%%%%%%%%%%%%%%%%%%%%%%%%%%%%%%%%%%%%%%%%%%%%%%%%%%%%%%%
\subsection{Esercizio}
Due carrellini di massa $m_1$ e $m_2$ rispettivamente sono tenuti fermi su dei binari a una distanza $L$ l'uno dall'altro.
Tra loro c'è una molla compressa.
%La lunghezza a riposo della molla è $\ell$ e la costante elastica della molla è $k_e$.
I due carrellini vengono lasciati andare: se un carrellino raggiunge velocità $v_1$, quale sarà la velocità raggiunta dal secondo carrellino?


%%%%%%%%%%%%%%%%%%%%%%%%%%%%%%%%%%%%%%%%%%%%%%%%%%%%%%%%%%%%%%%
\subsection{Esercizio}
Due carrellini di massa $m_1$ e $m_2$ rispettivamente sono tenuti fermi su dei binari a una distanza $L$ l'uno dall'altro.
Tra loro c'è una molla compressa.
La lunghezza a riposo della molla è $\ell$ e la costante elastica della molla è $k_e$.
I due carrellini vengono lasciati andare: quale sarà la velocità raggiunta da dai due carrellini?

[Suggerimento:
hai due incognite (le due velocità), quindi ti servono due equazioni.
La conservazione della quantità di moto ti da una equazione.
Devi usare la conservazione dell'energia per ottenere una seconda equazione.]

%%%%%%%%%%%%%%%%%%%%%%%%%%%%%%%%%%%%%%%%%%%%%%%%%%%%%%%%%%%%%%%
\subsection{Esercizio}
Un carellino di massa $M$ sta fermo su dei binari.
Sul carrellino è montata uno scivolo di altezza $h$ e in cima allo scivolo sta una pallina di massa $m$.
La pallina rotola giù dallo scivolo e lascia il carrellino con velocità perfettamente orizzontale.
%Sapendo che la massa del carrellino (scivolo incluso) è $M$ e la massa della pallina è $m$, e sapendo che l'altezza dello scivolo è $h$
Che velocità raggiungono carrellino ($v_M$) e pallina ($v_m$)?

\begin{center}
\includegraphics[width=.4\textwidth]{\pics/quantMoto-05}
\includegraphics[width=.4\textwidth]{\pics/quantMoto-06}
\end{center}

NB! In questo esercizio è importante notare che la quantità di moto non si conserva, perché c'è una forza esterna che agisce sul sistema: la forza di gravità.
Però, la forza di gravità agisce verticalmente. 
Quindi nella direzione orizzontale la quantità di moto si conserva.

%%%%%%%%%%%%%%%%%%%%%%%%%%%%%%%%%%%%%%%%%%%%%%%%%%%%%%%%%%%%%%%
\subsection{Esercizio}
Rhtü è un alieno con un'astronave a molla.
L'astronave con Rhtü e tutto quanto ha massa totale $M$ e
sta viaggiando a velocità $\vec v$.
Per frenare, Rhtü lancia un proiettile nella direzione di $\vec v$.
\begin{enumerate}
\item	Se il proiettile ha massa $m$, a che velocità deve essere lanciato perché Rhtü si fermi?
\item	Se il proiettile viene lanciato usando una molla di costante elastica $k_e$, quanto questa molla deve essere compressa per lanciare il proiettile?
\end{enumerate}

\begin{center}
\includegraphics[width=.7\textwidth]{\pics/quantMoto-07}
\end{center}

%%%%%%%%%%%%%%%%%%%%%%%%%%%%%%%%%%%%%%%%%%%%%%%%%%%%%%%%%%%%%%%
\subsection{Esercizio}
Annamaria vuole misurare la velocità di un proiettile quando esce da un fucile.
Siccome il proiettile viaggia molto veloce e la sua velocità è difficile da misurare,
decide di costruire il seguente marchingegno.
Mette il fucile su un carrello di massa $M$ molto grande (compresa la massa del fucile).
Quindi, con il carrello fermo ma libero di muoversi, fa sparare il fucile parallelamente ai binari del carrello e misura la velocità $V$ del carrello dopo lo sparo.
Se il proiettile ha massa $m$, quanto veloce è stato sparato?

Se vuoi dei numeri: $M=34\unit{kg}$, $V = 0{,}3 \unit{m/s}$, $m=53\unit{g}$.

%%%%%%%%%%%%%%%%%%%%%%%%%%%%%%%%%%%%%%%%%%%%%%%%%%%%%%%%%%%%%%%
\subsection{Urti}
Un urto tra due o più corpi è un urto, uno scontro, una interazione veloce e confusa.
È molto difficile studiare un urto nel dettaglio, ma possiamo fare una buona analisi del prima e del dopo.
Per esempio, se due carrelli si scontrano a velocità $v$ ciascuno,
dopo l'urto i due carrelli avranno certe velocità $v_1$ e $v_2$.
Di per se, non si possono prevedere queste nuove velocità senza ulteriori informazioni.
Però ci sono delle regole che devono essere soddisfatte.

In tutti gli urti viene rispettato il principio della conservazione della quantità di moto.
Noi considereremo per lo più urti in assenza di forze esterne.
Di fatti, se pure ci possono essere forze esterne, nei momenti immediatamente prima e dopo un urto queste sono trascurabili rispetto alle forze impegnate dall'urto stesso.
Per questo motivo, la quantità di moto si conserva tra (immediatamente) prima e (immediatamente) dopo l'urto.

Non tutti gli urti conservano l'energia cinetica.
Solitamente, parte dell'energia cinetica si trasforma nell'urto in energia termica dovuta alla deformazione dei corpi.
Per esempio, se lascio cadere un pallone per terra, dopo il rimbalzo non tornerà all'altezza iniziale: lo scarto di altezza corrisponde all'energia persa nel rimbalzo.
Infatti, durante l'impatto con il terreno, il pallone si deforma e questa deformazione consuma energia.

Però, ci sono molte situazioni in cui l'energia cinetica persa nell'urto è trascurabile ai fini pratici.
In questi casi, possiamo assumere che l'energia cinetica si conserva.

Per chiarezza, distinguiamo tre tipi di urti:
\begin{enumerate}
\item	Urti elastici, in cui l'energia cinetica prima e dopo l'urto è la stessa.
\item	Urti anaelastici, in cui l'energia cinetica non si conserva.
\item	Urti perfettamente anaelastici, in cui i due corpi che si scontrano rimangono uniti dopo l'urto.
\end{enumerate}

Vedremo negli esempi che nel caso di urti elastici e perfettamente anaelastici saremo in grado di prevedere le velocità dopo l'urto.

%%%%%%%%%%%%%%%%%%%%%%%%%%%%%%%%%%%%%%%%%%%%%%%%%%%%%%%%%%%%%%%
\subsection{Esempio: urto perfettamente anaelastico}
Due carrelli stanno sugli stessi binari e corrono l'uno contro l'altro a velocità $v$.
Supponendo che le masse dei carrelli siano $m_1$ e $m_2$, e che l'urto sia perfettamente anaelastico, quale sarà la velocità dei due carrelli attaccati dopo l'urto?\\

\begin{center}
\includegraphics[width=.75\textwidth]{\pics/quantMoto-08}
\end{center}

Sappiamo che deve conservarsi la quantità di moto.
Quindi, se chiamiamo $v'$ la velocità dopo l'urto, avremo
\begin{equation}\label{eq695adb2a}
	m_1v - m_2v = (m_1+m_2)v' ,
\end{equation}
dove $m_1v$ è la quantità di moto del primo carrello (che viaggia con verso positivo),
$-m_2v$ è la quantità di moto del secondo carrello (che viaggia con verso negativo),
e $(m_1+m_2)v'$ è la quantità di moto dei due carrelli uniti dopo l'urto.

NB! 
La legge di conservazione della quantità di moto che abbiamo scritto in~\eqref{eq695ada53} è una equazione vettoriale.
Per questo motivo dobbiamo tenere un occhio sul segno delle velocità: se corre in un verso è positiva, se punta nell'altro verso è negativa.

Dalla relazione~\eqref{eq695adb2a} otteniamo immediatamente che
\begin{equation}
	v' = \frac{ m_1v - m_2v }{ m_1+m_2 } = \frac{ m_1 - m_2 }{ m_1+m_2 } v .
\end{equation}

Per esempio, se $m_1=m_2$, allora $v'=0$.
In altre parole, se due carrelli di massa uguale si scontrano in modo completamente anaelastico, allora rimarranno fermi lì dove si sono scontrati.

Se invece $m_1 > m_2$, quindi il primo carrello è più massivo del secondo,
allora $v'>0$. 
Se però $m_1 < m_2$, allora $v'<0$.
In altre parole, se i due carrelli hanno masse diverse, dopo lo scontro proseguiranno nel verso in cui andava il carrello più pesante.
Penso sia quello che ci aspettavamo, non credi anche tu?

%%%%%%%%%%%%%%%%%%%%%%%%%%%%%%%%%%%%%%%%%%%%%%%%%%%%%%%%%%%%%%%
\subsection{Esercizio}
Due carrelli stanno sugli stessi binari e corrono l'uno contro l'altro a velocità $v_1$ e $v_2$.
Le masse dei carrelli sono rispettivamente $m_1$ e $m_2$.
Supponendo che l'urto sia perfettamente anaelastico, quale sarà la velocità dei due carrelli attaccati dopo l'urto?

%%%%%%%%%%%%%%%%%%%%%%%%%%%%%%%%%%%%%%%%%%%%%%%%%%%%%%%%%%%%%%%
\subsection{Esercizio}
Due carrelli stanno sugli stessi binari e corrono l'uno contro l'altro a velocità $v$.
Le masse dei carrelli sono rispettivamente $m_1$ e $m_2$.
L'urto è anaelastico, e viene misurata la velocità $v_1'$ del primo carrello dopo l'urto.
Quale è la velocità del secondo carrello dopo l'urto?

%%%%%%%%%%%%%%%%%%%%%%%%%%%%%%%%%%%%%%%%%%%%%%%%%%%%%%%%%%%%%%%
\subsection{Esercizio}
Due carrelli stanno sugli stessi binari e corrono l'uno contro l'altro a velocità rispettivamente $v_1$ e $v_2$.
Le masse dei carrelli sono rispettivamente $m_1$ e $m_2$.
L'urto è anaelastico, e viene misurata la velocità $v_1'$ del primo carrello dopo l'urto.
Quale è la velocità del secondo carrello dopo l'urto?

%%%%%%%%%%%%%%%%%%%%%%%%%%%%%%%%%%%%%%%%%%%%%%%%%%%%%%%%%%%%%%%
\subsection{Esempio: urto elastico di due carrelli -- parte prima}
Due carrelli stanno sugli stessi binari e corrono l'uno contro l'altro a velocità rispettivamente $v_1$ e $v_2$ (possibilmente con segno).
Le masse dei carrelli sono rispettivamente $m_1$ e $m_2$.
L'urto è elastico.
Quale velocità avranno i due carrelli dopo l'urto?
\\

\begin{center}
\includegraphics[width=.75\textwidth]{\pics/quantMoto-09}
\end{center}

Denotiamo con $v_1'$ e $v_2'$ le velocità dei due carrelli dopo l'urto.
Abbiamo due incognite, quindi ci servono due equazioni.
Dedurremo due equazioni dal principio di conservazione della quantità di moto e, siccome l'urto è elastico, dalla conservazione dell'energia cinetica.

Per esprimere la quantità di moto, abbiamo bisogno di orientare la direzione lungo cui corrono i due carrelli.
Quindi scegliamo un \emph{versore} $\vec e$ che punta la direzione di moto del primo carrello.
Così, $\vec{v_1} = v_1 \vec e$ e $\vec v_2 = v_2\vec e$.

La quantità di moto dei due carrelli prima dell'urto è
\begin{equation}
	\vec p = m_1\vec{v_1} + m_2 \vec{v_2} 
	= ( m_1 v_1 + m_2 v_2 ) \vec{e} .
\end{equation}
La quantità di moto dei due carrelli dopo l'urto è
\begin{equation}
	\vec{p'} = m_1\vec{v_1'} + m_2 \vec{v_2'}
	= ( m_1 v_1' + m_2 v_2' ) \vec{e} .
\end{equation}
Nota che $v_1'$ e $v_2'$ possono avere segno negativo: non lo sappiamo ancora.

Il principio di conservazione della quantità di moto implica che $\vec p = \vec{p'}$
e quindi
\begin{equation}\label{eq695bcdc0}
	 m_1 v_1 + m_2 v_2 = m_1 v_1' + m_2 v_2' .
\end{equation}
Questa è la prima equazione di cui abbiamo bisogno.

L'energia cinetica del sistema prima dell'urto è
\begin{equation}
	E_c = \frac12 m_1 v_1^2 + \frac12 m_2 v_2^2 ,
\end{equation}
metre dopo l'urto è
\begin{equation}
	E_c' = \frac12 m_1 v_1'^2 + \frac12 m_2 v_2'^2 .
\end{equation}
Siccome l'urto è elastico, abbiamo $E_c = E_c'$, cioè
\begin{equation}\label{eq695bce45}
	\frac12 m_1 v_1^2 + \frac12 m_2 v_2^2
	=
	\frac12 m_1 v_1'^2 + \frac12 m_2 v_2'^2 .
\end{equation}
Questa è la seconda equazione di cui abbiamo bisogno.

Quindi $v_1'$ e $v_2'$ sono le soluzioni delle due equazioni~\eqref{eq695bcdc0} e~\eqref{eq695bce45} messe a sistema, cioè
\begin{equation}\label{eq695bcfb8}
	\begin{cases}
	m_1 v_1 + m_2 v_2 = m_1 v_1' + m_2 v_2' , \\
	\frac12 m_1 v_1^2 + \frac12 m_2 v_2^2 = \frac12 m_1 v_1'^2 + \frac12 m_2 v_2'^2 .
	\end{cases}
\end{equation}
Risolvere questo sistema di equazioni è un problema puramente di matematica.
Lo faremo più sotto, ma prima proviamo a risolverlo in alcuni casi semplificati.

%%%%%%%%%%%%%%%%%%%%%%%%%%%%%%%%%%%%%%%%%%%%%%%%%%%%%%%%%%%%%%%
\subsection{Esempio: urto elastico di due carrelli -- parte seconda}
Risolviamo il sistema~\eqref{eq695bcfb8} con alcune ipotesi ulteriori.
Assumiamo che $m = m_1 = m_2$ e $v = v_1 = -v_2$.
In questo caso, abbiamo $m_1 v_1 + m_2 v_2 = 0$ e $\frac12 m_1 v_1^2 + \frac12 m_2 v_2^2 = m v^2$.
Quindi il sistema di equazioni~\eqref{eq695bcfb8} diventa
\begin{equation}
	\begin{cases}
	0 = m v_1' + m v_2' , \\
	m v^2 = \frac12 m v_1'^2 + \frac12 m v_2'^2 .
	\end{cases}
\end{equation}
che, dividendo per $m$ entrambe le equazioni (NB: $m\neq0$), è equivalente a
\begin{equation}
	\begin{cases}
	0 = v_1' + v_2' , \\
	v^2 = \frac12 v_1'^2 + \frac12 v_2'^2 .
	\end{cases}
\end{equation}
Dalla prima equazione otteniamo $v_2' = -v_1'$.
Quindi la seconda equazione diventa $v^2 =  v_1'^2$, ossia $\abs{v_1'} = v$.
Così, abbiamo ottenuto due soluzioni:
una in cui $v_1' = v$ e $v_2' = -v$,
e l'altra in cui $v_1' = -v$ e $v_2' = v$.
Di queste due soluzioni (matematiche) del sistema di equazioni, una ha senso fisico, l'altra no.
Infatti, la prima soluzione ($v_1' = v$ e $v_2' = -v$) descrive una situazione in cui i due carrelli si sono attraversati e poi proseguono dritti: questo è fisicamente impossibile.
La seconda soluzione invece ($v_1' = -v$ e $v_2' = v$) descrive i due carrelli che scappano via in versi opposti rispetto a come sono arrivati: questo è quello che ci aspettiamo.

%%%%%%%%%%%%%%%%%%%%%%%%%%%%%%%%%%%%%%%%%%%%%%%%%%%%%%%%%%%%%%%
\subsection{Esempio: urto elastico di due carrelli -- parte terza}
Risolviamo il sistema~\eqref{eq695bcfb8} con alcune ipotesi ulteriori.
Assumiamo che $m = m_1 = m_2$.
In questo caso, dividendo per $m$ entrambe le equazioni (NB: $m\neq0$), il sistema di equazioni~\eqref{eq695bcfb8} diventa
\begin{equation}
	\begin{cases}
	v_1 + v_2 = v_1' + v_2' , \\
	\frac12 v_1^2 + \frac12 v_2^2 = \frac12 v_1'^2 + \frac12 v_2'^2 .
	\end{cases}
\end{equation}
Dalla prima equazione otteniamo $v_2' = v_1 + v_2 - v_1'$.
Usando questa relazione nella seconda equazione, otteniamo
\begin{equation}\label{eq695bd3ad}
	\frac12 v_1^2 + \frac12 v_2^2 = \frac12 v_1'^2 + \frac12 (v_1 + v_2 - v_1')^2 .
\end{equation}
Qua sembra che dobbiamo fare un po' di conti.
Ricordiamoci che l'incognita qui è $v_1'$, tutto il resto sono quantità date dal problema.
Quindi, per aiutarci nei conti, diamo dei nomi.
Chiamiamo $A = \frac12 v_1^2 + \frac12 v_2^2$ e $B=v_1 + v_2$.
Così, l'equazione~\ref{eq695bd3ad} diventa
\begin{equation}
	A = \frac12 v_1'^2 + \frac12 (B - v_1')^2 .
\end{equation}
Svolgiamo i conti:
\begin{align}
	A 
	&= \frac12 v_1'^2 + \frac12 (B - v_1')^2 \\
	&= \frac12 v_1'^2 + \frac12 (B^2 -2Bv_1' + v_1'^2) \\
	&= ( \frac12 + \frac12) v_1'^2 - Bv_1' + \frac12 B^2 \\
	&= v_1'^2 - Bv_1' + \frac12 B^2 .
\end{align}
Quindi, l'equazione~\eqref{eq695bd3ad} è equivalente a
\begin{equation}
	v_1'^2 - Bv_1' + \frac12 B^2 - A = 0 .
\end{equation}
Questo è un polinomio di secondo grado in $v_1'$ e quindi le sue soluzioni sono:
\begin{equation}
	v_1'{}_{\pm} = \frac{ B \pm \sqrt{ B^2 - 4(\frac12 B^2 - A) } }{ 2 } 
	= \frac{ B \pm \sqrt{ 4A - B^2 } }{ 2 } .
\end{equation}
Sostituendo $A$ e $B$ con i loro valori, otteniamo
\begin{equation}
	v_1'{}_{\pm} = \frac{ v_1 + v_2 \pm (v_1 - v_2) }{ 2 } .
\end{equation}
Come prima, abbiamo ottenuto due soluzioni:
\begin{align*}
	&v_1' = v_1 \text{ e } v_2' = -v_2 , \text{ oppure } \\
	&v_1' = -v_2 \text{ e } v_2' = v_1 .
\end{align*}
La prima soluzione non ha senso fisicamente perché suppone che i due carrelli si siano attraversati.
Quindi succede la seconda: nello scontro, i due carrelli ``si scambiano le velocità''.

%%%%%%%%%%%%%%%%%%%%%%%%%%%%%%%%%%%%%%%%%%%%%%%%%%%%%%%%%%%%%%%
\subsection{Esempio: urto elastico di due carrelli -- parte quarta}
Ritorniamo al sistema di equazioni~\eqref{eq695bcfb8}, senza ipotesi ulteriori.
Denotiamo $p=m_1 v_1 + m_2 v_2$ e $E = \frac12 m_1 v_1^2 + \frac12 m_2 v_2^2$, così che~\eqref{eq695bcfb8} diventa 
\begin{equation}\label{eq695bd982}
	\begin{cases}
	P = m_1 v_1' + m_2 v_2' , \\
	E = \frac12 m_1 v_1'^2 + \frac12 m_2 v_2'^2 .
	\end{cases}
\end{equation}
Dalla prima equazione otteniamo 
\begin{equation}
	v_2' = \frac{ P - m_1 v_1' }{ m_2 } .
\end{equation}
Usando questa relazione nella seconda equazione, otteniamo
\begin{align*}
	E 
	&= \frac12 m_1 v_1'^2 + \frac12 m_2 \left(\frac{ P - m_1 v_1' }{ m_2 }\right)^2 \\
	&= \frac12 m_1 v_1'^2 + \frac12 m_2 \left(\frac{ P^2 + m_1^2 v_1'^2 - 2m_1P v_1' }{ m_2^2 }\right) \\
	&= \frac12 m_1 v_1'^2 + \frac{P^2}{2m_2} + \frac{ m_1^2 v_1'^2 }{ 2m_2 } - \frac{ m_1P v_1' }{ m_2 } \\
	&= \left( \frac12 m_1 + \frac{ m_1^2 }{ 2m_2 } \right) v_1'^2 
		-  \frac{ m_1P }{ m_2 } v_1' + \frac{P^2}{2m_2} \\
	&= \frac12 \frac{ m_1 }{ m_2 } (m_1 + m_2) v_1'^2 
		-  P\frac{ m_1 }{ m_2 } v_1' + \frac{P^2}{2m_2} .
\end{align*}
Quindi, $v_1'$ è soluzione dell'equazione quadratica:
\begin{equation}
	\frac12 \frac{ m_1 }{ m_2 } (m_1 + m_2) v_1'^2 
		-  P\frac{ m_1 }{ m_2 } v_1' + \frac{P^2}{2m_2} - E = 0,
\end{equation}
e così
\begin{align*}
	v_1'{}_\pm 
	&= \frac{ P\frac{ m_1 }{ m_2 } \pm \sqrt{ \left( P\frac{ m_1 }{ m_2 } \right)^2 - 4 \frac12 \frac{ m_1 }{ m_2 } (m_1 + m_2) \left( \frac{P^2}{2m_2} - E \right) } }{ \frac{ m_1 }{ m_2 } (m_1 + m_2) } \\
	&= \frac{ P }{ m_1+m_2 }
		\pm \frac{ \sqrt{ P^2 \left(\frac{m_1^2}{m_2^2} - 2 \frac{m_1}{m_2} (m_1+m_2) \frac1{2m_2} \right) + 2E \frac{m_1}{m_2} (m_1+m_2) } }{ { \frac{ m_1 }{ m_2 } (m_1 + m_2) } } \\
	&= \frac{ P }{ m_1+m_2 }
		\pm \frac{ \sqrt{ \frac{m_1}{m_2} (2(m_1+m_2)E-P^2 ) } }{ { \frac{ m_1 }{ m_2 } (m_1 + m_2) } } .
%	&= \frac{ P }{ m_1+m_2 }
%		\pm \frac{ \sqrt{ 2E-\frac{P^2}{m_1+m_2} } }{ \sqrt{ \frac{ m_1 }{ m_2 } (m_1 + m_2) } } \\
\end{align*}
Inserendo i valori di $E$ e $P$ di nuovo in questa espressione, otteniamo (dopo qualche conto)
\begin{equation}
	2(m_1+m_2)E-P^2
	= m_1m_2 (v_1-v_2)^2
\end{equation}
e quindi (dopo ulteriori conti)
\begin{equation*}
 	v_1'{}_\pm
	= \begin{cases}
	v_1, \text{ oppure }\\
	\frac{m_1-m_2}{m_1+m_2} v_1 + \frac{2m_2}{m_1+m_2} v_2 .
	\end{cases}
\end{equation*}
Usando queste soluzioni per $v_1'{}$, otteniamo (dopo alcuni conti)
\begin{equation}
	v_2'{}_\pm
	= \begin{cases}
	v_2, \text{ oppure }\\
	\frac{2m_1}{m_1+m_2} v_1 + \frac{m_2-m_1}{m_1+m_2} v_2 .
	\end{cases}
\end{equation}

Abbiamo due soluzioni al problema, ma solo una è fisicamente valida.
La prima soluzione (ossia $v_1' = v_1$ e $v_2'=v_2$) non è fisicamente valida perché presuppone che i due carrelli si siano attraversati.
Quindi deve succedere la seconda situazione:
\begin{equation}
	v_1' = \frac{m_1-m_2}{m_1+m_2} v_1 + \frac{2m_2}{m_1+m_2} v_2
	\quad\text{e}\quad
	v_2' = \frac{2m_1}{m_1+m_2} v_1 + \frac{m_2-m_1}{m_1+m_2} v_2 .
\end{equation}

%%%%%%%%%%%%%%%%%%%%%%%%%%%%%%%%%%%%%%%%%%%%%%%%%%%%%%%%%%%%%%%
\subsection{Altri esercizi}
\begin{enumerate}
\item Biliardo
\item Carrelli si scontrano elasticamente a valle: quanto risalgono?
\end{enumerate}







