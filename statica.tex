% !TEX encoding = UTF-8 Unicode
%!TEX root = FisMat.tex

\section{Statica}

%%%%%%%%%%%%%%%%%%%%%%%%%%%%%%%%%%%%%%%%%%%%%%%%%%%%%%%%%%%%%%%
\subsection{Introduzione}
Ci occuperemo di studiare sistemi fermi.
Per ``sistemi'' intendiamo gruppi di oggetti, come per esempio un candelabro appeso tramite una carrucola e controbilanciato da una molla.
Se un sistema è fermo, significa che è in equilibrio -- per definizione di \emph{equilibrio}.
Vogliamo trovare le condizioni che rendono equilibrato un sistema.

%\emph{equilibrio del punto materiale e del corpo rigido}.

Il nostro metodo di osservazione sarà il seguente:
\begin{enumerate}
\item	Facciamo un \emph{diagramma delle forze}, ossia un disegno del sistema con dei vettori applicati che rappresentano le forze esterne.
	A questo punto dovremo usare immaginazione e accettare un certo grado di incertezza.
\item	Per ciascun oggetto del sistema, studiamo le forze che agiscono su di esso e applichiamo le leggi di equilibrio del punto materiale e del corpo rigido.
\item	A questo punto abbiamo tutti gli elementi per dedurre ogni proprietà di equilibrio del sistema.
\end{enumerate}
Messo così, questo piano di lavoro è astratto, ma lo terremo come vademecum.

%%%%%%%%%%%%%%%%%%%%%%%%%%%%%%%%%%%%%%%%%%%%%%%%%%%%%%%%%%%%%%%
\subsection{La forza}
La \emph{forza} è una grandezza fisica vettoriale che non si vede, ma ha effetti visibili (questa non è una definizione).
La sua unità di misura è il \emph{Newton} che si denota con $\unit{N}$.
In relazione alle unità di misura fondamentali, dobbiamo ricordarci che:
\begin{equation}
	\unit{N} = \unit[per-mode = symbol]{\kilogram\,\metre\per\second^2}
\end{equation}

Esempi di forza: 
\begin{itemize}[leftmargin=*]
\item	Il \emph{peso} è una forza. 
	Noi sperimentiamo il peso costantemente sul pianeta Terra: esso è la conseguenza della nostra relazione con la Terra, che ci attrae ad essa e noi attraiamo la Terra a noi.
	Quando ci pesiamo sulla bilancia, esprimiamo questa misura in chilogrammi: tecnicamente, stiamo sbagliando. 
	Il chilogrammo $\unit{kg}$ è l'unità di misura della massa.
	Quindi se dico ``Io peso 70 chilogrammi'', dovrei invece dire ``Io ho una massa di 70 chilogrammi'', oppure ``Io peso $686{,}7$ Newton''.
	Nella vita quotidiana non è un problema, perché peso e massa sono direttamente proporzionali e questa proporzionalità non cambia mai:
	\begin{equation}
		\text{Peso}_{\text{sulla Terra}} = g \cdot \text{Massa},
	\end{equation}
	dove $g$ è l'accelerazione di gravità, 
	\begin{equation}
		g = 9{,}81\,\unit[per-mode = symbol]{\metre\per\second^2} .
	\end{equation}
\item	La Luna gira attorno alla Terra, come se Terra e Luna si tenessero per mano.
	Cosa le tiene assieme? 
	La stessa forza di gravità che tiene noi per terra.
	Che i fatti celesti siano soggetti alle stesse leggi dei fatti terrestri è una importante intuizione della fisica.
	La \emph{Legge di gravitazione universale}\footnote{La legge di gravitazione universale fu formulata da Isaac Newton nel 1687} afferma che due corpi di massa $M$ e $m$, rispettivamente, si attraggono con una forza
	\begin{equation}
		\vec F = G \frac{ M \cdot m}{r^2} ,
	\end{equation}
	dove $r$ è la distanza tra i (centri di massa dei) due corpi e $G$ è la \emph{costante gravitazionale universale}
	\begin{equation}
		G = 6{,}67\times 10^{-11}\ \unit{ \frac{ Nm^2 }{ kg^2 }} .
	\end{equation}
\item	Se schiacciamo o tiriamo una molla, la molla ci restituisce una forza, detta \emph{forza elastica}.
\item	Se una palla ci colpisce, la botta che percepiamo è una forza.
	Siccome però è una forza intensa che dura poco, quella nostra esperienza è meglio descritta dall'\emph{impulso}, che vedremo tra poco.
\item	Quando avviciniamo due magneti, l'attrazione o la repulsione tra loro è una forza, chiamata \emph{forza magnetica}.
\item	Se strofiniamo un maglione di lana e lo avviciniamo a dei pezzetti di carta, questi verranno attratti dal maglione e ci rimarranno attaccati.
	Anche questa attrazione è una forza: la \emph{forza elettrostatica}.
\item	Se proviamo a trascinare un oggetto pesante (per esempio una lavatrice), faremo un certo grado di fatica.
	Se però mettiamo l'oggetto su un carrellino munito di ruote, riusciremo a trascinare lo stesso oggetto con grande facilità.
	Senza carrellino, la \emph{forza di attrito} si opponeva al moto e noi dovevamo vincerla. 
	Con il carrellino, la forza di attrito è diminuita enormemente.
\end{itemize}


%%%%%%%%%%%%%%%%%%%%%%%%%%%%%%%%%%%%%%%%%%%%%%%%%%%%%%%%%%%%%%%
\subsection{Equilibrio del punto materiale}
Un \emph{punto materiale} è un punto adimensionale, ossia senza dimensioni spaziali, che ha una sua posizione e una sua massa.

Un punto materiale è in \emph{equilibrio}, se e solo se la somma di tutte le forze agenti su di esso è zero.

Un punto materiale è un'astrazione: non esiste un oggetto che non occupi un volume.
Però, ci sono situazioni in cui non importa il volume di un oggetto.
Per esempio, se studiamo il moto dei pianeti attorno al sole, non è importante che dimensioni abbiano e possiamo riassumere l'intero pianeta Terra, con tutti i casini che accadono su di essa, a un punto infinitesimo di massa $M_T = 5{,}972 \times 10^{24} \ \unit{kg}$.

%immaginiamo che un oggetto sia riassunto in un punto adimensionale (ossia che non occupa volume) che ha una sua posizione e una sua massa.



%%%%%%%%%%%%%%%%%%%%%%%%%%%%%%%%%%%%%%%%%%%%%%%%%%%%%%%%%%%%%%%
\subsection{Equilibrio del corpo rigido}
Un \emph{corpo rigido} è un oggetto che ha una massa e una forma rigida, ossia che non cambia.

Un corpo rigido è in \emph{equilibrio} se e solo se
\begin{enumerate}
\item	la somma di tutte le forze agenti su di esso è zero, e 
\item	la somma di tutti i momenti delle forze agenti su di esso rispetto ad un punto è zero.
\end{enumerate}

Anche il corpo rigido è un'astrazione, perché nella realtà gli oggetti cambiano forma quando sottoposti a delle forze.
In molti casi però tale deformazione è talmente piccola da potersi trascurare.

%%%%%%%%%%%%%%%%%%%%%%%%%%%%%%%%%%%%%%%%%%%%%%%%%%%%%%%%%%%%%%%
\subsection{Esempio di un sistema in equilibrio}
Consideriamo un lampadario di massa $m=3\unit{kg}$ appeso al soffitto.
Vogliamo sapere quale forza agisce sul soffitto.

Per prima cosa, facciamo un diagramma delle forze:
\begin{center}
\includegraphics[width=.75\textwidth]{pics/statica-01}
\end{center}
Sappiamo che il peso del lampadario è $P=gm$, dove $g$ è l'accelerazione di gravità,
e sappiamo che il peso del lampadario punta verso il basso.
Così abbiamo il peso come vettore $\vec P$.

In questo diagramma ci sono tre oggetti: il lampadario, il cavo che tiene il lampadario e il soffitto.
Guardiamo un oggetto alla volta.

Primo, il lampadario è fermo, quindi in equilibrio.
Questo significa che la somma delle forze che agiscono sul lampadario deve essere zero.
Quindi c'è una forza che bilancia la forza peso: ovviamente è la tensione $\vec T$ del cavo che tiene il lampadario:
\begin{center}
\includegraphics[width=.75\textwidth]{pics/statica-02}
\end{center}
Ricapitolando, siccome la somma delle forze che agiscono sul lampadario deve essere zero, abbiamo 
\begin{equation}
	\vec P + \vec T = 0 ,
\end{equation}
ossia $\vec T = - \vec P$.
Così, la tensione sul cavo è $T=P=gm$ ed è opposta alla forza peso, ossia verso l'alto.

Secondo, il cavo che tiene il lampadario è fermo, quindi la somma delle forze che agiscono su di esso deve essere zero.
\begin{center}
\includegraphics[width=.75\textwidth]{pics/statica-03}
\end{center}
Ho disegnato due forze sul cavo: una forza $\vec T_g$ ad un capo del cavo, e una forza $\vec T_s$ all'altro capo del cavo.
Siccome la somma è zero, dobbiamo avere 
\begin{equation}
	\vec T_g+\vec T_s = 0 ,
\end{equation}
ossia $\vec T_s = -\vec T_g$.

Naturalmente, $\vec T_g = -\vec T$.
Per questo motivo, abbiamo
\begin{equation}
	\vec T_s = -\vec T_g = -(-\vec T) = \vec T = -\vec P .
\end{equation}

Infine, chiamo $\vec F$ la forza che agisce sul soffitto:
\begin{center}
\includegraphics[width=.75\textwidth]{pics/statica-04}
\end{center}
Naturalmente, $\vec F = - \vec T_s$, e quindi $\vec F = \vec P$.

Concludiamo che la forza che agisce sul soffitto ha modulo $gm$ e punta verso il basso.

NB! 
Il soffitto è in equilibrio, però in questo problema non siamo interessati alle sue condizioni di equilibrio.

Questo esercizio può sembrare pedante, ma vuole mostrare come si propagano le forze grazie ai principi di equilibrio del corpo rigido.

%%%%%%%%%%%%%%%%%%%%%%%%%%%%%%%%%%%%%%%%%%%%%%%%%%%%%%%%%%%%%%%
\subsection{La Legge di Hooke}
La {Legge di Hooke} descrive la forza impressa da una molla quando compressa o allungata.
Quando una molla è a riposo ha una sua lunghezza.
Se allungata o compressa, denotiamo con $\Delta x$ la differenza rispetto alla lunghezza a riposo.
Per distinguere se questo scarto è di allungamento o compressione, definiamo il vettore $\Delta\vec x$ come il vettore che ha coda nel punto di riposo della molla e punta nella posizione attuale della molla:

\begin{center}
\includegraphics[width=.75\textwidth]{pics/statica-05}
\end{center}

La \emph{legge di Hooke} afferma che la \emph{forza elastica} $\vec F$ impressa dalla molla è
\begin{equation}
	\vec F = - k \Delta\vec x ,
\end{equation}
dove $k$ è la \emph{costante elastica} della molla.

La costante elastica dipende dalla molla e va misurata sperimentalmente.
L'unità di misura della costante elastica è $\unit{\frac{N}{m}}$.
Così, una molla con costante elastica $1\unit{N\per m}$ da una forza di un Newton per ogni metro di compressione (o allungamento).

\begin{center}
\includegraphics[width=.75\textwidth]{pics/statica-06}
\end{center}

NB!
La legge di Hooke vale solo se lo spostamento $\Delta x$ è piccolo abbastanza: 
se si allunga o contrae la molla troppo, questa subisce altri fenomeni di deformazione e quindi la legge di Hooke non vale più.
Per esempio, se si tira troppo la molla si rompe e quindi non esercita più alcuna forza.

NB!
Una molla esercita una forza su entrambi gli estremi, con la medesima intensità (ossia modulo), medesima direzione, ma verso opposto.

\begin{center}
\includegraphics[width=.75\textwidth]{pics/statica-06-1}
\end{center}

%%%%%%%%%%%%%%%%%%%%%%%%%%%%%%%%%%%%%%%%%%%%%%%%%%%%%%%%%%%%%%%
\subsection{Esercizio}
Una molla di costante elastica $k=34\unit{N/m}$ e di lunghezza a riposo $\ell=45\unit{cm}$ è dentro una scatola lunga $30\unit{cm}$.
Quali sono le forze impresse dalla molla sulle due pareti della scatola?

%%%%%%%%%%%%%%%%%%%%%%%%%%%%%%%%%%%%%%%%%%%%%%%%%%%%%%%%%%%%%%%
\subsection{Esercizio}
Nei seguenti disegni, sono rappresentate delle molle sia a riposo che sotto sforzo.
Nelle immagini delle molle sotto sforzo, disegna gli scarti $\Delta \vec x$ e le forze elastiche $\vec F$ date dalla legge di Hooke.

\begin{center}
\includegraphics[width=.75\textwidth]{pics/statica-07}
\end{center}

\begin{center}
\includegraphics[width=.75\textwidth]{pics/statica-08}
\end{center}

\begin{center}
\includegraphics[width=.75\textwidth]{pics/statica-09}
\end{center}

%%%%%%%%%%%%%%%%%%%%%%%%%%%%%%%%%%%%%%%%%%%%%%%%%%%%%%%%%%%%%%%
\subsection{Esercizio}
Appendo un corpo di massa $m=13\unit{g}$ a una molla con costante elastica $k=50\unit{N/m}$.
La molla a riposo è lunga $\ell = 7\unit{cm}$.
Quanto è lunga la molla quando il corpo è appeso?

%%%%%%%%%%%%%%%%%%%%%%%%%%%%%%%%%%%%%%%%%%%%%%%%%%%%%%%%%%%%%%%
\subsection{Esercizio}\label{par6958e97f}
Lucia e Marco hanno costruito una bilancia casalinga.
Hanno messo una molla in verticale (un tubo la tiene dritta).
In basso la molla appoggia al tavolo, in alto invece c'è una piano su cui appoggiare gli oggetti da pesare.

Lucia e Marco devono innanzitutto calcolare la costante elastica della molla.
Come fanno? 
Prova a rispondere prima di continuare a leggere.

\begin{center}
\includegraphics[width=.45\textwidth]{pics/statica-10}
\includegraphics[width=.45\textwidth]{pics/statica-11}
\end{center}

Sì, fanno così: ci appoggiano sopra un corpo di massa $m=100\unit{g}$ e misurano quanto la molla si contrae.
La loro misura da $\Delta x = 1\unit{cm}$.
Quanto vale la costante elastica $k$ della molla?

A questo punto possono provare a pesare un libro di massa $M$ (sconosciuta).
Lo appoggiano sulla bilancia e misurano che la molla si contrae di $3{,}4\unit{cm}$.
Quanto pesa il libro? Quanto vale $M$?


%%%%%%%%%%%%%%%%%%%%%%%%%%%%%%%%%%%%%%%%%%%%%%%%%%%%%%%%%%%%%%%
%\subsection{Esercizio}
%Due molle sono messe dentro a una scatola.
%Hanno costanti elastiche $k_1$ e $k_2$, rispettivamente, e lunghezze a riposo $\ell_1$ e $\ell_2$.
%La scatola è lunga $L$ e larga $H$.
%
%Possiamo mettere le due molle in due modi: in serie per il lungo o in parallelo per il largo.
%
%Se le mettiamo in serie per il lungo, quanto quali forze esercitano le molle sulle pareti della scatola?
%E se le mettiamo in parallelo per il largo?

% e supponiamo che $L<\ell_1 + \ell_2$.

%%%%%%%%%%%%%%%%%%%%%%%%%%%%%%%%%%%%%%%%%%%%%%%%%%%%%%%%%%%%%%%
\subsection{Esercizio}
Due molle identiche sono messe dentro a una scatola.
La loro costante elastica è $k$ e la loro lunghezza a riposo è $\ell$.
La scatola è lunga $L$ e larga $H$.

Possiamo mettere le due molle in due modi: in serie per il lungo o in parallelo per il largo.
Se le mettiamo in serie per il lungo, quanto quali forze esercitano le molle sulle pareti della scatola?
E se le mettiamo in parallelo per il largo?

\begin{center}
\includegraphics[width=.8\textwidth]{pics/statica-12}
\end{center}

Se preferisci avere dei numeri, considera $k=60\unit{N/m}$, $\ell = 10\unit{cm}$,
$L=15\unit{cm}$, $H=8\unit{cm}$.

%%%%%%%%%%%%%%%%%%%%%%%%%%%%%%%%%%%%%%%%%%%%%%%%%%%%%%%%%%%%%%%
\subsection{Esercizio}
Torniamo alla bilancia descritta in~\ref{par6958e97f}.
Tra Lucia e Marco nasce una diatriba.
Lucia sostiene che dentro al tubo ci siano due molle, una sopra l'altra, identiche tra loro.
Marco invece sostiene che ci sia una sola molla.
Se avesse ragione Lucia, quale sarebbe la costante elastica delle due molle?

C'è un modo, senza aprire il tubo che contiene le molle, di verificare chi dei due ha ragione?\footnote{La risposta è no. Questo è un esempio in cui possiamo fare modelli diversi per la stessa situazione pratica.
In altre parole, possiamo immaginare che dentro al tubo ci siano una o due molle (con lunghezze e costanti elastiche appropriate), e comunque dare le stesse previsioni corrette.
Se non possiamo aprire il tubo, non sapremo mai quante molle ci sono dentro.
Anzi, magari non c'è nemmeno una molla, ma un sofisticato meccanismo elettromagnetico.
}


\begin{center}
\includegraphics[width=.75\textwidth]{pics/statica-13}
\end{center}

%%%%%%%%%%%%%%%%%%%%%%%%%%%%%%%%%%%%%%%%%%%%%%%%%%%%%%%%%%%%%%%
\subsection{Esercizio ($\star$)}
Abbiamo due molle di costante elastiche $k_1$ e $k_2$, rispettivamente, e lunghezze a riposo $\ell_1$ e $\ell_2$.
Vengono messe in serie, una dopo l'altra, in una scatola di lunghezza totale $L$.
Supponiamo $L<\ell_1+\ell_2$ (quindi le molle sono compresse).
Nella scatola, misuriamo che la prima molla è ora lunga $x_1$ e la seconda molla è lunga $x_2$.
Ovviamente, $x_1+x_2 = L$.

\begin{center}
\includegraphics[width=.75\textwidth]{pics/statica-14}
\end{center}

\begin{enumerate}
\item	(Poco facile) 
	Quanto valgono $x_1$ e $x_2$ in termini dei dati sulle molle e $L$?
	(Se vuoi numeri: $k_1 = 10\unit{N/m}$, $k_2 = 25\unit{N/m}$, $\ell_1=34\unit{cm}$, $\ell_2=23\unit{cm}$, $L=49\unit{cm}$)
\item	(Meno difficile)
	Per semplificare, supponi $k_1=k_2=k$ e $\ell_1=\ell_2=\ell$. 
	Quanto valgono $x_1$ e $x_2$?
	(Se vuoi numeri: $k = 15\unit{N/m}$, $\ell=20\unit{cm}$, $L=32\unit{cm}$)
\item	(Meno difficile, ma poco facile)
	Supponi $k_2 = 2k_1$ e $\ell_2=\ell_1$.
	Quanto valgono $x_1$ e $x_2$?
	(Se vuoi numeri: $k_1 = 16\unit{N/m}$, $\ell_1=19\unit{cm}$, $L=53\unit{cm}$)
\end{enumerate}

%%%%%%%%%%%%%%%%%%%%%%%%%%%%%%%%%%%%%%%%%%%%%%%%%%%%%%%%%%%%%%%
\subsection{Esercizio ($\star\star$)}
Un lampadario è appeso al soffitto con due molle messe a triangolo.
Le molle sono identiche, con costante elastica $k$ e lunghezza a riposo $\ell$.
Il triangolo formato dalle molle ha base $L$ (lungo il soffitto) e altezza $h$.
Al vertice opposto al soffitto è appeso il lampadario, di peso $P$.

\begin{center}
\includegraphics[width=.75\textwidth]{pics/statica-15}
\end{center}

\begin{enumerate}
\item
	Queste cinque quantità $k$, $\ell$, $L$, $h$ e $P$,
	sono in relazione tra loro.
	Scrivere una formula che le mette in relazione.
	(Con questa formula, potrai rispondere velocemente a tutte le domande seguenti).
\item	
	Dati $k=125\unit{N/m}$, $\ell=1\unit{m}$, $L=1\unit{m}$, $h=1\unit{m}$,
	quanto vale $P$?
\item
	Dati $k=125\unit{N/m}$, $\ell=1\unit{m}$, $L=1\unit{m}$, $P=20\unit{N}$, quanto vale $h$?
\end{enumerate}

%%%%%%%%%%%%%%%%%%%%%%%%%%%%%%%%%%%%%%%%%%%%%%%%%%%%%%%%%%%%%%%
\subsection{Esercizio}
Due molle di costante elastica $k_1$ e $k_2$ rispettivamente, e di lunghezze a riposo $\ell_1$ e $\ell_2$, rispettivamente, sono appese dopo l'altra al soffitto, in serie.
Alla seconda molla, più in basso, viene agganciata una massa $m$.
Quanto saranno lunghe le due molle?

\begin{center}
\includegraphics[width=.75\textwidth]{pics/statica-16}
\end{center}

%%%%%%%%%%%%%%%%%%%%%%%%%%%%%%%%%%%%%%%%%%%%%%%%%%%%%%%%%%%%%%%%
%\subsection{Esercizio}
%Due molle di costante elastica $k_1$ e $k_2$ rispettivamente, e di lunghezze a riposo $\ell_1$ e $\ell_2$, rispettivamente, sono appese a fianco l'altra al soffitto, in parallelo.
%Viene agganciata una massa $m$ alle due
%Quanto saranno lunghe le due molle?
%
%\begin{center}
%\includegraphics[width=.75\textwidth]{pics/statica-16}
%\end{center}

%%%%%%%%%%%%%%%%%%%%%%%%%%%%%%%%%%%%%%%%%%%%%%%%%%%%%%%%%%%%%%%
\subsection{Esercizio}
Inventa tu un esercizio e risolvilo.

%%%%%%%%%%%%%%%%%%%%%%%%%%%%%%%%%%%%%%%%%%%%%%%%%%%%%%%%%%%%%%%
\subsection{Una questione di triangoli simili}\label{par695a342c}
Nel seguito, studieremo il piano inclinato e useremo una semplice proprietà dei triangoli simili.
Questa semplice proprietà ci permetterà di fare a meno di usare la trigonometria, presumo non si conosca ancora.

Innanzitutto, due triangoli sono simili se hanno gli stessi angoli.
Chiamiamo $\widetriangle{ABC}$ e $\widetriangle{A'B'C'}$ i due triangoli, dove a lettera uguale corrisponde angolo uguale.
È una proprietà dei triangoli simili che i lati corrispondenti siano proporzionali tutti con la stessa proporzione.
In altre parole:
\begin{equation}\label{eq695a3689}
	\frac{\overline{AB}}{\overline{A'B'}} 
	= \frac{\overline{AC}}{\overline{A'C'}}
	= \frac{\overline{BC}}{\overline{B'C'}} .
\end{equation}

\begin{center}
\includegraphics[width=.75\textwidth]{pics/statica-16-01}
\end{center}

Quindi, se ci viene dato $\widetriangle{ABC}$ e un lato di $\widetriangle{A'B'C'}$, possiamo trovare gli altri due lati di $\widetriangle{A'B'C'}$.
Per esempio, se $\widetriangle{ABC}$ è il triangolo di lati $\overline{AB}=3$, $\overline{AC}=4$ e $\overline{BC}=5$,
e se so che $\overline{A'B'} = 1$, allora 
\begin{equation}
	\overline{A'C'} = \frac{\overline{AC}}{\overline{AB}} \overline{A'B'} 
	= \frac{4}{3} ,
\end{equation}
e
\begin{equation}
	\overline{B'C'} = \frac{\overline{BC}}{\overline{AB}} \overline{A'B'}
	= \frac{5}{3} .
\end{equation}



%%%%%%%%%%%%%%%%%%%%%%%%%%%%%%%%%%%%%%%%%%%%%%%%%%%%%%%%%%%%%%%
\subsection{Il piano inclinato}
Se un oggetto, ad esempio un libro, sta su un tavolo orizzontale il peso è controbilanciato dalla forza vincolare del tavolo.
Quindi, il libro esercita una forza sul tavolo e il tavolo risponde con una forza uguale in modulo e direzione e contraria in verso. 
Questa forza del tavolo sul libro è chiamata \emph{forza vincolare}.
Il nome viene dal fatto che il tavolo è un \emph{vincolo}, ossia impedisce il movimento al libro.

\begin{center}
\includegraphics[width=.75\textwidth]{pics/statica-17}
\end{center}

Supponiamo ora che il tavolo non sia più orizzontale, ma bensì inclinato:
il libro è su un \emph{piano inclinato}.
Il peso del libro è comunque una forza diretta verso terra, verticalmente.
La forza vincolare del tavolo invece non può che essere perpendicolare al piano.
Se facciamo la somma di questi due vettori, non otteniamo zero.

\begin{center}
\includegraphics[width=.75\textwidth]{pics/statica-18}
\end{center}

Ora ti spiego un buon modo di analizzare questa situazione.
Innanzitutto, prendiamo la retta parallela al piano del tavolo e la retta perpendicolare al piano del tavolo.
Con queste due rette, scomponiamo le forze come abbiamo imparato in~\ref{par6959428e}.
Siccome la forza vincolare $\vec F$ è perpendicolare al piano, non c'è niente da scomporre.
La forza peso $\vec P$ invece si scompone nelle due componenti $\vec P_{\parallelo}$, parallela al piano del tavolo, e $\vec P_{\perp}$, perpendicolare al piano del tavolo.

\begin{center}
\includegraphics[width=.65\textwidth]{pics/statica-19}
\includegraphics[width=.34\textwidth]{pics/statica-20}
\end{center}

Il peso del libro, infatti, ha due conseguenze.
La prima è che il libro preme sul tavolo: questa forza è~$\vec P_{\perp}$.
La seconda è che il libro tende a scivolare giù lungo il piano: la causa di questo movimento è la forza~$\vec P_{\parallelo}$.

Siccome il libro non affonda nel tavolo, abbiamo
\begin{equation}
	\vec P_{\perp} + \vec F = 0 ,
\end{equation}
ossia, la forza vincolare $\vec F$ bilancia $\vec P_{\perp}$ e quindi $\vec F =  -\vec P_{\perp}$.

Cosa bilancia la forza $\vec P_{\parallelo}$?
La risposta dipende dalla situazione.
Può essere l'attrito, che vedremo tra poco, oppure un filo con un contrappeso, oppure una molla, o tutte queste e tante altre cose insieme.
Se niente bilancia la forza $\vec P_{\parallelo}$, o se non è abbastanza bilanciata, allora il libro scivolerà giù dal tavolo.

%%%%%%%%%%%%%%%%%%%%%%%%%%%%%%%%%%%%%%%%%%%%%%%%%%%%%%%%%%%%%%%
\subsection{Esempio}\label{subs695a37fb}
Un carrello sta su un piano inclinato.
Descriviamo questo piano inclinato con un triangolo rettangolo che ha cateto di base pari a $4\unit{m}$, cateto di altezza pari a $3\unit{m}$ e quindi ipotenusa pari a $5\unit{m}$.
Il carrello ha una massa $M=15\unit{kg}$.
Al carrello è agganciato un filo che, parallelo al piano inclinato, arriva in cima al piano inclinato e quindi scende verticale con una massa $m$ appesa.
Quanto deve valere $m$ per tenere in equilibrio il carrello?\\

Risolviamo l'esercizio insieme.
Innanzitutto facciamo un disegno con le forze principali:

\begin{center}
\includegraphics[width=.75\textwidth]{pics/statica-21}
\end{center}

Le forze disegnate sono:
\begin{itemize}
\item	$\vec P_p$: il peso del pesetto. 
	Il modulo di questa forza è $P_p = gm$, dove $g$ è l'accelerazione di gravità.
	Questa forza è verticale e punta verso il basso.
\item	$\vec T_p$: la tensione del filo attaccato al pesetto.
	Siccome il pesetto è fermo e le uniche forze che agiscono su di esso sono $\vec P_p$ e $\vec T_p$, possiamo già dire che $\vec P_p + \vec T_p = 0$, ossia 
		\begin{equation}
			\vec T_p = - \vec P_p .
		\end{equation} 
	Quindi il modulo di $\vec T_p$ è $T_p = gm$.
\item	$\vec P_c$: il peso del carrello.
	Il modulo di questa forza è $P_c = gM$.
	Questa forza è verticale e punta verso il basso.
\item	$\vec F_v$: la forza vincolare del piano inclinato.
	Questa forza è perpendicolare al piano inclinato, e il suo modulo dipende dal peso del carrello.
\item	$\vec T_c$: la tensione del filo attaccato al carrello.
	Siccome la tensione lungo il filo non cambia modulo\footnote{la carrucola non ha attrito},
	possiamo già dire che $T_c = T_p = gm$.
\end{itemize}

Studiamo in dettaglio la situazione delle forze sul carrello:

\begin{center}
\includegraphics[width=.5\textwidth]{pics/statica-22}
\end{center}

Tre forze agiscono sul carrello: $\vec P_c$, $\vec F_v$ e $\vec T_c$.
Siccome il carrello è in equilibrio abbiamo
\begin{equation}\label{eq695a1776}
	\vec P_c + \vec F_v + \vec T_c = 0 .
\end{equation}
Scomponiamo $\vec P_c$ nelle due componenti parallela $\vec P_{\parallelo}$ e perpendicolare $\vec P_\perp$ al piano inclinato.
Usando questa scomposizione in~\eqref{eq695a1776}, otteniamo
\begin{equation}
\begin{aligned}
	0 &= \vec P_c + \vec F_v + \vec T_c \\
	&= \vec P_{\parallelo} + \vec P_\perp + \vec F_v + \vec T_c \\
	&= (\vec P_{\parallelo} + \vec T_c ) + (\vec P_\perp + \vec F_v) .
\end{aligned}
\end{equation}
Per quello\footnotemark{} che abbiamo imparato in \ref{par695a248c},
\footnotetext{
In~\eqref{eq695a2590} abbiamo imparato che 
``prima sommo, poi scompongo $=$ prima scompongo, poi sommo''.
}
 sappiamo che 
i vettori $\vec P_{\parallelo} + \vec T_c$ e $\vec P_\perp + \vec F_v$
sono le componenti di $\vec P_c + \vec F_v + \vec T_c$ lungo le due rette.
Siccome $\vec P_c + \vec F_v + \vec T_c=0$, allora queste componenti devono essere zero, e quindi otteniamo
\begin{equation}
	\begin{cases}
	\vec P_{\parallelo} + \vec T_c &= 0 ,\\
	\vec P_\perp + \vec F_v &= 0 .
	\end{cases}
\end{equation}

Di queste due equazioni, quella che più ci interessa è la prima.
Infatti, $\vec P_{\parallelo} + \vec T_c = 0$ implica che il modulo di $\vec T_c$ 
(che sappiamo essere $T_c = gm$), è uguale al modulo di $\vec P_{\parallelo}$:
\begin{equation}\label{eq695a377b}
	gm = P_{\parallelo} .
\end{equation}
Siccome $m$ è l'incognita da determinare e siccome $\vec P_{\parallelo}$ dovremmo saperlo calcolare, questa dovrebbe darci la risposta.\\

Dobbiamo calcolare $\vec P_{\parallelo}$ e lo facciamo usando un po' di sana geometria euclidea, quella che conosciamo da almeno duemiladuecento anni.
Ho provato a riassumere in~\ref{par695a342c} quello che ci serve.

Se guardiamo bene lo schema delle forze sul carrello, e in particolare la scomposizione della forza peso, noteremo la presenza di due triangoli simili:
\begin{center}
\includegraphics[width=.45\textwidth]{pics/statica-23}
\includegraphics[width=.45\textwidth]{pics/statica-24}
\end{center}
Il triangolo piccolo è quello dato dalla scomposizione della forza peso,
quindi i suoi lati sono $P$, $P_\parallelo$ e $P_\perp$.
Il triangolo più grande invece è quello che descrive il piano inclinato, e i suoi lati sono la base $b=4\unit{m}$, l'altezza $a=3\unit{m}$ e l'ipotenusa $i=5\unit{m}$.

{\bf Esercizio:} individua gli angoli uguali.

Dopo che hai provato a individuare gli angoli uguali\footnote{Innanzi tutto, nota che in entrambi i triangoli c'è un angolo retto.
Poi, nota che $i$ e $P_\parallelo$ sono paralleli, e che $a$ e $P$ sono anche paralleli. Quindi l'angolo tra $i$ e $a$ deve essere uguale all'angolo tra $P_\parallelo$ e $P$.
Infine, rimane un angolo in ciascun triangoli, che quindi questi due angoli devono essere uguali.
Il disegno è:
\begin{center}
\includegraphics[width=.2\textwidth]{pics/statica-25}
\end{center}
}
potrai usare~\eqref{eq695a3689} e dire che
\begin{equation}
	\frac{ P_\parallelo }{ a } = \frac{ P }{ i },
\end{equation}
e quindi 
\begin{equation}\label{eq695a378d}
	P_\parallelo = \frac{ P }{ i } a 
	= \frac{gM }{5\unit{m}} 3\unit{m}
	= \frac{3}{5} g M .
\end{equation}

Finalmente possiamo concludere usando~\eqref{eq695a377b} e~\eqref{eq695a378d}:
\begin{equation}
	gm = \frac{3}{5} g M ,
\end{equation}
cioè
\begin{equation}
	m = \frac{3}{5} M 
	= \frac{3}{5} 15 \unit{kg}
	= 9 \unit{kg} .
\end{equation}

%%%%%%%%%%%%%%%%%%%%%%%%%%%%%%%%%%%%%%%%%%%%%%%%%%%%%%%%%%%%%%%
\subsection{Stesso esempio, meno parole}
Nel precedente paragrafo~\ref{subs695a37fb}, ho scritto tanto perché desidero mostrare il ragionamento nella sua interezza.
Quello è quel che dovrebbe succedere nella tua testa quando affronti un problema di fisica.
Però, quando ti chiedo di risolvere un problema in una verifica, non hai il tempo necessario per scrivere tutte quelle cose: il pensiero è molto, molto più veloce della penna (o della tastiera).
Devi essere sintetico nella scrittura, e qui ti mostro come io descriverei lo stesso problema con meno parole.
Non essere però sintetico nel pensiero!\\

Innanzitutto facciamo un disegno con le forze principali:
\begin{center}
\includegraphics[width=.75\textwidth]{pics/statica-21}
\end{center}

Le forze disegnate sono:
\begin{itemize}
\item	$\vec P_p$: il peso del pesetto.
	$P_p = gm$, $g$ è l'accelerazione di gravità.
\item	$\vec T_p$: la tensione del filo attaccato al pesetto.
	Siccome il pesetto è fermo: $\vec P_p + \vec T_p = 0$, 
	quindi $T_p = gm$.
\item	$\vec P_c$: il peso del carrello.
	$P_c = gM$.
\item	$\vec F_v$: la forza vincolare del piano inclinato.
\item	$\vec T_c$: la tensione del filo attaccato al carrello.
	Siccome la tensione lungo il filo non cambia modulo: $P_c = gM$.
\end{itemize}

Le forze sul carrello:

\begin{center}
\includegraphics[width=.5\textwidth]{pics/statica-22}
\end{center}

Siccome il carrello è in equilibrio:
\begin{equation}\label{eq695a1776}
	\vec P_c + \vec F_v + \vec T_c = 0 .
\end{equation}
Scomponiamo $\vec P_c$ nelle due componenti parallela $\vec P_{\parallelo}$ e perpendicolare $\vec P_\perp$ al piano inclinato.
Quindi:
\begin{equation}
	\begin{cases}
	\vec P_{\parallelo} + \vec T_c &= 0 ,\\
	\vec P_\perp + \vec F_v &= 0 .
	\end{cases}
\end{equation}
Quindi:
\begin{equation}\label{eq695a377b_bis}
	P_{\parallelo} = T_c .
\end{equation}


Per calcolare $\vec P_{\parallelo}$ usiamo la presenza di due triangoli simili:
\begin{center}
\includegraphics[width=.5\textwidth]{pics/statica-25}
\end{center}
%Il triangolo piccolo è quello dato dalla scomposizione della forza peso,
%quindi i suoi lati sono $P$, $P_\parallelo$ e $P_\perp$.
dove $b=4\unit{m}$,  $a=3\unit{m}$ e  $i=5\unit{m}$.

Per la proporzionalità tra lati corrispondenti in triangoli simili, otteniamo
\begin{equation}
	\frac{ P_\parallelo}{ a } = \frac{ P }{ i },
\end{equation}
e quindi 
\begin{equation}\label{eq695a378d_bis}
	P_\parallelo = \frac{ P }{ i } a 
	= \frac{gM }{5\unit{m}} 3\unit{m}
	= \frac{3}{5} g M .
\end{equation}

Finalmente possiamo concludere usando~\eqref{eq695a377b_bis} e~\eqref{eq695a378d_bis}:
\begin{equation}
	gm = \frac{3}{5} g M ,
\end{equation}
cioè
\begin{equation}
	m = \frac{3}{5} M 
	= \frac{3}{5} 15 \unit{kg}
	= 9 \unit{kg} .
\end{equation}

%%%%%%%%%%%%%%%%%%%%%%%%%%%%%%%%%%%%%%%%%%%%%%%%%%%%%%%%%%%%%%%
\subsection{Esercizio}
Un carrello sta su un piano inclinato.
Descriviamo questo piano inclinato con un triangolo rettangolo che ha cateto di base pari a $24\unit{m}$, cateto di altezza pari a $7\unit{m}$.
% e quindi ipotenusa pari a $25\unit{m}$.
Il carrello ha una massa $M=23\unit{kg}$.
Al carrello è agganciato un filo che, parallelo al piano inclinato, arriva in cima al piano inclinato e quindi scende verticale con una massa $m$ appesa.
Quanto deve valere $m$ per tenere in equilibrio il carrello?\\

%%%%%%%%%%%%%%%%%%%%%%%%%%%%%%%%%%%%%%%%%%%%%%%%%%%%%%%%%%%%%%%
\subsection{Esercizio}
Un carrello sta su un piano inclinato.
Descriviamo questo piano inclinato con un triangolo rettangolo che ha cateto di base pari a $4\unit{m}$, cateto di altezza pari a $3\unit{m}$ e quindi ipotenusa pari a $5\unit{m}$.
Il carrello ha una massa $M=20\unit{kg}$.
Al carrello è agganciata una molla con costante elastica $k=250 \unit{N/m}$.
Di quanto si allunga la molla?

\begin{center}
\includegraphics[width=.5\textwidth]{pics/statica-26}
\end{center}

%%%%%%%%%%%%%%%%%%%%%%%%%%%%%%%%%%%%%%%%%%%%%%%%%%%%%%%%%%%%%%%
\subsection{Esercizio}
Inventa un esercizio e risolvilo.


%%%%%%%%%%%%%%%%%%%%%%%%%%%%%%%%%%%%%%%%%%%%%%%%%%%%%%%%%%%%%%%
\subsection{L'attrito}
Quando un oggetto è premuto contro una superficie subisce una \emph{forza d'attrito (radente)}.
La forza d'attrito si oppone al moto con una intensità che dipende dalla forza a cui si oppone.
Per esempio, se un baule sta sul pavimento senza che nessuno lo spinga, la forza d'attrito è nulla.
Se però provo a spingerlo, inizialmente il baule non si sposta: la mia spinta viene controbilanciata dalla forza d'attrito.

Se aumento gradualmente la mia spinta, ad un certo punto il baule si sposta.
Cosa succede?
La forza d'attrito tra baule e pavimento può raggiungere un valore massimo.
Se io spingo con una forza superiore a quel valore massimo, la mia spinta vincerà sulla forza d'attrito e quindi il baule si muove.

La \emph{forza d'attrito statico massimale} $F_{a,\max}$ è direttamente proporzionale alla forza che preme l'oggetto contro la superficie, ossia
\begin{equation}
	F_{a,\max} = \mu_s F_n ,
\end{equation}
dove
$F_n$ è il modulo della forza perpendicolare che preme l'oggetto sulla superficie,
mentre la costante $\mu_s$ (``mü esse'') si chiama \emph{coefficiente d'attrito statico} e dipende dalle due superfici in contatto.
Il coefficiente d'attrito è un numero puro, ossia non ha unità di misura.

\begin{center}
\includegraphics[width=.75\textwidth]{pics/statica-27}
\end{center}

%%%%%%%%%%%%%%%%%%%%%%%%%%%%%%%%%%%%%%%%%%%%%%%%%%%%%%%%%%%%%%%
\subsection{Esempio}
Un libro sta su un tavolo orizzontale e viene premuto dalla mia mano.
Il peso del libro è $P$ mentre la forza impressa da me è $F_{\text{mano}}$.
Il coefficiente d'attrito statico tra libro e tavolo è $\mu_s$.
Quanto devo spingere il libro perché si scivoli lungo il tavolo?

La risposta è semplice.
Siccome la forza totale perpendicolare al tavolo è la somma del peso e della forza impressa dalla mia mano, ottengo
$F_n = P + F_{\text{mano}}$.
Per poter smuovere il libro devo spingere con una forza maggiore della forza d'attrito massimale, ossia
$F_{a,\max} = \mu_s F_n = \mu_s (P + F_{\text{mano}})$.

Per esempio, se $\mu_s = 0{,}23 $ il peso è $P=5\,\unit{N}$ e io premo con $F_{\text{mano}} = 4{,}5\,\unit{N}$,
allora la forza necessaria per smuovere il libro è
\begin{equation}
	F_{a,\max} = \mu_s(P + F_{\text{mano}})
	= 0{,}23 (5\,\unit{N} + 4{,}5\,\unit{N})
	= 2{,}185 \,\unit{N} .
\end{equation}
Per referenza, $2{,}185 \,\unit{N}$ è il peso di un oggetto di massa $223\,\unit{g}$ circa.

\begin{center}
\includegraphics[width=.75\textwidth]{pics/statica-28}
\end{center}


%%%%%%%%%%%%%%%%%%%%%%%%%%%%%%%%%%%%%%%%%%%%%%%%%%%%%%%%%%%%%%%
\subsection{Esercizio}
Studiamo la situazione dell'esempio in~\ref{subs695a37fb}, ma con l'aggiunta dell'attrito statico.

Un carrello sta su un piano inclinato.
Descriviamo questo piano inclinato con un triangolo rettangolo che ha cateto di base pari a $4\unit{m}$, cateto di altezza pari a $3\unit{m}$ e quindi ipotenusa pari a $5\unit{m}$.
Il carrello ha una massa $M=15\unit{kg}$.
Al carrello è agganciato un filo che, parallelo al piano inclinato, arriva in cima al piano inclinato e quindi scende verticale con una massa $m$ appesa.
Il carrello ha le ruote bloccate e quindi tra carrello e piano c'è una forza d'attrito con coefficiente d'attrito statico $\mu_s = 0{,}2$.

Quanto deve valere $m$ per tenere in equilibrio il carrello?
\\

%%%%%%%%%%%%%%%%%%%%%%%%%%%%%%%%%%%%%%%%%%%%%%%%%%%%%%%%%%%%%%%
\subsection{Esercizio $\star$}
Abbiamo due corpi di massa $m$ ciascuno appoggiati su un piano.
Il coefficiente d'attrito statico tra i corpi e la superficie è $\mu_s$.
Le due masse sono a distanza $L$ e sono unite da una molla.
La molla ha lunghezza a riposo $\ell$ e costante elastica $k_e$.
Quanto vale $L$ al massimo prima che i corpi vengano avvicinati dalla molla?
In altre parole, qual è la distanza massima $L_{\max}$ tra i due corpi per cui rimangono fermi?
\begin{center}
\includegraphics[width=.75\textwidth]{pics/statica-29}
\end{center}

Per avere dei numeri: 
$m= 0,{935} \,\unit{kg}$,
$\mu_s = 0{,}35$, 
$\ell = 10\,\unit{cm}$,
$k_e = 115\,\unit{N/m}$.









































