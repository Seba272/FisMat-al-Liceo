% !TEX encoding = UTF-8 Unicode
%!TEX root = FisMat.tex

\section{I vettori}
I vettori sembrano sconvolgere le studentesse e gli studenti a tal punto che qualcuno vuole smettere di studiare matematica e fisica.
Prima di fare scelte drastiche di cui potreste pentirvi in futuro, dovete leggere questa frase:
{\it un vettore è una freccia.}

Mi direte che la sto facendo troppo facile, ma fidatevi. 
I vettori sono frecce, e noi li useremo come tali.
Nelle prossime pagine cercheremo esaminare attentamente i vettori, introdurremo un gergo tecnico appropriato, ma non ci allontaneremmo dal concetto di freccia.
Questo lavoro serve solo ad affinare la nostra consapevolezza sulle frecce: adesso un vettore è solo una freccia, ma per la fine di questo modulo i vettori saranno frecce -- consapevolmente.

Questi sono dei vettori:

\usetikzlibrary{arrows.meta}
\tikzset{>=Latex} % Set tip of arrows as black triangle
\usetikzlibrary{decorations.pathreplacing}

\begin{center}
\begin{tikzpicture}
	\draw[->] (0,0) -- (0,2);
	\draw[->] (0,0) -- (1,3);
	\draw[->] (4,2) -- (2,1);
	\draw[->] (4,1) -- (5,3);
\end{tikzpicture}
\end{center}

%%%%%%%%%%%%%%%%%%%%%%%%%%%%%%%%%%%%%%%%%%%%%%%%%%%%%%%%%%%%%%%
\subsection{Il vettore}
Un \emph{vettore}, in quanto freccia, porta con se tre informazioni:
\begin{enumerate}
\item	La \emph{direzione}, ossia la retta su cui giace la freccia. Attenzione, due rette parallele sono la stessa direzione.
\item	Il \emph{verso}: data la retta, il vettore (la freccia) può puntare in due direzioni opposte -- di là o di qua.
\item	Il \emph{modulo}, ossia la lunghezza del vettore. Il modulo di un vettore $v$ si denota con $\norm{v}$.
\end{enumerate}

Un vettore, in quanto freccia, ha due punti speciali:
\begin{enumerate}
\item	la \emph{coda} è il punto da cui il vettore parte,
\item	la \emph{punta} è dove il vettore arriva.
\end{enumerate}

\begin{center}
\includegraphics[width=.75\textwidth]{pics/vettori-01}
%\begin{tikzpicture}
%	\draw[->] (0,0) -- (5,0);
%	\draw[decorate, decoration={brace, amplitude=10pt, raise=5pt, mirror}] 
%    (0,0) -- (5,0) 
%    node[midway, below=15pt] {modulo};
%\end{tikzpicture}
\end{center}

%%%%%%%%%%%%%%%%%%%%%%%%%%%%%%%%%%%%%%%%%%%%%%%%%%%%%%%%%%%%%%%
\subsection{Esercizio}\label{par6957d6fa}
Tra i seguenti vettori:
\begin{itemize}
\item	quali hanno la stessa direzione?
\item	quali hanno la stessa direzione e lo stesso verso?
\item	quali hanno lo stesso modulo?
\end{itemize}

Quindi, quali delle seguenti frecce sono lo stesso vettore?

\begin{center}
\includegraphics[width=.75\textwidth]{pics/vettori-02}
\end{center}

%%%%%%%%%%%%%%%%%%%%%%%%%%%%%%%%%%%%%%%%%%%%%%%%%%%%%%%%%%%%%%%
\subsection{Usanze di nome}
Se dobbiamo dare un nome a un vettore, possiamo usare una lettera qualunque, come $v$, $w$ oppure $z$, perché non siamo in dittatura quindi è nostro dovere perseguire la chiarezza, non la uniformità dei costumi.
È costume mettere una freccina sopra il nome di un vettore per indicare che stiamo parlando di un vettore.
Per esempio, $\vec v$, $\vec w$ e $\vec z$.
Questo costume risulta particolarmente utile se vogliamo parlare del modulo di un vettore senza usare le barrette: così, il modulo di $\vec v$ viene indicato con $v$ (senza freccina):
\begin{equation}
	\norm{\vec v} = v .
\end{equation}

Questa usanza è tipica in fisica, ma meno in matematica.
Io cercherò di adottarla il più possibile, anche se sono un matematico.

%%%%%%%%%%%%%%%%%%%%%%%%%%%%%%%%%%%%%%%%%%%%%%%%%%%%%%%%%%%%%%%
\subsection{Il vettore applicato}
Un \emph{vettore applicato} contiene una informazione in più: dove sta la coda, ossia il \emph{punto di applicazione}.
Così, nell'esercizio~\ref{par6957d6fa} abbiamo visto che lo stesso vettore può comparire in posti diversi del foglio.
Un vettore applicato sta invece in un punto preciso.

%%%%%%%%%%%%%%%%%%%%%%%%%%%%%%%%%%%%%%%%%%%%%%%%%%%%%%%%%%%%%%%
\subsection{Esercizio}
Abbiamo tre vettori, che chiamiamo $\vec v$, $\vec w$ e $\vec z$, e due punti che chiamiamo $P$ e $Q$.
Disegna i vettori applicati $(\vec v,P)$, $(\vec w,Q)$, e $(\vec v,Q)$.
In altre parole, applica il vettore $\vec v$ al punto $P$, il vettore $\vec w$ al punto $Q$ e il vettore $\vec v$ al punto $Q$.

\begin{center}
\includegraphics[width=.75\textwidth]{pics/vettori-03}
\end{center}

%%%%%%%%%%%%%%%%%%%%%%%%%%%%%%%%%%%%%%%%%%%%%%%%%%%%%%%%%%%%%%%
\subsection{Somma di vettori}\label{par6957e2da}
I vettori si possono sommare con il così detto metodo del ``punto-coda''.
Quindi, se abbiamo due vettori $\vec v$ e $\vec w$, la somma $\vec v+\vec w$ è un vettore ottenuto così:
\begin{enumerate}
\item	si fissa un punto e si applica $\vec v$ a questo punto;
\item 	si applica $\vec w$ alla punta di $\vec v$;
\item 	il vettore $\vec v+\vec w$ è il vettore con la coda nella coda di $\vec v$ e la punta nella punta di $\vec w$.
\end{enumerate}

\begin{center}
\includegraphics[width=.75\textwidth]{pics/vettori-04}
\end{center}

Si noti che $\vec v+\vec w$ e $\vec w+\vec v$ sono uguali!

%%%%%%%%%%%%%%%%%%%%%%%%%%%%%%%%%%%%%%%%%%%%%%%%%%%%%%%%%%%%%%%
\subsection{Esercizio}
Disegnare dei vettori e farne la somma.
Disegnare quattro vettori $\vec v_1$, $\vec v_2$, $\vec v_3$ e $\vec v_4$ e fare la somma $\vec v_1+\vec v_2+\vec v_3+\vec v_4$.

%%%%%%%%%%%%%%%%%%%%%%%%%%%%%%%%%%%%%%%%%%%%%%%%%%%%%%%%%%%%%%%
\subsection{Moltiplicazione di un vettore per uno scalare}\label{par6957e5e4}
La parola ``\emph{scalare}'' è un sinonimo di ``numero'' nel gergo dei vettori.
Possiamo moltiplicare un vettore $\vec v$ per un numero reale qualunque $t\in\R$.

Per esempio, $2\vec v$ è $\vec v+\vec v$, e $3\vec v=\vec v+\vec v+\vec v$, mentre $\frac12\vec v$ è metà di $\vec v$.

Quindi, dati un vettore $\vec v$ e uno scalare $t\in\R$, costruiamo il vettore $t\vec v$ così:
\begin{enumerate}
\item	$t\vec v$ ha la stessa direzione di $\vec v$;
\item	se $t=0$, $t\vec v$ è un punto; se $t>0$, $t\vec v$ ha lo stesso verso di $\vec v$; se $t<0$, $t\vec v$ ha verso opposto di $\vec v$;
\item	$t\vec v$ ha modulo $|t|\cdot\norm{\vec v}$.
\end{enumerate}


In particolare, $-\vec v$ è il vettore ``opposto'' a $\vec v$.
Giustamente, $\vec v+(-\vec v) = 0$.

\begin{center}
\includegraphics[width=.75\textwidth]{pics/vettori-05}
\end{center}

%%%%%%%%%%%%%%%%%%%%%%%%%%%%%%%%%%%%%%%%%%%%%%%%%%%%%%%%%%%%%%%
\subsection{Esercizio}
Disegnare tre vettori, scegliere tre numeri, e fare almeno tre moltiplicazioni scalare per vettore.

%%%%%%%%%%%%%%%%%%%%%%%%%%%%%%%%%%%%%%%%%%%%%%%%%%%%%%%%%%%%%%%
\subsection{Vettori nel piano cartesiano}\label{par6957e126}
Possiamo descrivere ciascun vettore nel piano cartesiano usando due numeri, chiamati \emph{coordinate del vettore}:
\begin{enumerate}
\item	Applichiamo il vettore all'origine, ossia mettiamo la coda del vettore nel punto $(0,0)$;
\item	I due numeri che descrivono il vettore sono le coordinate (ascissa e ordinata) della punta del vettore.
\end{enumerate}


\begin{center}
\includegraphics[width=.75\textwidth]{pics/vettori-06}
\end{center}

Se il vettore è applicato altrove, questi due numeri comunque determinano, data la coda del vettore, dove cadrà la punta del vettore.

\begin{center}
\includegraphics[width=.75\textwidth]{pics/vettori-07}
\end{center}


%%%%%%%%%%%%%%%%%%%%%%%%%%%%%%%%%%%%%%%%%%%%%%%%%%%%%%%%%%%%%%%
\subsection{Esercizio}\label{par6957e1d8}
Disegna i vettori con le seguenti coordinate:
$\vec v=(1,1)$, $\vec w=(0,3)$, $\vec u=(2,-1)$, $\vec k=(-1,-1)$.

Poi, disegna tu tre vettori e determinane le coordinate.

%%%%%%%%%%%%%%%%%%%%%%%%%%%%%%%%%%%%%%%%%%%%%%%%%%%%%%%%%%%%%%%
\subsection{Coordinate di un vettore applicato}
Questo concetto lo abbiamo già visto in~\ref{par6957e126}, ma vale la pena sottolinearlo.

Un vettore nel piano cartesiano è determinato da due numeri, le sue due coordinate.
Un vettore applicato nel piano cartesiano ha bisogno del doppio di coordinate: due coordinate del vettore e due coordinate per il suo punto di applicazione.

Domanda: dato un vettore $\vec v=(a,b)$ applicato nel punto $(x,y)$, la sua coda è nel punto $(x,y)$ per definizione, ma dove sta la sua punta?

Domanda: possiamo immaginare vettori anche nello spazio 3D.
Quante coordinate ci servono per determinare un vettore nello spazio tridimensionale?
Quante coordinate per un vettore applicato nello spazio tridimensionale?

%%%%%%%%%%%%%%%%%%%%%%%%%%%%%%%%%%%%%%%%%%%%%%%%%%%%%%%%%%%%%%%
\subsection{Esercizio}
Disegna i vettori $\vec v$, $\vec w$, $\vec u$ e $\vec k$ dall'Esercizio in~\ref{par6957e1d8} applicati al punto $P=(2,3)$.

%%%%%%%%%%%%%%%%%%%%%%%%%%%%%%%%%%%%%%%%%%%%%%%%%%%%%%%%%%%%%%%
\subsection{Somma di vettori in coordinate}
Abbiamo visto come si sommano i vettori usando il metodo ``punta-coda'' in~\ref{par6957e2da}.
Con le coordinate, la somma di vettori diventa molto facile: basta sommare i numeri!
Per esempio, se abbiamo due vettori $\vec v = (v_x,v_y)$ e $\vec w=(w_x,w_y)$, possiamo trovare la somma $\vec v+\vec w$ facendo
\begin{equation}
	\vec v + \vec w 
	= (v_x,v_y) + (w_x,w_y)
	= (v_x + w_x, v_y + w_y) .
\end{equation}
Per esempio,
\begin{equation}
	(2,4) + (1,2) = (3,6) .
\end{equation}

A questo punto la lettrice diligente dovrebbe riflettere sul perché sommare le coordinate sia equivalente a eseguire il metodo ``punto-coda''.

%%%%%%%%%%%%%%%%%%%%%%%%%%%%%%%%%%%%%%%%%%%%%%%%%%%%%%%%%%%%%%%
\subsection{Esercizio}
Si prendano i vettori dall'esercizio in~\ref{par6957e1d8} e si facciano almeno tre somme, sia con il disegno che con le coordinate.

%%%%%%%%%%%%%%%%%%%%%%%%%%%%%%%%%%%%%%%%%%%%%%%%%%%%%%%%%%%%%%%
\subsection{Moltiplicazione scalare per vettore in coordinate}
Abbiamo visto in~\ref{par6957e5e4} come moltiplicare un vettore per uno scalare.
Se si hanno le coordinate del vettore, questa moltiplicazione è semplicemente la moltiplicazione delle coordinate!

Per esempio, se abbiamo un vettore $\vec v = (v_x,v_y)$ e uno scalare $t\in\R$, allora
\begin{equation}
	t\vec v
	= t(v_x,v_y)
	= (t v_x, t v_y) .
\end{equation}
Per esempio, 
\begin{equation}
	2 (2,-3) = (4,-6) . 
\end{equation}

Si guardi con attenzione la seguente riga ti conti:
\begin{equation}
	\vec v + \vec v = (v_x,v_y) + (v_x,v_y) = (v_x + v_x, v_y + v_y) = (2v_x,2v_y) = 2\vec v .
\end{equation}

%%%%%%%%%%%%%%%%%%%%%%%%%%%%%%%%%%%%%%%%%%%%%%%%%%%%%%%%%%%%%%%
\subsection{Esercizio}
Si prendano tre numeri reali e tre vettori in coordinate e si facciano almeno tre moltiplicazioni scalare per vettore.

%%%%%%%%%%%%%%%%%%%%%%%%%%%%%%%%%%%%%%%%%%%%%%%%%%%%%%%%%%%%%%%
\subsection{Il modulo in termini delle coordinate}
Il modulo di un vettore si può ricavare dalle coordinate usando il teorema di Pitagora:\footnote{Il Teorema di Pitagora dice che: ``In ogni triangolo rettangolo, la somma dei quadrati costruiti sui cateti è uguale al quadrato costruito sull'ipotenusa.'' Qui per ``quadrato'' si intende ``l'area del quadrato''.}
se $\vec v=(a,b)$, allora $v=\norm{\vec v} = \sqrt{a^2 + b^2}$.

\begin{center}
\includegraphics[width=.75\textwidth]{pics/vettori-08}
\end{center}

%%%%%%%%%%%%%%%%%%%%%%%%%%%%%%%%%%%%%%%%%%%%%%%%%%%%%%%%%%%%%%%
\subsection{Esercizio}
Calcola il modulo dei vettori $\vec v$, $\vec w$, $\vec u$ e $\vec k$ dall'Esercizio in~\ref{par6957e1d8}.

%%%%%%%%%%%%%%%%%%%%%%%%%%%%%%%%%%%%%%%%%%%%%%%%%%%%%%%%%%%%%%%
\subsection{Esercizio}
Io ti do due vettori:
\begin{equation}
\begin{aligned}
	\vec e_1 &= (1,0) ,\\
	\vec e_2 &= (0,1) .
\end{aligned}
\end{equation}
Trova due scalari $a,b\in\R$ tali per cui
\begin{equation}
	a \vec e_1 + b \vec e_2 = (4,-5) .
\end{equation}

Possiamo interpretare questo problema così: 
immagina che i due vettori $\vec e_1$ e $\vec e_2$ siano le uniche due direzioni in cui puoi muoverti.
Se ti muovi in direzione $\vec e_1$ per una distanza $a$, significa che dal punto in cui sei vai fino alla punta di $a\vec e_1$.
Quindi, partendo dall'origine $(0,0)$ devi arrivare al punto $(4,-5)$ seguendo un po' $\vec e_1$ e un po' $\vec e_2$.
Il problema ti chiede di trovare quanto devi seguire l'uno e l'altro vettore.

%%%%%%%%%%%%%%%%%%%%%%%%%%%%%%%%%%%%%%%%%%%%%%%%%%%%%%%%%%%%%%%
\subsection{Esercizio}
Io ti do due vettori:
\begin{equation}
\begin{aligned}
	\vec v &= (2,1) ,\\
	\vec w &= (-1,2) .
\end{aligned}
\end{equation}
Trova due scalari $a,b\in\R$ tali per cui
\begin{equation}
	a \vec e_1 + b \vec e_2 = (5,-2) .
\end{equation}




